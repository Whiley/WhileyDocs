\chapter{Assertion Language}

The Whiley Compiler (WyC) checks the correctness of Whiley programs by
first converting them into assertions written in the Whiley Assertion
Language (WyAL).  Then, it checks them using the Whiley Theorem Prover
(WyTP) and, if all of the generated assertions (known as {\em
  verification conditions}) hold, then the original program was
correct.  This chapter provides a high-level introduction to the
Whiley Assertion Language (WyAL).  The purpose is to help the reader
understand {\em what} the system is trying to achieve without going
into details about {\em how} it is achieved.  To that end, some
discussion of how this relates to the Whiley language is provided.

\section{Overview}
The fundamental construct in WyAL is the \clstinline{assert} statement.
The following illustrates a very simple example:
\begin{tcolorbox}\begin{lstlisting}[language=WyAL]
assert:
  forall(int x):
     if:
       x > 0
     then:
       x >= 0
\end{lstlisting}\end{tcolorbox}
An assertion is simply a statement whose truth will be checked by the
Whiley Theorem Prover (WyTP).  If the above were stored in a file
\clstinline{test.wyal}, then we could perform this check as follows:
\begin{verbatim}
% wy verify test.wyal
\end{verbatim}
The tool accepts this without producing an error message, thus
indicating that it believes this statement to be correct.  To
understand what happens when an incorrect assertion is given, consider
the following:
\begin{tcolorbox}\begin{lstlisting}[language=WyAL]
assert:
  forall(int x):
     if:
       x >= 0
     then:
       x > 0
\end{lstlisting}\end{tcolorbox}

This assertion is incorrect because if, for example,
\clstinline{x == 0} then it is clearly not the case that
\clstinline{x > 0}.  As such, we have found a {\em counter-example}
to our assertion.  Running WyTP identifies the problem:
\begin{tcolorbox}\begin{lstlisting}
./test.wyal:1: assertion failure
assert:
^^^^^^^
\end{lstlisting}\end{tcolorbox}

Assertions in WyAL are simply formulae written in an extended form of
first-order logic.  The goal of WyTP is simply to ``decide'' whether
or not such formulae are {\em satisfiable} or {\em unsatisfiable}.  


\section{Syntax}
As discussed above, the primary construct in WyAL is the
\clstinline{assert} statement.  We now explore the syntax of
\clstinline{assert} statements in an informal fashion.


\subsection{Built-In Types}

The set of types provided in WyAL exactly matches those found in
Whiley.  This includes: {\em primitive types} (i.e. \clstinline{null},
\clstinline{bool}, \clstinline{int} and \clstinline{any}; {\em array
    types} (e.g. \clstinline{int[]}); {\em record types}
  (e.g. \clstinline{\{int x, int y\}}); {\em union, intersection and
    negation connectives} (e.g. \clstinline{int\|null}); and,
  {\em reference types} (e.g. \clstinline{&int}).  We now explore these
  in more detail.

  \paragraph{Array types.} These have the general form \clstinline{T[]}
  where \clstinline{T} is the {\em element type}.  They support the
  {\em array access} (e.g. \clstinline{xs[i]}) and {\em array length}
  (e.g. \clstinline{\|xs\|}) operators , and provide the {\em array
    initialiser} (e.g. \clstinline{[1,2,3]}) and {\em array generator}
  constructors (e.g. \clstinline{[0;n]}).  The following illustrates a
  simple assertion involving arrays:

\begin{tcolorbox}\begin{lstlisting}[language=WyAL]
assert:
   forall(int[] xs, int i):
      if:
         |xs| == 1
         xs[i] == 0
      then:
         xs == [0]
\end{lstlisting}\end{tcolorbox}

This simple assertion employs both array access and length operators,
as well as the array initialiser constructor.

\paragraph{Record types.} These have the general form \clstinline{\{T1 f1,...,Tn fn\}} where \clstinline{f1} ... \clstinline{fn} denote the
{\em field names}.  They support the {\em record access} operator and
provide the {\em record initialiser} constructor.  The following
illustrates a simple assertion involving records:

\begin{tcolorbox}\begin{lstlisting}[language=WyAL]
assert:
   forall({int x, int y} point):
      if:
         point.x == 0
         point.y == 0
      then:
         point == {x:0, y:0}
\end{lstlisting}\end{tcolorbox}

This employs both the record access operator
(e.g. \clstinline{point.x}) and record initialiser constructor
(e.g. \clstinline{\{x:0, y:0\}}).

\paragraph{Union Types.}  These have the general form \clstinline{T1 \| ... \| Tn} where \clstinline{T1} ... \clstinline{Tn} denote the
{\em element types}.  A variable of union type may hold any value
permitted by one of its element types.  Unions support the {\em type
  test} operator.  The following illustrates a simple assertion
involving unions:
\begin{tcolorbox}\begin{lstlisting}[language=WyAL]
assert:
   forall(int|null x, int y):
      if:
         x is int
         x > y
      then:
         x != y
\end{lstlisting}\end{tcolorbox}
The positioning of \clstinline{x is int} is important here.  For
example, if \clstinline{x > y} came before \clstinline{x is int}
then the above would not compile.  This is because the type checker
only assumes that \clstinline{x} has type \clstinline{int} after it sees
\clstinline{x is int}, but not before.

\subsection{User-Defined Types}
 
The Whiley Assertion Language supports {\em user-defined types} which
allow names to be given to types.  The following illustrates a simple
example:
\begin{tcolorbox}\begin{lstlisting}[language=WyAL]
type nat is (int x) where x >= 0

assert:
   forall(nat x):
      x >= 0
\end{lstlisting}\end{tcolorbox}
Here, a new type \clstinline{nat} has been defined to represent the
{\em naturals} (i.e. the non-negative integers).  We say that
\clstinline{nat} is a {\em constrained} type because it includes a type
invariant (i.e. \clstinline{x >= 0}).  We note that type invariants
are optional.

In addition to allowing the definition of constrained types, we can
also define {\em recursive} types.  The following illustrates a simple
example:

\begin{tcolorbox}\begin{lstlisting}[language=WyAL]
type LinkedList is (null | { LinkedList next, nat data } l)

assert:
   forall(LinkedList x):
      if:
         !(x is null)
      then:
          x.data >= 0
\end{lstlisting}\end{tcolorbox}
This defines a linked list where each node holds a field
\clstinline{data} of type \clstinline{nat}.  Observe how the type test
operator is used to distinguish between a leaf node
(i.e. \clstinline{null}) and a non-leaf node.

\subsection{Blocks, Statements \& Expressions}

% Talk about expression versus block syntax.
% Talk about different kinds of statements supported.

In WyAL a distinction is made between {\em statements} and {\em
  expressions}.  The purpose of this distinction is purely syntactic
as every statement corresponds to an equivalent expression.
Statements aid readability of complex assertions and also help
connect components of the assertion with the original Whiley program.
The following illustrates a conditional in expression form:

\begin{tcolorbox}\begin{lstlisting}[language=WyAL]
assert:
   forall(int x):
       (x > 0) ==> (x >= 0)
\end{lstlisting}\end{tcolorbox}

This employs an implication expression rather than an \clstinline{if}
statement (as seen before).  As another example, consider the
following:

\begin{tcolorbox}\begin{lstlisting}[language=WyAL]
assert:
   forall(int[] xs, int i):
      if:
         (i >= 0) && (i < |xs|)
         forall(int i).(xs[i] >= 0)
      then:
         xs[i] >= 0
\end{lstlisting}\end{tcolorbox}

This illustrates a mixture of statement and expression forms.  For
example, the \clstinline{forall} statement is shown in both statement
and expression forms.  Likewise, multiple lines are used in the
condition for the \clstinline{if} statement, and these constitute a
statement block.  Furthermore, statements within a block are always
conjuncted together.

\subsubsection{Operator Precedence}

Expressions in WyAL do not support operator precedence.  This does not
mean that all operators have equal precedence but, rather, that {\em
  ambiguous expressions are not permitted}.  For example,
\clstinline{x < 0 \|\| x > 0} is syntactically incorrect as there is
ambiguity between the precedence of the three operators involved.
Instead, this expression should be written as \clstinline{(x < 0) \|\|
  (x > 0)} to resolve the ambiguity (for example).

\subsection{Macros}

WyAL permits the definition of {\em macros} which are expanded at
their call site.\footnote{Note, recursive macros are not currently
  permitted}  Macros are helpful for grouping conditions together and
reusing them.  The following illustrates a typical usage:

\begin{tcolorbox}\begin{lstlisting}[language=WyAL]
define invariant(int i) where:
   i >= 0

assert:
   forall(int i):
       if:
          i == 0
       then:
          invariant(i)
\end{lstlisting}\end{tcolorbox}

Such an assertion may have been generated whilst checking a Whiley
program consisting of the following snippet:
\begin{tcolorbox}\begin{lstlisting}[language=Whiley]
  ...
  int i = 0
  while i < |xs| where i >= 0:
     ...
\end{lstlisting}\end{tcolorbox}

Here, macro \lstinline{invariant()} represents the loop invariant
given above.  Thus, we see how macros can be used to aid in connecting
the original Whiley program with the generated assertion.

\subsection{Uninterpreted Functions}

WyAL permits the definition of {\em uninterpreted functions}.  These
introduce function signatures which can be used within expressions.
The following illustrates:
\begin{tcolorbox}\begin{lstlisting}[language=WyAL]
function id(int x)->(int r)

assert:
   forall(int x, int y):
      (x == y) ==> (id(x) == id(y))
\end{lstlisting}\end{tcolorbox}

The body of function \clstinline{id()} is unknown to the verifiers and,
hence, is referred to as {\em uninterpreted}.  This reflects what the
verifier is permitted to know when verifying programs in Whiley.  The
only axiom known about uninterpreted functions is that, when given the
same arguments, they produce the same result.

\section{Verification Condition Generation}

When verifying the correctness of a Whiley program, the Whiley
Compiler generates assertions to be checked by WyTP.  These assertions
are known as {\em verification conditions}, and the process of
creating them as {\em verification condition generation}.  We now
briefly examine this process as it provides valuable insight into the
purpose of WyTP.

As a first example, consider the following Whiley implementation of
the well-known {\em absolute} function:

\begin{tcolorbox}\begin{lstlisting}[language=Whiley]
function abs(int x) -> (int r)
ensures r >= 0
ensures (r == x) || (r == -x):
   //
   if x >= 0:
      return x
   else:
      return -x
\end{lstlisting}\end{tcolorbox}

This has two postconditions denoted by the \clstinline{ensures}
clauses: the first states that the return value cannot be negative;
the second states that the return value is either \clstinline{x} or its
negation.  Both of these conditions must be true of {\em any possible}
return value for the function to be considered correct.

The Whiley Compiler performs verification condition generation by
traversing the \gls{control_flow_graph} of the function in a
path-sensitive fashion.  When a branch is encountered, the generator
forks and proceeds independently down each path.  In this case, four
verification conditions will be generated (i.e. since there are two
\clstinline{ensures} clauses and two execution paths).  The following
two verification conditions are generated for the first clause by each
execution path:

\begin{center}
\begin{tabular}{c c c}
\begin{minipage}[t]{0.45\textwidth}
\begin{tcolorbox}\begin{lstlisting}[language=WyAL]
assert:
  forall(int x):
    if:
      x >= 0
    then:
      x >= 0
\end{lstlisting}\end{tcolorbox}
\end{minipage}&&
\begin{minipage}[t]{0.45\textwidth}
\begin{tcolorbox}\begin{lstlisting}[language=WyAL]
  assert:
    forall(int x):
      if:
        x < 0
      then:
        -x >= 0
\end{lstlisting}\end{tcolorbox}
\end{minipage}\\
\end{tabular}
\end{center}
The left assertion corresponds to the execution path through the true
branch of the \clstinline{if} statement in the original Whiley
function; likewise, the right assertion corresponds to the path
through the false branch.

\subsection{Arrays}

Arrays provide an interesting example as the array access operator
causes verification conditions to be generated.  Specifically, such
verification conditions are needed to ensure that all array accesses
are within bounds.  The following illustrates a simple example in
Whiley:

\begin{tcolorbox}\begin{lstlisting}[language=Whiley]
type Point is {int x, int y}

function getPointX(int i, Point[] points) -> (int r)
requires (i >= 0) && (i < |points|):
    return points[i].x;
\end{lstlisting}\end{tcolorbox}

This simply extracts the \clstinline{x} component of a given
\clstinline{Point} in an array.  The array access generates two
verification conditions, including the following:
\begin{tcolorbox}\begin{lstlisting}[language=WyAL]
type Point is ({int x,int y} p)

define getPointX_requires(int i, Point[] points) is:
   (i >= 0) && (i < |points|)

assert:
  forall(int i, Point[] points):
    if:
      getPointX_requires(i,points)
    then:
      i < |points|
\end{lstlisting}\end{tcolorbox}

This assertion checks that, if the precondition of function
\clstinline{getPointX()} holds, then \clstinline{i} is less than the
length of \clstinline{points}.  The other verification condition
generated is very similar, and ensures \clstinline{i} is not negative.

\subsection{Loops}

The process of generating verification conditions for loop is rather
complex, and we will only give a brief overview here.  The main point
is that a given loop invariant will generate two verification
conditions: the first for checking the loop invariant holds on entry
to the loop; the second for checking the loop invariant holds at end
of the loop body, assuming it did at the start.  The following
illustrates a very simple example:

\begin{tcolorbox}\begin{lstlisting}[language=Whiley]
function counter(int n) -> (int r):
   //
   int i = 0
   //
   while i < n where i >= 0:
      i = i + 1
   //
   return i
\end{lstlisting}\end{tcolorbox}

This function simply counts up from \clstinline{0} to a given number
\clstinline{n}.  It generates two verification conditions which look
roughly as follows:

\begin{center}
\begin{tabular}{c c c}
\begin{minipage}[t]{0.45\textwidth}
\begin{tcolorbox}\begin{lstlisting}[language=WyAL]
define inv(int x) where:
  x >= 0

assert:
  forall(int i, int n):
    if:
      i == 0
    then:
      inv(i)

\end{lstlisting}\end{tcolorbox}
\end{minipage}&&
\begin{minipage}[t]{0.45\textwidth}
\begin{tcolorbox}\begin{lstlisting}[language=WyAL]
define inv(int x) where:
  x >= 0

assert:
  forall(int i, int n):
    if:
      inv(i)
      i < n
    then:
      inv(i+1)
\end{lstlisting}\end{tcolorbox}
\end{minipage}\\
\end{tabular}
\end{center}

The left assertion checks that the loop invariant holds on entry to
the loop, whilst the right checks that it holds at the end of the body
assuming it did at the start.  The second assertion includes the loop
condition as this is known to hold within the loop body.

% \subsection{References}

% \section{Proofs}

% We can examine the tools ``proof'' of this as follows:
% \begin{tcolorbox}
% \begin{verbatim}
% % wy verify --proof test.wyal

%  36. exists(int x).(((((x + 1) >= ((0 + 1) + 1))) ...    () 
%  41. exists(int x).((x >= 1) && (0 >= (1 + x)))   (Simp 36) 
%  40. (x >= 1) && (0 >= (1 + x))               (Exists-E 41) 
%  37. x >= 1                                      (And-E 40) 
%  39. 0 >= (1 + x)                                (And-E 40) 
%  44. -1 >= 1                                  (Ieq-I 39,37) 
%  20. false                                        (Simp 44) 
% \end{verbatim}
% \end{tcolorbox}
% This proof simplify identifies the rules applied by WyTP in the order
% they were applied.  For now, the details of this are not so important
% and we will return to this later.
