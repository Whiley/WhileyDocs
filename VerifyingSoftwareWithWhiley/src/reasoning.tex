\chapter{Simple Reasoning}

Introduction to reasoning about a function in a forwards or backwards direction.  For forwards direction, need a staged approach where we eventually introduce the concept of a skolem.  Cover control-flow constructs upto, but not including loops.  This includes if-conditions, switch statements, function calls, array assignments, reference assignments, etc.  Also, introduce basic concept of a weakest precondition versus a strongest postcondition.  Introduce control-flow graph 
and the path sensitive traversal?

\section{Functions}

Begin with pre / post-conditions and return statements.

\section{Assignments}

Examine assignments.

\section{Conditionals}
Examine simple examples involving conditions, such as \lstinline{abs()} and \lstinline{max()}

\section{Function Calls}
Examine involving function calls

\section{Compound Data Types}
Examine examples with updating lists, records, etc.

\section{Complex Reasoning}
Examine problem with our simplistic way of thinking about assignments.

\section{Forwards versus Backward Reasoning}
Discuss the so-called weakest precondition transformer.

