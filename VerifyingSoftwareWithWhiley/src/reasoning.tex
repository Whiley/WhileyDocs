\chapter{Reasoning About Programs}
In the previous Chapter, we learned how to write specifications for our programs in the form of pre-/post-conditions and invariants.  The purpose of this was to allow us to state more clearly what our programs are supposed to do.  An important step here lies in checking that our functions do indeed meet their specifications.  In general, this is a tricky and time consuming process.  Fortunately, the Whiley system takes much of the labour out of this process and can automatically that our programs meet their specifications.  

In this chapter, we begin the process of understanding how to check that our programs meet their specifications.  To this end, we consider two approaches: {\em forwards reasoning} and {\em backwards reasoning}.  Whilst neither of these is strictly better than the other, you will most likely find forward reasoning more natural.  Whilst this might seem to make backward reasoning redundant, we will find that it can still be very useful in Chapter~\ref{c_reasoning_loops}.  In this chapter, we will largely ignore functions which contains {\em loops} and/or {\em recursion} because these constructs present additional challenges and will be discussed later in Chapters~\ref{c_reasoning_loops} and~\ref{c_reasoning_recursion}.  This does not mean the programs considered in this chapter are uninteresting and, indeed, they may still contain a range of important constructs, including: {\em if statements}, {\em switch statements}, {\em list assignments}, {\em indirect assignments}, etc.

{\em Also, introduce basic concept of a weakest precondition versus a strongest postcondition.  Introduce control-flow graph and the path sensitive traversal?}

\section{Forward Reasoning}
\label{c_reasoning_forward}

As a very simple example, consider the following function which accepts a positive integer and returns a
non-negative integer (i.e. a natural number):
\begin{lstlisting}
function decrement(int x) => (int y) 
// Parameter x must be greater than zero
requires x > 0
// Return must be greater or equal to zero
ensures y >= 0:
    //
    return x - 1
\end{lstlisting}
Here, the \lstinline{requires} and \lstinline{ensures} clauses define the function's precondition and postcondition.  With verification enabled, the Whiley compiler will verify that the implementation of this function meets its specification.  In fact, we can see this for ourselves by manually constructing an appropriate {\em verification condition} (that is, a logical condition whose truth establishes that the implementation meets its specification).  In this case, the appropriate verification condition is \lstinlinec{x > 0 ==> x-1 >= 0}.  Unfortunately, although constructing a verification condition by hand was possible in this case, in general it's difficult if not impossible for more complex functions.

\subsection{Assignments}

We can reason simply about assignments by applying their effect to what we currently know.  The following illustrates:

\begin{lstlisting}
// y < 0 
x = y + 1
// y < 0 $\land$ x = y + 1
\end{lstlisting}

Here, we know \lstinlinec{y<0} is true before the assignment and that \lstinlinec{x == y + 1} must be true after it.  Furthermore, since \lstinlinec{y} is unaffected by the assignment it follows that \lstinlinec{y < 0} continues to hold.  From this, we can conclude, for example, that \lstinlinec{x < 1} is true afterwards.  

\subsection{Conditionals}

Conditionals introduce information about variables into the program.  Consider the following snippet:

\begin{lstlisting}
if x >= 0:
   // x $\ge$ 0
   ...
else:
   // x < 0
   ...
\end{lstlisting}

Here, we know nothing about variable \lstinlinec{x} before the statement.  However,, the conditional provides gives us more information.  On the true branch, we have that \lstinlinec{x >= 0} which follows immediately from the condition itself.  On the false branch, we have \lstinline{x < 0} which follows from the negated condition.

In the above example, information about variable \lstinlinec{x} is determined directly from the conditional itself.  However, information can also be determined about variables which are {\em not even used in the conditional}.  The following illustrates:

\begin{lstlisting}
int y = x + 1
// y == x + 1
if x >= 0:
   // x $\ge$ 0 and y = x + 1
   ...
else:
   // x < 0 and y = x + 1
   ...
\end{lstlisting}

Again, in this example we know nothing about either variable \lstinlinec{x} or \lstinlinec{y} beforehand.  After the assignment, we know the relationship between variable \lstinlinec{x} and \lstinlinec{y}.  Hence, on the true branch, we can conclude, for example, that \lstinlinec{y > 0}.  Similarly, on the false branch, we can conclude that \lstinlinec{y <= 0}.

\subsection{Invocations}

To reason about a function call, we must first establish that its precondition is met.  Once we know this, we can then assume that its postcondition holds true for any value returned.  Consider the following example:

\begin{lstlisting}
// x > 0
x = decrement(x)
// x $\ge$ 0
\end{lstlisting}

Recall the definition of the \lstinlinec{decrement()} function from \S\ref{c_reasoning_forward}.  Since we know \lstinlinec{x > 0} is true before the function call, we know: firstly, that the precondition of \lstinlinec{decrement()} is satisfied; hence, secondly, that the postcondition of \lstinlinec{decrement()} implies \lstinlinec{x >= 0} after the statement.

\subsection{Records}
Records are compound data types made up from one or more fields.  The fields of a record can be individually assigned, and this makes it easier to manage complex data.  As with general assignments, we can reason about assignments to records by considering the effect of the assignment on the given field:

\begin{lstlisting}
// point.x > 0 $\land$ point.y > 0
point.x = 0
// point.x = 0 $\land$ point.y $\ge$ 0
\end{lstlisting}

Here, we can see that the assignment to the field \lstinline{x} does not affect what is known about the other fields stored in the record (i.e. \lstinline{y}).

\subsection{Arrays}
As with records, arrays are compound data types made up from zero or more elements.  The elements of an array can be individually assigned and arrays make for a powerful building block for more complex data structures.  When dealing with arrays we can reason about the effect of an assignment to a specific element, if we know what that element was:

\begin{lstlisting}
// $|xs| > 0\land\forall i.(0 \le i < |xs|\implies xs[i] > 0)$
xs[0] = 0
// $xs[0] == 0\land |xs| > 0\land\forall i.(0 \le i < |xs|\implies xs[i] > 0)$
\end{lstlisting}

Here, we initially have some knowledge about the entire array but, after the assignment, we have split this into two regions.  In general, we do not often know which specific element has been updated and, hene, we must assume that any could have been:

\begin{lstlisting}
// $n >= 0 \land n < |xs|\land\forall i.(0 \le i < |xs|\implies xs[i] > 0)$
xs[n] = 0
// $n >= 0 \land n < |xs|\land\forall i.(0 \le i < |xs|\implies xs[i] \ge 0)$
\end{lstlisting}

Here, our reasoning is a little coarse as we have conservatively updated our knowledge by assuming that any possible element of the array could be updated.

\subsection{Assignments Reconsidered}

{\bf PRECONDITIONS}
Examine examples with updating lists, records, etc.

\section{Complex Reasoning}
Examine problem with our simplistic way of thinking about assignments.

\section{Forwards versus Backward Reasoning}
Discuss the so-called weakest precondition transformer.

\section{Example}

\lstinputlisting{../examples/reasoning/sum.whiley}