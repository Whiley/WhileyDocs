\chapter{Simple Reasoning}

Introduction to reasoning about a function in a forwards or backwards direction.  For forwards direction, need a staged approach where we eventually introduce the concept of a skolem.  Cover control-flow constructs upto, but not including loops.  This includes if-conditions, switch statements, function calls, array assignments, reference assignments, etc.  Also, introduce basic concept of a weakest precondition versus a strongest postcondition.  Introduce control-flow graph 
and the path sensitive traversal?

\section{Functions}

Begin with pre / post-conditions and return statements.

\section{Assignments}

Examine assignments.

\section{Conditionals}
Examine simple examples involving conditions, such as \lstinline{abs()} and \lstinline{max()}

\section{Function Calls}
Examine involving function calls

\section{Compound Data Types}
Examine examples with updating lists, records, etc.

\section{Complex Reasoning}
Examine problem with our simplistic way of thinking about assignments.

\section{Forwards versus Backward Reasoning}
Discuss the so-called weakest precondition transformer.

\section{Exercises}

\begin{ex}
The function \lstinline{neg()} returns the arithmetic negation of a value.
For example, \lstinline{neg(1) = -1}.  An implementation of
this function is given as follows:
\begin{lstlisting}
function neg(int x) => (int r):
    return -x
\end{lstlisting}
Provide an appropriate {\em post-condition} for this function.
\end{ex}

\begin{ex}
  The following function computes the absolute difference between two values:

\begin{lstlisting}
function diff(int x, int y) => (int r):
    //
    if x > y:
        return x - y
    else:
        return y - x
\end{lstlisting}

A {\em pre-condition} of this function is that parameter \lstinline{x}
is between \lstinline{0} and \lstinline{255} (inclusive) and,
likewise, that variable \lstinline{y} is between \lstinline{-128} and
\lstinline{127} (inclusive).  Provide an appropriate specification for
this function, such that the post-condition is as tight as possible.
\end{ex}
