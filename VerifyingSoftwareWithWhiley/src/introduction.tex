\chapter{Introduction}
The Whiley programming language has been in active development since
2009.  The language was designed specifically to help the programmer
eliminate bugs from his/her software.  The key feature is that Whiley
allows programmers to write {\em specifications} for their functions,
which are then checked by the compiler.  For example, here is the
specification for the \lstinline{max()} function which returns the
maximum of two integers:

\begin{lstlisting}
function max(int x, int y) => (int z)
// must return either x or y
ensures x == z || y == z
// return must be as large as x and y
ensures x <= z && y <= z:
    // implementation
    if x > y:
        return x
    else:
        return y
\end{lstlisting}

Here, we see our first piece of Whiley code.  This declares a function
called \lstinline{max} which accepts two integers \lstinline{x} and
\lstinline{y}, and returns an integer \lstinline{z}.  The body of the
function simply compares the two parameters and returns the largest.
The two \lstinline{requires} clauses form the function's {\em
  post-condition}, which is a guarantee made to any caller of this
function.  In this case, the \lstinline{max} function guarantees to
return one of the two parameters, and that the return will be as large
as both of them.  In plain English, this means it will return the
maximum of the two parameter values.

When verification is enabled the Whiley compiler will check that every
function meets its specification.  For our \lstinline{max()} function,
this means it will check that body of the function guarantees to
return a value which meets the function's post-condition.  To do this,
it will explore the two execution paths of the function and check each
one separately.  If it finds a path which does not meet the
post-condition, the compiler will report an error.  In this case, the
\lstinline{max()} function above is implemented correctly and so it
will find no errors.  The advantage of providing specifications is
that they can help uncover bugs and other, more serious, problems
earlier in the development cycle.  This leads to software which is
both more reliable and more easily maintained (since the
specifications provide important documentation).

\subsection{Objectives}
