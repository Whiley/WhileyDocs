\chapter{Introduction}

Today, software is found in cars, phones, aeroplanes, trains, power stations, medical equipment, military weapons and just about every other aspect of life.  Failures in such software can have a significant effect.  On Thursday August 14, 2003, a significant power outage spread across parts of the northeast and midwest of the United States, affecting an estimated 55 million people.  During this time emergency services struggled to cope with the demand, backup generators failed, telephone networks overloaded, city water systems lost pressure and a state of emergency was declared in New York.  In 2015, Air New Zealand announced that it would reboot the computers every three months on its new fleet of Boeing 787 {\em Dreamliners}.  This was in response to a software bug identified by Boeing (in fact, an {\em arithmetic overflow}) which can cause the US\$250 million plane to lose all power mid-flight. 

\marginpar{Not good}{The} rise of the internet has also given rise to the {\em black hats}.  That is hackers who break into computer systems for criminal purposes, such as stealing credit card numbers.  Many attacks on computer systems are made possible because of hidden software bugs.  One such example was the infamous infamous {\em Heartbleed} bug which was disclosed in 2014.  This was another simple software bug (in fact, a {\em buffer overrun}) found in the widely used OpenSSL cryptography library.   There was concern across the internet for several reasons:  firstly, a large number of people were affected (at the time, around half a million secure web servers were believed to be vulnerable); secondly, OpenSSL was an {\em open source} project but the bug had gone unnoticed for over two years. Joseph Steinberg of Forbes wrote, {\em ``Some might argue that [Heartbleed] is the worst vulnerability found (at least in terms of its potential impact) since commercial traffic began to flow on the Internet''.}


% Introduce some classic historical bugs, and emphasis that we want to reduce these as much as possible.  There are lots of good examples, some of which are not coding failures but failures of understanding the requirements, etc.

% Numerous important software systems have failed due to program bugs. Historic examples include the Therac-25 disaster where a computer-operated X-ray machine gave lethal doses to patients, the 1988 worm which wreaked havoc on the internet by exploiting a buffer overrun, and the (unmanned) Ariane 5 rocket which exploded shortly after launch because of an integer overflow (see this video and this list for more).

\section{Software Verification}

Given the woeful state of much of the software we rely on every day, the question is: {\em what can we as computer scientists do about it?}  Of course, we want to ensure that software when written is correct.  {\em But how?}  To answer this, we need to go back and rethink what software development is.  When we write a program, we have in mind some idea of what this means.  When we have finished our program, we might run it to see whether it appears to do the right thing.  However, as anyone who has ever written a program will know: {\em this is not always enough!}  Even if our program appears to work after a few tests, there is still a good chance it will go wrong for other inputs we have not yet tried.  The question is: {\em how can we be sure our program is correct?}

In trying to determine whether our program is correct, our first goal is to determine precisely what it should do.  In writing our program, we may not have had a clear idea of this from the outset.  Therefore, we need to determine a {\em specification} for our program.  This is a precise description of what the program should and should not do.  Only with this can we begin to consider whether or not our program actually does the right thing.  If we have used a modern programming language,  such as Java, C\# or C++, then we are already familiar with this idea.  These languages require a limited specification be given for functions in the form of {\em types}.  That is, when writing a function in these languages we must specify the permitted type of each parameter and the return.  These types put requirements on our code and ensure that certain errors are impossible.  For example, when calling a function we must ensure that the argument values we give have the correct type.  If we fail to do this, then the compiler will complain with an error message of some kind and force us to immediately {\em fix the problem}.

Software verification is the process of checking that a program meets its specification.  In this book we adopt a specific approach referred to as {\em automated software verification}.  This is where a tool is used to automatically check a function meets its specification or not.  This is very similar to the way that compilers for languages like Java check that types are used correctly.  The tool we choose for this is called {\em Whiley}.  This programming language allows us to write specifications for our functions, and provides a special compiler which will attempt to check them for us automatically.  We choose this tool because we are familiar with it, but it is not the only tool we could have chosen.  Other suitable tools such as Spark/ADA, Dafny, Spec\#, ESC/Java provide similar functionality and could also be used with varying degrees of success.

\section{Whiley}
The Whiley programming language has been in active development since 2009.  The language was designed specifically to help the programmer eliminate bugs from his/her software.  The key feature is that Whiley allows programmers to write {\em specifications} for their functions, which are then checked by the compiler.  For example, here is the specification for the \lstinline{max()} function which returns the maximum of two integers:

\begin{tcolorbox}\begin{lstlisting}[language=Whiley]
function max(int x, int y) => (int z)
// must return either x or y
ensures x == z || y == z
// return must be as large as x and y
ensures x <= z && y <= z:
    // implementation
    if x > y:
        return x
    else:
        return y
\end{lstlisting}\end{tcolorbox}

Here, we see our first piece of Whiley code.  This declares a function called \lstinline{max} which accepts two integers \lstinline{x} and \lstinline{y}, and returns an integer \lstinline{z}.  The body of the function simply compares the two parameters and returns the largest.
The two \lstinline{requires} clauses form the function's {\em post-condition}, which is a guarantee made to any caller of this function.  In this case, the \lstinline{max} function guarantees to return one of the two parameters, and that the return will be as large as both of them.  In plain English, this means it will return the maximum of the two parameter values.

When verification is enabled the Whiley compiler will check that every function meets its specification.  For our \lstinline{max()} function,
this means it will check that the body of the function guarantees to return a value which meets the function's post-condition.  To do this, it will explore the two execution paths of the function and check each one separately.  If it finds a path which does not meet the post-condition, the compiler will report an error.  In this case, the \lstinline{max()} function above is implemented correctly and so it will find no errors.  The advantage of providing specifications is that they can help uncover bugs and other, more serious, problems earlier in the development cycle.  This leads to software which is both more reliable and more easily maintained (since the specifications provide important documentation).

\subsection{Objectives}
