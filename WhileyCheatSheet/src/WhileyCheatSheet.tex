\documentclass[a4paper,10pt,twcolumn]{article} 
\usepackage[a4paper,twocolumn,noheadfoot,width=170mm,height=267mm,lmargin=20mm,tmargin=15mm,twoside=true,marginpar=16.5mm]{geometry}

\usepackage{times}
\usepackage{graphicx}
\usepackage{url}
\usepackage{color}
\usepackage{enumerate}
\usepackage{tabularx}

\definecolor{lightgray}{RGB}{225,225,225}

\usepackage{listings}

% ==========================================
% Whiley LstListing Mode
% ==========================================

\pagestyle{plain}
\lstloadlanguages{Java}
\lstset{
        language=Java,
        keywords={function, method, type, constant, assert, for, while, switch, is, if, case, return,
          else, package, define, as, requires, ensures, where, no, new,
          all, bool, int, byte, char, string, void, real, in, any,
          null, var, public, protected, private, skip, break, do,
          throws, catch, continue, default, try
        },
        basicstyle=\small\ttfamily,
        commentstyle=\rmfamily\itshape,
        stringstyle=\small\itshape,
        moredelim=*[s][commentstyle]{/*}{*/}, % allows keyword highlighting inside comments
        morecomment=[l][commentstyle]{//},      % single line comments are set by...
        backgroundcolor=\color{lightgray},
        frame=single, % adds a frame around the code
%        frameround=tttt,
        framesep=0.25cm,
        texcl,                                                          % ...LaTeX
        moredelim=[is][\ttfamily]{<code>}{</code>}, % allows setting code inside multi-line comments
        moredelim=*[s][\ttfamily]{/*@}{*/}, % JML annotations
        moredelim=*[l][\ttfamily]{//@}, % JML annotations
        moredelim=**[is][\itshape]{/_}{_/}, % allows emphasizing sub-expressions
        moredelim=**[is][\bfseries]{/b_}{_b/}, % allows emphasizing sub-expressions
        escapeinside={(*@}{@*)}, %use (*@\label{line:desc}@*) to label lines for \ref
        mathescape=true,                % allow $ $ for math mode (will this break things?}
        tabsize=4,
	tab=\rightarrowfill,
        xleftmargin=0.25cm,
        xrightmargin=0.25cm,
        showspaces=false,
        showtabs=false,
        columns=fullflexible,
        numberstyle=\tiny,
        numbers=left,
        keepspaces=true,
        mathescape=true, % allows $ to switch in and out of math mode within listings
        literate={->}{{$\rightarrow$}}1 
           {tau}{{$\tau$}}1 {tau'}{{$\tau\prime$}}1 {(|}{{$\lpbar$}}1 {---}{{$\hole$}}1
           {-*-}{{$\times$}}1 {||_}{{$\lceilfloor$}}1 {_||}{{$\rceilfloor$}}1
           {/0}{{$\emptyset$}}1 {/bul}{{$\unit$}}1 {:->}{{$\mapsto$}}1
           {~}{$\sim$}1,
}

\newcommand{\token}[1]{\Large\strut\scriptsize\fcolorbox{black}{lightgray}{\strut#1}}

\pagestyle{empty}
\date{}

\begin{document}
\twocolumn[\noindent {\Large{\bf Whiley Cheat
    Sheet}}\\\noindent\rule{\textwidth}{1pt}\begin{flushright}{\em By
    David J. Pearce, 2014.  See \url{http://whiley.org}}\end{flushright}]

\section*{Values}
Values are the fundamental units of execution in Whiley and have value semantics, rather than reference semantics (as in many object-oriented languages).

\noindent \begin{tabular}{l@{\hspace*{0.1cm}}r}
\token{\lstinline+null+} & {\em Null value}\\
\token{\lstinline+true+} \token{\lstinline+false+} & {\em Boolean values}\\
\token{\lstinline+123+} \token{\lstinline+-99+} \token{\lstinline+0xFF+} & {\em Integer values}\\
\token{\lstinline+"Hello"+} \token{\lstinline+"new\\n line"+} & {\em String values}\\
\token{\lstinline+(1,2,3)+} \token{\lstinline+(true,null)+} & {\em Tuple values}\\
\token{\lstinline+[1,2,3]+} \token{\lstinline+[1,"xyz"]+} & {\em Array values}\\
\token{\lstinline+\{name: "dave"\}+}, \token{\lstinline+\{x: 1, y: 0\}+} & {\em Record values}\\
\end{tabular}
\section*{Types}
The Whiley programming language is {\em statically typed}, meaning that every expression has a type determined at compile time.  Furthermore, evaluating an expression is guaranteed to yield a value of its type.\\\\ 
\begin{tabular}{l@{\hspace*{0.1cm}}r}
\token{\lstinline+null+} \token{\lstinline+bool+} \token{\lstinline+int+} & {\em Primitive types}\\
\token{\lstinline+int|null+} \token{\lstinline+bool|int+} & {\em Union types}\\
\token{\lstinline+(int,int)+} \token{\lstinline+(int,null,bool)+} & {\em Tuple types}\\
\token{\lstinline+int[]+} \token{\lstinline+bool[][]+} \token{\lstinline+(int|null)[]+} & {\em Array types}\\
\token{\lstinline+\{bool f\}+} \token{\lstinline+\{int len, int[] is\}+} & {\em Record types}\\
\end{tabular}

\section*{Expressions}
The majority of work performed by a Whiley program is through the execution of {\em expressions}.  Every expression produces a value and may have additional side effects.\\\\
\begin{tabular}{l@{\hspace*{-0.2cm}}r}
\token{\lstinline|x + 1|} \token{\lstinline|2 * y|} \token{\lstinline|z - 1|} \token{\lstinline|(x + y)/2|}& {\em Arithmetic}\\
\token{\lstinline|x < y|} \token{\lstinline|0 >= z|} \token{\lstinline|x == y|} \token{\lstinline|x != y|}& {\em Comparisons}\\
\token{\lstinline+!x+} \token{\lstinline+x&&y+} \token{\lstinline+x||y+} \token{\lstinline+x==>y+} \token{\lstinline+x<==>y+} & {\em Boolean}\\
\token{\lstinline+|ls|+} \token{\lstinline+ls[0]+} \token{\lstinline+[true; n]+} & {\em Arrays}\\
\token{\lstinline|[1,x+y]|} \token{\lstinline+x in xs+} & \\
\token{\lstinline|\{x: 1+y\}|} \token{\lstinline+xr.f+} \token{\lstinline+xr.f.g+} & {\em Records}\\
\token{\lstinline+all \{ i in 0..|xs| | xs[i]>= 0\}+}& {\em Quantifiers}\\
\token{\lstinline+no \{ i in 0..|xs| | xs[i]<0 \}+}& \\
\token{\lstinline+some \{ i in 0..|xs| | xs[i]>=0 \}+}&\\
\token{\lstinline+x is null+} \token{\lstinline+x is int+} & {\em Type Tests}
\end{tabular}
\newpage
\section*{Statements}
The execution of a Whiley program is controlled by {\em statements}, which cause effects on the environment.  Statements in Whiley do not produce values.  Compound statements may contain other statements.

Variables are declared and initialised through {\em variable declarations}.  Variables must be declared before being used.

\noindent \begin{tabular*}{\columnwidth}{c @{\extracolsep{\fill}} c @{\extracolsep{\fill}} c}
\begin{minipage}[t]{2.25cm}
\begin{lstlisting}
int x
\end{lstlisting}
\end{minipage}
&
\begin{minipage}[t]{2.25cm}
\begin{lstlisting}
int x = 1
\end{lstlisting}
\end{minipage}
&
\begin{minipage}[t]{2.75cm}
\begin{lstlisting}
int x, int y
\end{lstlisting}
\end{minipage}
\\
\end{tabular*}

Variables, fields and map or list elements can be {\em assigned}.  Variables must be defined before being used.

\noindent \begin{tabular*}{\columnwidth}{c @{\extracolsep{\fill}} c @{\extracolsep{\fill}} c @{\extracolsep{\fill}} c}
\begin{minipage}[t]{2.25cm}
\begin{lstlisting}
x = x + y
\end{lstlisting}
\end{minipage}
&
\begin{minipage}[t]{2cm}
\begin{lstlisting}
x[0] = 1
\end{lstlisting}
\end{minipage}
&
\begin{minipage}[t]{1.75cm}
\begin{lstlisting}
r.f = 3
\end{lstlisting}
\end{minipage}
&
\begin{minipage}[t]{1.75cm}
\begin{lstlisting}
x,y = t
\end{lstlisting}
\end{minipage}
\\
\end{tabular*}

{\em Conditional} statements control the flow of execution based on the result of a boolean expression.

\noindent \begin{tabular*}{\columnwidth}{c @{\extracolsep{\fill}} c @{\extracolsep{\fill}} c}
\begin{minipage}[t]{2.25cm}
\begin{lstlisting}
if x < 0:
   ...
...
\end{lstlisting}
\end{minipage} 
&
\begin{minipage}[t]{2.25cm}
\begin{lstlisting}
if x < 0:
   ...
else:
   ...
\end{lstlisting}
\end{minipage}
&
\begin{minipage}[t]{3cm}
\begin{lstlisting}
if x < 0:
   ...
else if x > 0:
   ...
\end{lstlisting}
\end{minipage}\\
\end{tabular*}

{\em Looping statements} control the flow of execution by repeating some sequence of statements zero or more times.

\noindent\begin{tabular*}{\columnwidth}{c @{\extracolsep{\fill}} c}
\begin{minipage}[t]{3.75cm}
\begin{lstlisting}
while x<0:
   ...
...
\end{lstlisting}
\end{minipage}
&
\begin{minipage}[t]{3.75cm}
\begin{lstlisting}
do:
   ...
while x<0
...
\end{lstlisting}
\end{minipage}
\end{tabular*}

{\em Switch statements} control execution flow by matching the result of an expression.

\noindent\begin{tabular*}{\columnwidth}{c @{\extracolsep{\fill}} c}

\begin{minipage}[t]{3.75cm}
\begin{lstlisting}
switch x:
   case 1:
       x = x + 1
   case 1,2:
       x = 0
...
\end{lstlisting}
\end{minipage}
&
\begin{minipage}[t]{3.75cm}
\begin{lstlisting}
switch x:
   case 1:
       x = x + 1
   default:
       x = 0
...
\end{lstlisting}
\end{minipage}
\\
\end{tabular*}

{\em Return statements} terminate the execution of a function or method and may return the result of an expression.  

\noindent\begin{tabular*}{\columnwidth}{c @{\extracolsep{\fill}} c @{\extracolsep{\fill}} c}
\begin{minipage}[t]{1.75cm}
\begin{lstlisting}
return
\end{lstlisting}
\end{minipage}
&
\begin{minipage}[t]{3cm}
\begin{lstlisting}
return x + 3
\end{lstlisting}
\end{minipage}
&
\begin{minipage}[t]{2.75cm}
\begin{lstlisting}
return x,y
\end{lstlisting}
\end{minipage}
\\
\end{tabular*}

{\em Assertion} and {\em assumption} statements enable the programmer to express knowledge at a given point.

\noindent\begin{tabular*}{\columnwidth}{c @{\extracolsep{\fill}} c}
\begin{minipage}[t]{3cm}
\begin{lstlisting}
assert x > 0
\end{lstlisting}
\end{minipage}
&
\begin{minipage}[t]{4.75cm}
\begin{lstlisting}
assume x > 0 ==> y < 3
\end{lstlisting}
\end{minipage}\\
\end{tabular*}

{\em Break statements} terminate loops early; {\em debug} statements enable output from functions; {\em skip} statements are a no-op.

\noindent\begin{tabular*}{\columnwidth}{c @{\extracolsep{\fill}} c @{\extracolsep{\fill}} c}
\begin{minipage}[t]{1.5cm}
\begin{lstlisting}
break
\end{lstlisting}
\end{minipage}
&
\begin{minipage}[t]{4cm}
\begin{lstlisting}
debug "got here"
\end{lstlisting}
\end{minipage}
&
\begin{minipage}[t]{1.5cm}
\begin{lstlisting}
skip
\end{lstlisting}
\end{minipage}\\
\end{tabular*}

% {\em Exception statements} give exceptional termination from function and methods to support error handling.

% \noindent\begin{tabular*}{\columnwidth}{c @{\extracolsep{\fill}} c @{\extracolsep{\fill}} c}
% \begin{minipage}[t]{3.5cm}
% \begin{lstlisting}
% try:
%   ...
% catch(Ex e):
%   ...
% \end{lstlisting}
% \end{minipage}
% &
% \begin{minipage}[t]{4cm}
% \begin{lstlisting}
% throw {msg: "err"}
% \end{lstlisting}
% \end{minipage}\\
% \end{tabular*}

\section*{Declarations}
A {\em declaration} declares a named entity within a source file and may refer to named entities in this or other source files and (in some cases) may {\em recursively} refer to itself.

{\em Constant declarations} define constants with known values at compile-time (they cannot be recursively defined).

\begin{lstlisting}
constant TEN is 10
constant TWENTY is TEN * 2
\end{lstlisting}

{\em Type declarations} define named types composed from other types (they may be recursively defined).

\begin{lstlisting}
type Point is { int x, int y }
\end{lstlisting}

\begin{lstlisting}
type Link is { LinkedList next, int data }
type LinkedList is null | Link
\end{lstlisting}

{\em Function declarations} define functions which are {\em pure} and may not have side-effects.  They are guaranteed to return the same result given the same arguments, and are permitted within specifications. 

\begin{lstlisting}
function find(int[] xs, int x) -> int:
   ...
\end{lstlisting}

{\em Method declarations} define methods which are {\em impure} and
may have side-effects.  They cannot be used within specifications.

\begin{lstlisting}
method main(System.Console console):
   console.out.println("Hello World")
\end{lstlisting}

\section*{Specifications}

A {\em precondition} is a condition over the parameters of a function
that must hold when the function is called.  A {\em
  postcondition} is a condition over the return values of a function
which is required to be true after the function is called.

\begin{lstlisting}
function decrement(int x) -> (int y) 
// Parameter x must be greater than zero
requires x > 0
// Return must be greater or equal to zero
ensures y >= 0
// Return must be less than input
ensures y < x:
    //
    return x - 1
\end{lstlisting}

A {\em data-type invariant} is a constraint on the values of a
declared type which must be true for any instance of it.

\begin{lstlisting}
type nat is (int n) where n >= 0
type pos is (int p) where p > 0
\end{lstlisting}

A {\em loop invariant} is a property which holds before and after each
iteration of the loop, such that: {\bf (1)} the loop invariant must hold on
entry to the loop; {\bf (2)} assuming the loop invariant holds at the start of
the loop body (along with the condition), it must hold at the end; {\bf (3)}
the loop invariant (along with the negated condition) can be assumed
to hold immediately after the loop.
\newline
\begin{lstlisting}
 ...
 int i = 0
 while i < x where i >= 0:
     i = i + 1 
 ...
\end{lstlisting}


\section*{Examples}

The following function computes the maximum value of two integer parameters.

\lstinputlisting{../examples/max.whiley}

The following function uses a \lstinline{break} to exit a \lstinline{while} loop when the first element matching parameter \lstinline{x} is found.  

\lstinputlisting{../examples/indexOf.whiley}

The following function computes the length of a linked list.  

\lstinputlisting{../examples/list.whiley}

The following function reverses the values in a list of integers.

\lstinputlisting{../examples/reverse.whiley}



\end{document}

