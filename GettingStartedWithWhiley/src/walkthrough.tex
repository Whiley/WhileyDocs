\newpage
\section{Quick Walkthrough}
This section provides a quick walk through of the main concepts and
ideas in the Whiley language.  Through a series of short examples,
we'll introduce the basic building blocks of the language.

\subsection{Booleans and Numbers}

As found in many languages, Whiley supports a range of primitive
datatypes for representing boolean, integers, real numbers, bytes,
characters, etc.  Of these, the most commonly used are:

\begin{itemize}
\item {\bf Booleans} are denoted by the type \lstinline{bool}.  This is the
  simplest of the primitive datatypes, and has only two possible
  values: \lstinline{true} or \lstinline{false}.
\item {\bf Integers} are denoted by the type \lstinline{int}.  Integers in
  Whiley are {\em unbounded}. This means that, in theory at least, a
  variable of type int can take on {\em any possible integer value};
  this differs from many other languages (e.g. Java), which limit the
  number of possible values (e.g. following 32-bit two's complement).
\item {\bf Real Numbers} are denoted by the type
  \lstinline{real}. Reals in Whiley are {\em unbounded
    rationals}. This means that, in theory at least, a variable of
  type real can take on {\em any possible rational value}. Again, this
  offers significantly better precision than, for example,
  \lstinline{float} or \lstinline{double} types based on IEEE754 as
  found in other languages (e.g. Java).
\end{itemize}


\noindent A very simple example which illustrates the \lstinline{int} and
\lstinline{bool} types is the following:

\lstinputlisting{../examples/walkthrough_1.whiley}

This declares a simple function which returns \lstinline{true} if the
first parameter, \lstinline{x}, is less than the second,
\lstinline{y}, and \lstinline{false} otherwise.

\begin{insight}{Indentation Syntax.}
  From the above example you should notice that Whiley, unlike many
  languages, does not use curly braces (i.e. \verb+{ ... }+) to
  demarcate blocks of code.  Instead, Whiley uses {\em indentation
    syntax} which was popularised by the Python programming language.
  The start of a new code block is signalled by a preceding \verb+:+
  on the previous line.  The new block must be indented by at least
  one space (the actual amount doesn't matter) and all subsequent
  statements with the same indentation are included.
\end{insight}

\begin{insight}{Ints versus Reals.}
Unlike many other languages, Whiley provides a strong relationship
between values of type \lstinline{int} and those of type
\lstinline{real}.  Specifically, every \lstinline{int} value can be
represented precisely as a \lstinline{real} value.  Although it may
seem surprising, this is not true for many other languages
(e.g. Java), where there are \lstinline{int} (resp. \lstinline{long})
values which cannot be represented using \lstinline{float}
(resp. \lstinline{double})~\cite{GJSB05}.
\end{insight}

% \begin{lstlisting}
% function floor(real x) => int:
%     int num / int den = x    // extract numerator and denominator
%     int r = num / den        // integer division
%     if x < 0.0 && den != 1: 	 
%         return r - 1 
%     else:
%         return r 
% \end{lstlisting}

\newpage
\subsection{Sets, Lists and Maps}
\label{walkthrough_collections}
Like many modern programming languages, Whiley provides built-in types
for representing collections.  The following illustrates a short
function which multiplies a vector by a scalar:

\lstinputlisting{../examples/walkthrough_2.whiley}

This illustrates a few of the common collection operations.  Firstly,
the size of a collection is obtained using the {\em length} operator
(i.e. \lstinline{|vector|} returns the length of \lstinline{vector}).
Secondly, the \lstinline{for} loop is useful for iterating over the
elements of a collection.  In this case, \lstinline{0 .. |vector|}
returns a {\em list} of consecutive integers from \lstinline{0} up to
(but not including) \lstinline{|vector|}.  Finally, the {\em list
  access} operator, \lstinline{vector[i]}, returns the element at
index \lstinline{i}.  The three different kinds of collections
supported in Whiley are:
\begin{itemize}
\item {\bf Sets} (e.g. \lstinline|{int}|) provide the simplest form of collection, and are constructing using a {\em set constructor} (e.g. \lstinline|{1,2,3}|).  They support the {\em set union} (e.g. \lstinline{xs + ys}), {\em set intersection} (e.g. \lstinline{xs & ys}) and {\em set difference} (e.g. \lstinline{xs - ys}) operators.  One can test for inclusion using the {\em element of} operator (e.g. \lstinline{x in xs}), or the {\em subset} (e.g. \lstinline{xs $\subset$ ys}) and {\em subset or equal} (e.g. \lstinline{xs $\subseteq$ ys}) operators.
  
\item {\bf Maps} (e.g. \lstinline|{int=>string}|) provide a halfway point between sets and lists.  They are similar to dictionaries in Python or the \lstinline{Map} interface in Java.  They can be viewed as a set of {\em key$\times$ value} pairs, where every key maps to exactly one value.  Maps are constructed using a {\em map constructor} (e.g. \lstinline|{1=>"hello",2=>"world"}|), and elements are accessed using the {\em map access} operator (e.g. \lstinline{map[i]}).  

\item {\bf Lists} (e.g. \lstinline{[int]}) are similar to arrays
  (e.g. in Java), but they can also be resized.  As can be seen above,
  they support the {\em list access} operator
  (e.g. \lstinline{vector[i]}).  They also support the {\em list
    append} operator (e.g. \lstinline{[1,2,3] ++ [4,5,6]}) and {\em
    sublist} operator (e.g. \lstinline{vector[0..2]}).  Finally, lists are constructed using a {\em list constructor} (e.g. \lstinline{[1,2,3]}).

\end{itemize}

All collection kinds can be iterated using the built-in
\lstinline{for} loop construct.  For example, here is a function to iterate a map looking for a particular value:

\lstinputlisting{../examples/walkthrough_3.whiley}

\noindent This function iterates over each key$\times$value pair (i.e. \lstinline{k,v}) in a from integers to strings (i.e. \lstinline|{int=>string}|) using the \lstinline{for} statement.

\subsection{Records and Tuples}
Aside from collection types, Whiley also allows provides {\em records} and {\em tuples} for grouping items together.  Records are similar to \lstinline{struct}s (as found in C) and objects (as found in JavaScript).  A record is constructed from one or more {\em fields}, each of which has a unique name and type.  For example, the following defines a simple record type for representing 2D points and a function for translating their position:

\lstinputlisting{../examples/walkthrough_4.whiley}

Here, a {\em user-defined type} named \lstinline{Point} has been defined.  This is a record type containing two \lstinline{int} fields, \lstinline{x} and \lstinline{y}.  In the \lstinline{translate()} function, a {\em record literal} is used to construct a new \lstinline{Point} to be returned.

Records are an important mechanism for giving meaning to data in Whiley.  For example, consider the following declaration of a rectangle:

\lstinputlisting{../examples/walkthrough_5.whiley}

Here, we see that a rectangle has a position, a width and a height.  The names of the fields are important for conveying their meaning in the real-world.

Sometimes, using a record type is a little more than necessary because the data being described is really quite simple.  In such cases we can use a {\em tuple} in Whiley, which provide a lightweight mechanism for grouping data.  For example, we might consider that using a record to defined our \lstinline{Point} is a little verbose.  Instead, we could rework the example to use a tuple as follows:

\lstinputlisting{../examples/walkthrough_6.whiley}

Since there is a widely used notation for describing 2D points --- namely $(x,y)$ --- it makes sense in this case to use a tuple over a record.  Tuples are also useful for describing functions which return multiple values.  For example, the following illustrates a simple function which swaps the order of its parameters:

\lstinputlisting{../examples/walkthrough_7.whiley}

Here, we see that the return type is a tuple and this provides a convenient and lightweight mechanism for returning multiple values.  Following on from this, Tuple types in Whiley also supports {\em destructuring syntax} (e.g. as found in Python).  The following example illustrates this:

\begin{lstlisting}
    int x
    int y
    ...
    x, y = swap(x,y) // destructuring assignment
    ...        
\end{lstlisting}

Here, we see that variables \lstinline{x} and \lstinline{y} are assigned their respective component of the tuple returned from \lstinline{swap}.  Again, this syntax simplies the use of functions with multiple return values.

\subsection{Strings and Characters}

\marginpar{Encodings?}Like most programming languages, Whiley provides explicit support for dealing with strings and characters.  This is achieved through the \lstinline{char} and \lstinline{string} data types.  Here, \lstinline{char} represents unicode characters, whilst \lstinline{string} is a list of \lstinline{char}s and supports all list operations (recall~\S\ref{walkthrough_collections}).  Here's a simple example illustrating the well-known \lstinline{replace()} function:

\lstinputlisting{../examples/walkthrough_8.whiley}

This function simply iterates through the elements of a string replacing every occurrence of the character \lstinline{old} with the character \lstinline{n}.