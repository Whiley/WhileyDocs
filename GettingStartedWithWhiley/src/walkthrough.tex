\newpage
\section{Quick Walkthrough}
This section provides a quick walk through of the main concepts and
ideas in the Whiley language.  Through a series of short examples,
we'll introduce the basic building blocks of the language.

\subsection{Booleans and Numbers}

As found in many languages, Whiley supports a range of primitive datatypes for representing boolean, integers, bytes, etc.  Of these, the most commonly used are:

\begin{itemize}

\item {\bf Booleans} are denoted by the type \lstinline{bool}.  This is the simplest of the primitive datatypes, and has only two possible values: \lstinline{true} or \lstinline{false}.

\item {\bf Integers} are denoted by the type \lstinline{int}.  Integers in Whiley are {\em unbounded}. This means that, in theory at least, a variable of type int can take on {\em any possible integer value}; this differs from many other languages (e.g. Java), which limit the number of possible values (e.g. following 32-bit two's complement).

\end{itemize}


\noindent A very simple example which illustrates the \lstinline{int} and
\lstinline{bool} types is the following:

\lstinputlisting{../examples/walkthrough_1.whiley}

This declares a simple function which returns \lstinline{true} if the
first parameter, \lstinline{x}, is less than the second,
\lstinline{y}, and \lstinline{false} otherwise.

\begin{insight}{Indentation Syntax.}
  From the above example you should notice that Whiley, unlike many
  languages, does not use curly braces (i.e. \verb+{ ... }+) to
  demarcate blocks of code.  Instead, Whiley uses {\em indentation
    syntax} which was popularised by the Python programming language.
  The start of a new code block is signalled by a preceding \verb+:+
  on the previous line.  The new block must be indented by at least
  one space (the actual amount doesn't matter) and all subsequent
  statements with the same indentation are included.
\end{insight}

\subsection{Arrays}
\label{walkthrough_collections}
Like many modern programming languages, Whiley provides a built-in array type for representing collections of data.  The following illustrates a short function which multiplies a vector by a scalar:

\lstinputlisting[lastline=7]{../examples/walkthrough_2.whiley}

This illustrates a few of the common array operations.  Firstly, the size of an array is obtained using the {\em length} operator (i.e. \lstinline{|vector|} returns the length of \lstinline{vector}).  Secondly, the \lstinline{while} loop is used to iterate over the elements of the array, whilst the {\em array access} operator, \lstinline{vector[i]}, returns the element at index \lstinline{i}.  

Arrays can be constructed using either {\em array initialisers} or {\em array generators}.  The following illustrates both syntaxes being used to call the function above:

\lstinputlisting[firstline=10]{../examples/walkthrough_2.whiley}

The array initialiser syntax simply constructs an array using the values as provided.  The array generator syntax, \lstinline{[e; n]}, constructs an array containing \lstinline{n} elements whose value is given by \lstinline{e}.
\subsection{Records}
Aside from collection types, Whiley also provides {\em records} for grouping items together.  Records are similar to \lstinline{struct}s (as found in C) and objects (as found in JavaScript).  A record is constructed from one or more {\em fields}, each of which has a unique name and type.  For example, the following defines a simple record type for representing 2D points and a function for translating their position:

\lstinputlisting{../examples/walkthrough_4.whiley}

Here, a {\em user-defined type} named \lstinline{Point} has been declared.  This is a record type containing two \lstinline{int} fields, \lstinline{x} and \lstinline{y}.  In the \lstinline{translate()} function, a {\em record literal} is used to construct a new \lstinline{Point} to be returned.

Records are an important mechanism for giving meaning to data in Whiley.  For example, consider the following declaration of a rectangle:

\lstinputlisting{../examples/walkthrough_5.whiley}

Here, we see that a rectangle has a position, a width and a height.  The names of the fields are important for conveying their meaning in the real world.

\subsection{Multiple Returns}

Unlike many programming languages, Whiley supports {\em multiple} return values.  This can be useful as a lightweight mechanism for grouping data.  For example, the following illustrates a simple function which swaps the order of its parameters:

\lstinputlisting{../examples/walkthrough_7.whiley}

Here, we see that the return type is a tuple and this provides a convenient and lightweight mechanism for returning multiple values.  Following on from this, Whiley also supports {\em destructuring syntax} for handling multiple return values (e.g. as found in Python).  The following example illustrates this:

\begin{lstlisting}
    int x
    int y
    ...
    x, y = swap(x,y) // destructuring assignment
    ...        
\end{lstlisting}

Here, we see that variables \lstinline{x} and \lstinline{y} are assigned their respective component of the return values from \lstinline{swap}.  Again, this syntax simplifies the use of functions with multiple return values.

\subsection{Strings and Characters}

Unlike many programming languages, Whiley does not provide explicit data types representing strings and characters.  The reason for this is that such data types can already be encoded within the language and, by doing so, we can easily support the full range of different encodings (e.g. ASCII versus UTF-8, etc).  In Whiley, a string is simply a list of integers.  The following example illustrates the well-known \lstinline{replace()} function:

\begin{lstlisting}
import string from std.ascii
import char from std.ascii

function replace(string str, char old, char n) => string:
  //
  int i = 0
  while i < |str|:
     if str[i] == old:
        str[i] = n
     i = i + 1
  return str
\end{lstlisting}

This uses the standard ASCII representation of strings and characters, which are imported from the standard library.  