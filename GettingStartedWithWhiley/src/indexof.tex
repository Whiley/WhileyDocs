\newpage
\section{Example: IndexOf Function}

To better illustrate verification in Whiley, we'll develop the
specification for a slightly more challenging function.  This is the
\lstinline{indexOf()} function, described as follows:

\begin{lstlisting}
// Return the lowest index in the items array which equals the given item.
// If no such index exists, returns null.
function indexOf(int[] items, int item) -> int|null:
    ...
\end{lstlisting}

This is a common function found in the standard libraries of many
programming languages.  The body of the function examines each element
of the \lstinline{items} array and check whether or not it equals
\lstinline{item}.  To start with, we won't worry too much about the
body of the \lstinline{indexOf()} function.  Instead, we'll
progressively build up the specification until we are happy with it.
Then, we'll give an implementation of the function which meets this
specification.\\

\noindent To specify this function, we want to ensure three properties:

\begin{enumerate}
\item If the return is an integer \lstinline{i}, then
  \lstinline{items[i] == item}.
\item If the return is \lstinline{null}, there is no index
  \lstinline{j} where \lstinline{items[j] == item}.
\item If the return is an integer \lstinline{i}, then there is
  no index \lstinline{j} where \lstinline{j < i} and
  \lstinline{items[j] == item}.
\end{enumerate}

These properties determine how a correct implementation of the
\lstinline{indexOf()} function should behave.  We refer to them as the
{\em specification} of the \lstinline{indexOf()} function.

\subsection{Specifying Property 1 --- Return Valid Index}

The first of the above properties is the easiest, so lets start by
specifying that in Whiley.  At the same time, we'll also give an
initial implementation which satisfies this partial specification:

\lstinputlisting{../examples/indexof_1.whiley}

Here, we can see property (1) above written as an \lstinline{ensures}
clause in Whiley.  In particular, the phrase ``the return value is an
integer'' is translated into the condition ``\lstinline{i is int}''.
Likewise, the implication operator (i.e. \lstinline{==>}) is used to
say ``If ... then ...''.  We've also given an initial implementation
for the \lstinline{indexOf()} function which simply checks whether or
not \lstinline{items[0] == item}.  This implementation meets the
specification we have so far although, obviously, this is an
incomplete implementation of the \lstinline{indexOf} function!

\subsection{Specifying Property 2 --- Return Null if No Match}
Property (2) from our list above is more difficult to specify, because
it requires {\em quantification}.  There are several quantifiers
available in Whiley, including: \lstinline{all}, which allows us to
say ``for all elements in an array something is true''; and
\lstinline{no}, which allows us to say ``there is no element in the array
where something is true''. 

In Whiley, we can express property (2) from above in several different
ways.  The most direct translation would be:

\begin{lstlisting}
...
// If return is null, then for all j we have items[j] != item
ensures i is null ==> all { j in 0..|items| | items[j] != item }:
    ...
\end{lstlisting}

\noindent Here, the expression \lstinline{|items|} gives the length of
the items array, whilst the range expression \lstinline{0..|items|}
returns an array of consecutive integers from \lstinline{0} up to, but
not including, \lstinline{|items|}.

\subsection{Specifying Property 3 --- Return Least Index}

Property (3) from our array above is similar to property (2), except
that we not considering all elements of \lstinline{items}:

\begin{lstlisting}
...
// If return is an int i, then no index j where j $<$ i and items[j] == item
ensures i is int ==> all { j in 0 .. i | items[j] != item }:
\end{lstlisting}

\subsection{Working Implementation}

At this point, we can now give the complete specification for the
\lstinline{indexOf()} function, along with an initial implementation:

\lstinputlisting{../examples/indexof_2.whiley}

The implementation of \lstinline{indexOf()} given above meets the
function's specification.  Unfortunately, whilst this is true, the
Whiley compiler needs help to determine this.  Figure~\ref{eg_indexOf}
illustrates what happens when we compile the above code with
verification enabled.

\begin{figure}[!t]
\centering
\includegraphics[width=1.0\textwidth]{../images/indexOf.png}
\caption{Illustrating our first working version of the
  \lstinline{indexOf} function being compiled with verification
  enabled.  The compiler is reporting an error stating ``{\em index out of
  bounds (negative)}''.  This is because the compiler believes
  \lstinline{i} may be negative at this point.  Although we know this
  is not true, we must write a {\em loop invariant} to help the
  compiler see this.}
\label{eg_indexOf}
\end{figure}

\subsection{Verified Implementation}
Although our implementation of \lstinline{indexOf()} given above is
correct, it currently does not verify.  Although this distinction may
seem unimportant, it goes to the heart of what verification is about.
That is, we know the implementation of \lstinline{indexOf()} is
correct because we, {\em as humans}, have looked at it and believe it
is.  Whilst may be a reasonable approach for small examples, it
certainly is not for larger and more complex programs.  Humans are
fallible and we can easily believe something is true when it is not.
Therefore, we want a mechanical system which can examine a program and
report ``{\em Yes, I agree that this is correct}''.  Whiley provides
such a system when verification is enabled.  

Unfortunately, Whiley is not as smart as a human and often there will
be things we know that it does not.  In such cases, we need to help
Whiley by adding hints into our programs.  In this case, we need to
add some loop invariants (recall \S\ref{loop_invariants}) to help
Whiley verify our implementation of \lstinline{indexOf()}.  The first
part of the loop invariant we need is straightforward.  Since
\lstinline{i} is modified in the loop, we need am invariant to ensure
\lstinline{i >= 0} when \lstinline{items[i]} is accessed:

\begin{lstlisting}
    ...
    i = 0
    while i < |items| where i >= 0:
        ...
        i = i + 1    
\end{lstlisting}
We can see that this invariant holds on entry to the loop (i.e. since
\lstinline{i = 0} on entry).  Furthermore, if \lstinline{i >= 0} then \lstinline{i+1 >= 0} follows and, hence, the loop
invariant holds after each iteration.
