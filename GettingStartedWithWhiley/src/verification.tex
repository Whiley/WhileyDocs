\newpage
\section{Verification}

As discussed in the introduction, an important feature of Whiley is
{\em verification}.  That is made up of two aspects: firstly, the
ability to write specifications for functions and methods in Whiley;
secondly, the ability of the compiler to check the body of a function
or method meets its specification.

Unfortunately, specifications is not always straightforward and
requires considerable attention to detail.  Nevertheless, with
practice, it can easily fit into the routine of day-to-day
development.  In this section, we'll explore the basics of
verification in Whiley and, in the following section, we'll look at
practical example.

\paragraph{Example.}  To illustrate verification in Whiley, we'll
consider specifying a simple example function.  This is the
\lstinline{contains()} function, described as follows:

\begin{lstlisting}
// Return the lowest index in the items list which equals the given item.
// If no such index exists, return null.
function indexOf([int] items, int item) => int|null:
    ...
\end{lstlisting}

This is a common function found in the standard libraries of many
programming languages.  The body of the function examines each element
of the \lstinline{items} list and check whether or not it equals
\lstinline{item}.  To start with, we won't worry too much about the
body of the \lstinline{indexOf()} function.  Instead, we'll
progressively build up the specification until we are happy with it.
Then, we'll give an implementation of the function which meets this
specification.\\

\noindent To specify this function, we want to ensure three things:

\begin{enumerate}
\item When the return is an integer \lstinline{i}, then
  \lstinline{items[i] == item}.
\item When the return is \lstinline{null}, there is no index
  \lstinline{i} where \lstinline{items[i] == item}.
\item When the return is an integer \lstinline{i}, then there is
  no index \lstinline{j} where \lstinline{j < i} and
  \lstinline{items[j] == item}.
\end{enumerate}

The first of the above properties is the easiest, so lets start by
specifying that in Whiley.  At the same time, we'll also give an
initial implementation which satisfies this partial specification:

\begin{lstlisting}
function indexOf([int] items, int item) => (int|null i)
// When return value is an int i, then items[i] == item
ensures i is int ==> items[i] == item:
    //
    if |items| > 0 && items[0] == item:
        return item
    else:
        return null
\end{lstlisting}

Here, we can see property (1) above written as an \lstinline{ensures}
clause in Whiley.  In particular, the phrase ``When the return value
is an integer'' is translated into the condition ``\lstinline{i is int}''.  The implementation given checks whether or not the
first element of \lstinline{items} equals \lstinline{item}.  If it
does, then the index \lstinline{0} is returned; otherwise,
\lstinline{null} is returned.  This implementation meets the
specification we currently have although, obviously, it's not how we want the
\lstinline{indexOf} function to work!

Property (2) from our list above is more difficult to specify, because
it requires {\em quantification}.  There are several quantifiers
available in Whiley, including: \lstinline{all}, which allows us to
say ``for all elements in a list something is true''; and,
\lstinline{no}, which allows us to say ``there is no element in the list
where something is true''.

\begin{lstlisting}
function indexOf([int] items, int item) => (int|null i)
// When return value is an int i, then items[i] == item
ensures i is int ==> items[i] == item
// When
ensures no { x in items | x == item }:
\end{lstlisting}
