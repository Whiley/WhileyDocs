\newpage
\section{Flexible Types}
The previous section introduced us to the basic types found in Whiley,
such as integers (\lstinline{int}), rationals (\lstinline{real}) and
booleans (\lstinline{bool}).  However, unlike many languages, Whiley
provides a flexible and powerful approach to typing which go well
beyond the basic forms.  In this section, we will examine this in more
detail.

\subsection{Flow Typing}
To improve the programmer experience and reduce unnecessary tedium,
Whiley employs a {\em flow typing} system.  What this means is that
the type of a variable can vary at different points within a function.
To make this work, Whiley employs {\em union types}~\cite{BC91,IN07} along with {\em variable retyping}.  The
following example illustrates how this works (where the body of
\lstinline{indexOf()} is left out for brevity):

\begin{lstlisting}
 function indexOf(string str, char c) => null|int:
    ...

 function split(string str, char c) => [string]:
    var idx = indexOf(str,c)
    // idx has type null$|$int
    if idx is int:
        // idx now has type int
        int below = str[0..idx]
        int above = str[idx..]
        return [below,above]
    else:
        // idx now has type null
        return [str] // no occurrence
\end{lstlisting}
Here, \lstinline{indexOf()} returns the first index of a character in
the string, or \lstinline{null} if there is none.  The type
\lstinline{null|int} is a {\em union type}, meaning it is either an
\lstinline{int} or \lstinline{null}.  The \lstinline{split()} function
splits a string into two pieces based on the first occurrence of a
given character \lstinline{c}, or leaves the string as is otherwise.
It calls \lstinline{indexOf()} to determine the first occurrence of
\lstinline{c} in \lstinline{str}.  Observe that variable
\lstinline{idx} has been declared as type \lstinline{var}, meaning the
compiler will automatically infer the best possible type for it.

In the above example, Whiley's flow typing system seamlessly ensures
that \lstinline{null} is never dereferenced.  This is because the type
\lstinline{null|int} cannot be treated as an \lstinline{int}.
Instead, one must first check it is an \lstinline{int} using a type
test, such as ``\lstinline{idx is int}''.  Whiley automatically
{\em retypes} \lstinline{idx} to \lstinline{int} when this is known to be
true, thereby avoiding any awkward and unnecessary syntax (e.g. a cast
as required in many languages).

\begin{insight}{Null References.}  In many languages (e.g. C/C++,
  Java, etc) the use of \lstinline{null} is a significant source of
  error (see e.g.~\cite{Hoa09}).  For example, in Java dereferencing
  the \lstinline{null} value gives rise to a
  \lstinline{NullPointerException}, which is regarded as the most
  common form of error in Java~\cite{XYZ}.  The issue is that, in such
  languages, one can treat {\em nullable} references as though they
  are {\em non-null} references~\cite{Pier02}.  In the research
  literature, there have been many proposals to solve this problem
  using static type systems
  (e.g.~\cite{PQVHV01,FL03,KH07,CFJJ06,CJ07,MPPD08,Hub08,HJP08}).
  Unfortunately, at the time of writing, very few languages have
  incorporated such ideas.
\end{insight}

\begin{insight}{Intersections and Negations.}
  Whiley also supports so-called {\em intersection} and {\em negation}
  types.  Whilst these can be expressed directly in source code, they
  are generally less useful than unions.
\end{insight}

% \begin{insight}{Untagged Unions}.  Often confusion surrounding untagged
%     versus tagged unions.  The latter are more common, and sometimes
%     known as {\em sum types}.
% \end{insight}

\subsection{Recursive Types}
To represent tree-like structures, Whiley provides {\em recursive
  types} which are similar to the algebraic data types found in
functional languages (e.g. Haskell, ML, etc).  For example:
\begin{lstlisting}
// A linked list is either the empty list or a link
type LinkedList is EmptyList | Link

// The empty list contains no links
type EmptyList is null

// A single link in a linked list
type Link is {int data, LinkedList next}

// Return the length of a linked list (i.e. the number of links it contains)
function length(LinkedList l) => int:
  if l is null:    
    return 0 // l now has type null
  else:    
    return 1 + length(l.next) // l now has type \{int data, LinkedList next\}
\end{lstlisting}
Here, \lstinline{LinkedList} is a recursive type representing a linked
list (i.e. a sequence of zero or more links).  The empty list is
defined as \lstinline{null}, whilst each link contains a
\lstinline{data} field.  The type \lstinline{LinkedList} is defined in
terms of itself (i.e. it is recursive) and describes linked lists of
arbitrary size.

\begin{insight}{Value Semantics.}
  As discussed in \S\ref{value_semantics} all compounds structures in
  Whiley are passed by value, {\em including recursive types}.  This
  differs from common languages (e.g. Java), where linked structures
  are typically composed from {\em references} to link objects.  This
  means, for example, that linked structures in such languages can
  share substructures, leading to subtle and hard-to-find bugs.  In
  Whiley linked structures, such as \lstinline{LinkedList}, can never
  share substructure.
\end{insight}

The above example also serves as another illustration of flow typing
in Whiley.  More specifically, on the false branch of the type test
``\lstinline{l is null}'', variable \lstinline{l} is automatically
retyped to \lstinline+{int data, LinkedList next}+ --- thus ensuring
the subsequent dereference of \lstinline{l.next} is safe.  No casts
are required as would be needed for a conventional imperative language
(e.g. Java).  Finally, like all compound structures, the semantics of
Whiley dictates that recursive data types are passed by value (or, at
least, appear to be from the programmer's perspective).

\subsection{Structural vs Nominal Types}
Statically typed languages, such as Java, employ {\em nominal typing}
for recursive data types.  This means that two otherwise identical
types with different names are considered distinct and, for example, a
variable of one type cannot flow into the other.  In contrast, Whiley employs {\em structural typing} of records~\cite{Card88} to give greater flexibility.  This means that the name of a type is, generally speaking, unimportant.  Instead, identical types (i.e. those with identical {\em structure}) with different names are still considered identical in Whiley.  For example:

\begin{lstlisting}
// Define the notion of a "rectangle"
type Rectangle is { int x, int y, int width, int height }
// Define the notion of a "bounding box"
type BoundingBox is { int x, int y, int width, int height }

// Define a function for computing the area of a rectangle
function area(Rectangle rect) => int:
    return rect.width * rect.height
\end{lstlisting}

In this example, the types \lstinline{Rectangle} and \lstinline{BoundingBox} are {\em identical} and can be used interchangeably.  For example, if we have a variable of type \lstinline{BoundingBox}, we can safely pass it to the \lstinline{area()} function above to compute its area.

% \subsubsection{Nominal Types}

% It is possible to simulate nominal types in Whiley.

\subsection{Coercions}

A {\em coercion} converts a value of one type into a corresponding value of another type.  For example, in Whiley, we can coerce the \lstinline{int} value ``\lstinline{0}'' into the \lstinline{real} value ``\lstinline{0.0}''.  Many programming languages permit both {\em implicit} and {\em explicit} coercions, with the latter more commonly referred to as {\em casting}.  Implicit coercions occur without explicit direction from the programmer, and are often considered dangerous because of this.  

\begin{insight}{Lossless Coercions?}
The Java programming language attempts to enforce a requirement that implicit coercions are {\em lossless}.  Thus, any coercion which may result in a loss of information must be made explicit through the use of a cast.  Unfortunately, Java does permit implicit lossy coercions by, for example, allowing \lstinline{int} values to be implicitly coerced into \lstinline{float} values --- because not every \lstinline{int} value can be represented by a \lstinline{float} in Java.
\end{insight}

To address the issues surrounding implicit coercions, Whiley only permits implicit coercions which are lossless.  That is, which do not result in a loss of information.  In contrast, {\em lossy} coercions require an explicit cast be used to ``force'' the coercion.  The following example illustrates:
\begin{lstlisting}
type Link is {int data, LinkedList next}
type LinkedList is null | Link
type OrderedList is null | {
  int data, int order, OrderedList next
}
\end{lstlisting}
Here, we have defined a standard linked list and a specialised
``ordered'' list.  The intuition is that \lstinline{order < next.order} for each node in an \lstinline{OrderedList} (although the details of how this is done are unimportant here).   These two types are not considered identical because they have different structure (i.e. \lstinline{OrderedList} has an additional field, \lstinline{order}).  However, there is still a subtyping relationship between them (i.e. \lstinline{OrderedList} subtypes \lstinline{LinkedList}).  Thus, an instance of \lstinline{OrderedList} can be used where a \lstinline{LinkedList} was expected.  For example:

\begin{lstlisting}
function sum(LinkedList l) => int:
    if l is null:
        return 0
    else:
        return l.data + sum(l.next)
\end{lstlisting}

This defines a simple recursive function for computing the length of a \lstinline{LinkedList}.  Instances of \lstinline{OrderedList} can be passed into this function by {\em coercing} them to instances of \lstinline{LinkedList}:

\begin{lstlisting}
function sum(OrderedList l) => int:
    return sum((LinkedList) l)
\end{lstlisting}

Here, we have used an explicit coercion (i.e. a cast) from \lstinline{OrderedList} to \lstinline{LinkedList}.  This must be done explicitly because the coercion is lossy because the field \lstinline{order} is discarded during the coercion.

\begin{insight}{Lossless Coercions.}
Whiley (unlike Java) supports lossless coercion from \lstinline{int} values to \lstinline{real} values.  This is because arithmetic in Whiley is unbounded and, hence, every value of \lstinline{int} type has a corresponding value of \lstinline{real} type (though not vice-versa).
\end{insight}

\subsection{Subtyping}

An important concept in many modern programming languages is that of {\em subtyping}.  This defines a relationship between otherwise different types (i.e. which do not have identical structure).  Subtyping allows data from a variable of one type to flow into a variable of another.  However, unlike a coercion, subtyping does not physically change the value itself.  

The most common form of subtyping in Whiley is through the use of union types.  Indeed, we have encountered this already.  The following illustrates a very simple example:

\begin{lstlisting}
// A circle is defined by its position and radius
type Circle is { int x, int y, int radius }

// A rectangle is defined by its position and dimensions
type Rectangle is { int x, int y, int width, int height }

// A shape is either a circle or a rectangle
type Shape is Circle | Rectangle

// Determine the area of a shape
function area(Shape s) => real:
   if s is Rectangle:
      // case for rectangle
      return s.width * s.height
   else:
      // case for circle
      return Math.PI * s.radius * s.radius   
\end{lstlisting}

Here, we see that a \lstinline{Shape} is either a \lstinline{Rectangle} or \lstinline{Circle}.  We say that \lstinline{Rectangle} and \lstinline{Circle} are subtypes of \lstinline{Shape}.  The Whiley compiler knows that there are only two possibilities and, hence, automatically retypes variable \lstinline{s} to \lstinline{Circle} on the false branch.

\begin{insight}
A useful analogy to help understanding the concept of a subtype is that of a {\em subset}.  In this way, we think of types as representing the set of values that their variables may hold.  Then, one type is a subtype of another if its set of values is a subset of the other's.  Conversely, one type is a {\em supertype} of another if its set of values is a {\em superset} of the other's.  Furthermore, a union type ``\lstinline{T1 | T2}'' corresponds to a set union of those values represented by ``\lstinline{T1}'' and ``\lstinline{T2}''.
\end{insight}

Another situation where subtyping occurs in Whiley is with the types \lstinline{any} and \lstinline{void}.  Specifically, every type is a subtype of \lstinline{any}, whilst every type is a {\em supertype} of \lstinline{void}.  To understand this, consider the following:

\begin{lstlisting}
function toInteger(Any x) => int:
    if x is int:
        return x
    else if x is real:
        return Math.round(x)
    else if x is [any]:
        return |x|
    else:
        return 0 // default value
\end{lstlisting}

In this function, parameter \lstinline{x} can hold {\em any possible value on entry}.  It could be an integer, a real, a list, a set, a record, etc.  In this case, we have picked a few examples which we can easily convert into \lstinline{int} values.  Note that the type \lstinline{[any]} describes all possible lists, including e.g. instances of \lstinline{[int]}, \lstinline{[[int]]}, \lstinline{[char]}, etc.

% could talk about subtyping between collections here
