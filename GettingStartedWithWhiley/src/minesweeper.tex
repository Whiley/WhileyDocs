\newpage
\section{Example: Minesweeper}
In this section, we will develop a simple implementation of the well-known {\em Minesweeper} game.  Typically the minesweeper game is played through a graphical user interface, illustrated as follows:
\begin{center}
\includegraphics[width=0.35\textwidth]{../images/kmines.png}
\end{center}
Here, we can see the main aspects of the game.  The {\em game board} is a two-dimensional grid of {\em squares}.  Each square holds {\em nothing} or a {\em bomb} and is in one of the three states: {\em hidden}, {\em exposed} or {\em flagged} (with a flag).  An exposed square shows either the total number of bombs in the nine adjacent squares, referred to as its {\em rank}.  If an exposed square contains a bomb, then the game is over and the player has lost.  Flagged squares are protected and cannot be exposed unless they are {\em unflagged}.  The intuition here, is that the player marks those squares believed to contain a bomb.  

Let's analyse the above board.  In the following diagram of the above minesweeper game, gray squares represent hidden squares in the game.  For our benefit here, we've split them into two categories: those which contain a bomb (dark gray); and, those which don't: 

\begin{center}
\includegraphics[width=0.35\textwidth]{../images/kmines_analysis.png}
\end{center}

In our discussion, we'll use $(x,y)$ to indicate a position on the board where $x$ gives the horizontal component, and $y$ the vertical component.  So, for example, the squares $(2,4)$, $(4,3)$ and $(6,4)$ are all marked with a flag.  Indeed, we can see that the above player has correctly flagged the three bombs in these squares, and that there are seven remaining to be identified and flagged.  Of course, unlike us, the player cannot see exactly where the bombs are.  However, he/she can easily determine that the square $(2,6)$ must contain a bomb.  This is because the exposed square at $(1,4)$ has a rank of $1$, and a bomb is flagged at $(2,4)$.  Therefore, there can be no bomb in square $(2,5)$ as this mean the rank of square $(1,4)$ was incorrect.  Finally, the rank of the square at $(1,5)$ is $2$ with only three unexposed squares, of which one is known already to contain a bomb and the other is known {\em not} to contain a bomb.  Therefore, the $(2,6)$ must contain a bomb.

The player plays the game by repeatedly selecting a square to expose.  When all squares are exposed, except for those containing bombs, the game is over and the player wins.  However, if a square holding a bomb is exposed, then the game is over immediately and the player loses.  A {\em blank} square is one with no adjacent bombs.  When a blank square is exposed, every adjacent blank square is recursively exposed.

\subsection{Squares}
We're now going to begin implementing the game of Minesweeper in Whiley.  To start with, we'll implement the game board in Whiley and provide functions for manipulating it; then, we'll implement the gameplay itself.  

The first aspect of the game board we'll implement is the concept of a {\em square}.  There are essentially two broad categories of square in the game: {\em exposed squares} and {\em hidden squares}.  Therefore, our implementation will reflect this.  Exposed squares either have a {\em rank} or are {\em blank} (i.e. have a rank of zero).  Furthermore, they may or may not hold a bomb.  We can implement this in Whiley like so:
\begin{lstlisting}
type ExposedSquare is { 
  int rank,       // Number of bombs in adjacent squares
  bool holdsBomb  // true if the square holds a bomb
}
\end{lstlisting}
Here, we can see that an integer field called \lstinline{rank} is used to store the rank of the square.  Likewise, a boolean field called \lstinline{holdsBomb} is used to indicate whether or not the square holds a bomb.  To simply creating values of type \lstinline{ExposedSquare}, it is common to additionally provide one or more {\em constructors}.  These are functions of the same name which create values of the given type.  Here is our \lstinline{ExposedSquare} constructor:

\begin{lstlisting}
// ExposedSquare constructor
function ExposedSquare(int rank, bool bomb) => ExposedSquare:
    return { rank: rank, holdsBomb: bomb }
\end{lstlisting}


Hidden squares may or may not hold a bomb, and may or may not have been flagged.  We can implement this in Whiley as follows:

\begin{lstlisting}
type HiddenSquare is { 
  bool holdsBomb,  // true if the square holds a bomb
  bool flagged     // true if the square is flagged
}

function HiddenSquare(bool bomb, bool flag) => HiddenSquare:
    return { holdsBomb: bomb, flagged: flag }
\end{lstlisting}

As before, a boolean field called \lstinline{holdsBomb} is used to signal whether or not the square holds a bomb.  Likewise, a boolean field called \lstinline{flagged} signals whether or not the square is flagged.  

We can now define the concept of a square in our Whiley implementation by combining the notions of exposed and hidden squares together as follows:

\begin{lstlisting}
type Square is ExposedSquare | HiddenSquare
\end{lstlisting}

Here, the type \lstinline{Square} is a union of the types \lstinline{ExposedSquare} and \lstinline{HiddenSquare}.  In otherwords, it is either an \lstinline{ExposedSquare} or a \lstinline{HiddenSquare}.  Notice that we don't provide a constructor for \lstinline{Square}.  This is because a \lstinline{Square} is merely the composition of two existing types with their own constructors.

\subsection{Board}

Using our above \lstinline{Square} data type, we can now define the game board in our Whiley implementation as follows:

\begin{lstlisting}
type Board is {
   [Square] squares,  // List of squares making up the board
   int width,         // Width of the game board (in squares)
   int height        // Height of the game board (in squares)
}
\end{lstlisting}

The main component of \lstinline{Board} is the \lstinline{squares} list.  Although this is a one-dimensional list, we'll see shortly that it is treated a two dimensional way.  The remaining fields record the width and height of the board, which is needed in order to safely manipulate the board.  To accompany this data type, we define a simple constructor as follows:
\begin{lstlisting}
// Create a board of given dimensions which contains no bombs, and
// where all squares are hidden.
Board Board(int width, int height):
    [Square] squares = []
    //
    for i in 0 .. width * height:
      squares = squares ++ [HiddenSquare(false,false)]
    //
    return {
        squares: squares,
        width: width,
        height: height
    }
\end{lstlisting}
This constructor creates a \lstinline{Board} of given width and height containing only hidden squares and no bombs.  Later, we will return to consider how to randomly place bombs on the board..

We will return to define a constructor for \lstinline{Board} shortly, but first we will provide some simpler helper functions for updating the board.  First, we provide a function to read the \lstinline{Square} at a given position on a \lstinline{Board}:

\begin{lstlisting}
// Return the square on a given board at a given position
function getSquare(Board b, int col, int row) => Square:
    int rowOffset = b.width * row // calculate start of row
    return b.squares[rowOffset + col]
\end{lstlisting}

This function performs a simple calculation to determine the start of the row in the \lstinline{Board.squares} list.  To understand this calculation, we need to view the board in a 1-Dimensional manner, as follows:

\begin{center}
\includegraphics[width=0.95\textwidth]{../images/kmines_flat.png}
\end{center}

Here, we can see how each row is laid out in the 1-Dimensional list \lstinline{Board.squares}.  To calculate the start of a given row, we multiple the row number by the width of the board.  Then, to calculate a given column within that row, we simply add the column number.  For example, the position $(3,2)$ represents column 2, row 3; therefore, the position in the example board above would be: $(2 * 9) + 3 = 21$.

The corresponding function to the above provides a way to change the square at a given position on the board:

\begin{lstlisting}
// Set the square on a given board at a given position
function setSquare(Board b, int col, int row, Square sq) => Board:
    int rowOffset = b.width * row // calculate start of row
    b.squares[rowOffset + col] = sq
    return b
\end{lstlisting}

Here, the same calculation is performed as before to determine the actual position within the \lstinline{Board.squares} list.  This time, the \lstinline{Board.squares} array is updated with the new \lstinline{Square}.  Note that we must return the updated board in order for this change to be visible (see~\ref{value_semantics} for more on this).  Notice also that we are not attempting to control how the \lstinline{Board.squares} list may be updated.  That is, any \lstinline{Square} can be passed into this function, even it doesn't make sense in the wider context of the game.  This is because, at this stage, we are simply providing general-purpose mechanisms for manipulating a \lstinline{Board}.

\subsection{Game Play}

Having defined the basic data types for the Minesweeper game above, we'll now put this altogether to implement the main actions of the game.  In particular, the user can {\em flag squares} and {\em expose squares}.  We also need to know when its {\em game over} and the player has either {\em won} or {\em lost}.  The easiest of these is that for flagging squares:

\begin{lstlisting}
// Flag (or unflag) a given square on the board.  If this operation is not permitted, then do nothing
// and return the board; otherwise, update the board accordingly.
function flagSquare(Board b, int col, int row) => Board:
   Square sq = getSquare(b,col,row)
   // check whether permitted to flag
   if sq is HiddenSquare:
      // yes, is permitted so reverse flag status
      sq.flagged = !sq.flagged
      // and Update board
      setSquare(b,col,row,sq)
   //
   return b
\end{lstlisting}

This function simply checks whether the square at the given position in the board is hidden or not.  If not, then nothing is changed.  Otherwise, the flagged status of that square is flipped (i.e. if it was not flagged then it is now, etc).  As before,  we must return the updated board in order for any change to be visible (see~\ref{value_semantics} for more on this).