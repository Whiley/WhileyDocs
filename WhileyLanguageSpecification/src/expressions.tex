\chapter{Expressions}
The majority of work performed by a Whiley program is through the execution of \glspl{expression}.  Every expression produces a \gls{value} and may have additional side effects.

\section{Abrupt Execution}

\section{Evaluation Order}

% =======================================================================
% Multi Expression
% =======================================================================

\section{Multi Expressions}
\label{c_expr_tuple}
A multi-expression is an expression composed of two or more unit expressions and which returns a tuple value (\S\ref{c_types_tuple}).  The operands of a multi-expression are evaluated in a strict left-to-right order.

\begin{syntax}
  \verb+TupleExpr+ & $::=$ & \verb+UnitExpr+ \big(\ \token{,}\ \verb+UnitExpr+\ \big)$^+$ \\
\end{syntax}

\paragraph{Example.}  The following example illustrates the use of a multi-expression:

\begin{lstlisting}
function scale(real x, real y, real p) => (real,real):
    return (x*p), (y*p)
\end{lstlisting}

This function accepts three \lstinline{real} values and returns two.  In the body, the \lstinline{return} statement contains a multi-expression which produces a tuple composed of two \lstinline{real} values.

% =======================================================================
% Unit Expression
% =======================================================================

\section{Unit Expressions}
\label{c_expr_unit}

A unit expression is an expression which returns exactly one value.  There is a large range of possible unit expressions, including comparators, arithmetic operators, logical operators, etc.

\begin{syntax}
  \verb+UnitExpr+ & $::=$ & \verb+LogicalExpr+\\
                  &  $|$  & \verb+BitwiseExpr+\\
                  &  $|$  & \verb+ConditionExpr+\\
                  &  $|$  & \verb+QuantifierExpr+\\
                  &  $|$  & \verb+AppendExpr+\\
                  &  $|$  & \verb+RangeExpr+\\
                  &  $|$  & \verb+ShiftExpr+\\
                  &  $|$  & \verb+AdditiveExpr+\\
                  &  $|$  & \verb+MultiplicativeExpr+\\
                  &  $|$  & \verb+AccessExpr+\\
                  &  $|$  & \verb+TermExpr+\\
                
\end{syntax}

% =======================================================================
% Logical Expressions
% =======================================================================

\section{Logical Expressions}
\label{c_expr_logical}

A logical expression operates on values of \lstinline{bool} type (\S\ref{c_types_bool}) to produce another \lstinline{bool} value.  The {\em if-and-only-if (iff)} operator, \lstinline{<==>}, returns \lstinline{true} if either both operands are \lstinline{true} or both are \lstinline{false}.  The {\em implication} operator, \lstinline{==>}, returns \lstinline{true} if either the left operand is \lstinline{false}, or both operands are \lstinline{true}.  The {\em logical OR} operator returns \lstinline{true} if either operand is \lstinline{true}, whilst the {\em logical AND} operator returns \lstinline{true} if both operands are \lstinline{true}.

\begin{syntax}
  \verb+LogicalExpr+ & $::=$ & \verb+LogicalOrExpr+ \big[ \token{<==>} \verb+LogicalExpr+\ \big]\\
                     &  $|$  & \verb+LogicalOrExpr+ \big[ \token{==>} \verb+LogicalExpr+\ \big]\\
  &&\\
  \verb+LogicalOrExpr+ & $::=$ & \verb+LogicalAndExpr+ \\
                           & $|$ & \verb+LogicalOrExpr+ \token{||} \verb+LogicalAndExpr+\\
  &&\\
  \verb+LogicalAndExpr+ & $::=$ & \verb+BitwiseExpr+ \\
                            & $|$ & \verb+LogicalAndExpr+ \token{\&\&} \verb+BitwiseExpr+\\
\end{syntax}

\paragraph{Example.}  The following examples illustrate some of the logical operators:

\begin{lstlisting}
function implies(bool x, bool y) => bool:
    return !x || y

function iff(bool x, bool y) => bool:
    return implies(x,y) && implies(y,x)
\end{lstlisting}

The function \lstinline{implies()} implements the well-known equivalence between implication and logical OR.  The function \lstinline{iff()} implements the well-known equivalence between implication and iff.

% =======================================================================
% Bitwise Expressions
% =======================================================================

\section{Bitwise Expressions}
\label{c_expr_bitwise}
A bitwise expression operators on values of \lstinline{byte} type (\S\ref{c_types_byte}).  The {\em bitwise OR} operator, \lstinline{&}

\begin{syntax}
  \verb+BitwiseExpr+ & $::=$ & \verb+BitwiseOrExpr+ \\
  &&\\
  \verb+BitwiseOrExpr+ & $::=$ & \verb+BitwiseXorExpr+ \\
                           & $|$ & \verb+BitwiseOrExpr+ \token{|} \verb+BitwiseXorExpr+\\
  &&\\
  \verb+BitwiseXorExpr+ & $::=$ & \verb+BitwiseAndExpr+ \\
                            & $|$ & \verb+BitwiseXorExpr+ \token{\^} \verb+BitwiseAndExpr+\\
  &&\\
  \verb+BitwiseAndExpr+ & $::=$ & \verb+ConditionExpr+ \\
                            & $|$ & \verb+BitwiseAndExpr+ \token{\&\&} \verb+ConditionExpr+\\

\end{syntax}

\paragraph{Description.}

\paragraph{Examples.}

\paragraph{Notes.} 

% =======================================================================
% Condition Expressions
% =======================================================================

\section{Condition Expressions}
\label{c_expr_condition}

\begin{syntax}
  \verb+ConditionExpr+ & $::=$ &\\
\end{syntax}

\paragraph{Description.}

\paragraph{Examples.}

\paragraph{Notes.} 

% =======================================================================
% Quantifier Expressions
% =======================================================================

\section{Quantifier Expressions}
\label{c_expr_quantifier}

\begin{syntax}
\verb+QuantExpr+ & $::=$ & \big(\ \token{no} $|$ \token{some} $|$
\token{all}\ \big)\ \token{\{}\\
&&\ \verb+Ident+ \token{in} \verb+Expr+ \big( \token{,}\ \verb+Ident+\
\token{in}\ \verb+Expr+\ \big)$^+$\ \token{|}\ \verb+LogicalExpr+\\
&& \token{\}}\\
\end{syntax}

\paragraph{Description.}

\paragraph{Examples.}

\paragraph{Notes.} 

% =======================================================================
% Append Expressions
% =======================================================================

\section{Append Expressions}
\label{c_expr_append}

\begin{syntax}
  \verb+AppendExpr+ & $::=$ & \verb+RangeExpr+\ \big(\ \token{++}\
  \verb+RangeExpr+\ \big)$^*$\\
\end{syntax}

\paragraph{Description.}

\paragraph{Examples.}

\paragraph{Notes.} 


% =======================================================================
% Range Expressions
% =======================================================================

\section{Range Expressions}
\label{c_expr_range}

\begin{syntax}
  \verb+RangeExpr+ & $::=$ & \verb+ShiftExpr+\ \big[\ \token{..}\ \verb+ShiftExpr+\ \big]\\
\end{syntax}

\paragraph{Description.}

\paragraph{Examples.}

\paragraph{Notes.} 

% =======================================================================
% Shift Expressions
% =======================================================================

\section{Shift Expressions}
\label{c_expr_shift}

\begin{syntax}
  \verb+ShiftExpr+ & $::=$ & \verb+AdditiveExpr+\ \big[\ \big(\
  \token{<<} $|$ \token{>>}\ \big)\ \verb+AdditiveExpr+ \big]
\end{syntax}

\paragraph{Description.}

\paragraph{Examples.}

\paragraph{Notes.} 

% =======================================================================
% Addititve and Multiplicative Expressions
% =======================================================================

\section{Additive/Multiplicative Expressions}
\label{c_expr_addmul}

\begin{syntax}
  \verb+AdditiveExpr+ & $::=$ &\\
  \verb+MultiplicativeExpr+ & $::=$ &\\
\end{syntax}

\paragraph{Description.}

\paragraph{Examples.}

\paragraph{Notes.} 

% =======================================================================
% Access Expressions
% =======================================================================

\section{Access Expressions}
\label{c_expr_access}

\begin{syntax}
  \verb+AccessExpr+ & $::=$ &\\
\end{syntax}

\paragraph{Description.}

\paragraph{Examples.}

\paragraph{Notes.} 

% =======================================================================
% Term Expressions
% =======================================================================

\section{Term Expressions}
\label{c_expr_term}

\begin{syntax}
  \verb+TermExpr+ & $::=$ &\\
\end{syntax}

\paragraph{Description.}

\paragraph{Examples.}

\paragraph{Notes.} 

% =======================================================================
% Dereference Expressions
% =======================================================================

\section{Dereference Expressions}
\label{c_expr_dereference}


 The dereference operation ``\lstinline{e->f}'' is a short-hand notation for ``\lstinline{(*e).f}'' and can be used when \lstinline{e} has effective record type (\S\ref{c_types_effective_records}).

