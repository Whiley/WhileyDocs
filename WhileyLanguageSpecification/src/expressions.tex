
\chapter{Expressions}
\label{c_expressions}
The majority of work performed by a Whiley program is through the execution of \glspl{expression}.  Every expression produces a \gls{value} and may have additional side effects.

\section{Evaluation Order}

The operands for operators in Whiley are evaluated in a specific left-to-right {\em evaluation order}.  This always respects parentheses and operator precedence.  Furthermore, aside from the short-circuiting operators (\S\ref{c_expr_logical_connectives}), operands are always fully evaluated before any part of the operation is performed.

\subsection{Operator Precedence}

To determine the evaluation order for mixed-operator expressions without explicit parenthesis, a fixed {\em operator precedence} is used.  This is first determined by {\em operator class}:

\begin{enumerate}
\item {\bf Unary Expressions.} This operator class represents operators which take exactly one operand.  This class takes highest precedence, and includes operators such as arithmetic negation (\S\ref{c_expr_negation}) and logical not (\S\ref{c_expr_logical_not}).
\item {\bf Binary (Infix) Expressions.}  This operator class represents operators which accept two operands with an infix syntax.  This class includes the usual range of common binary operators, such as arithmetic operators (\S\ref{c_expr_additive},\S\ref{c_expr_multiplicative}), logical connectives (\S\ref{c_expr_logical_connectives}), etc.

\item {\bf Binary (Mixfix) Expressions.}  This operator class represents operators which accept two operands but which are non-infix operators and, hence, precedence is not ambiguous.  This class includes the array access (\S\ref{c_expr_array_access}) and field access operator (\S\ref{c_expr_access}).
\item {\bf N-Ary Expressions.}  This operator class represents operators which accept an arbitrary number of operands.  This class includes array constructors (\S\ref{c_expr_array_initialiser}), record constructors (\S\ref{c_expr_record_initialiser}), etc.
\end{enumerate}

Within the class of binary infix expressions, an explicit precedence rank is given for each operator:

\begin{syntax}
\begin{tabular}{cl}
1 & \token{*} \token{/}\\
&\\
2 & \token{+} \token{-}\\
&\\
3 & \token{==} \token{!=} \token{<} \token{<=} \token{>=} \token{>}\\
&\\
4 & \token{\&}\\
&\\
5 & \token{|}\\
&\\
6 & $\token{\^}$\\
&\\
7 & \token{\&\&}\\
&\\
8 & \token{||}\\
&\\
9 & \token{==>}\\
&\\
10 & \token{<==>}\\
\end{tabular}
\end{syntax}

Lower ranked operators bind more tightly (i.e. take higher precedence) than higher ranked operators.

% =======================================================================
% Unit Expression
% =======================================================================

\section{Unit Expressions}
\label{c_expr_unit}

An expression returns exactly one value.  There is a large range of possible unit expressions, including comparators, arithmetic operators, logical operators, etc.

\begin{syntax}
  \verb+Expr+ & $::=$ & \verb+ArithmeticExpr+\\
                  &  $|$  & \verb+BitwiseExpr+\\
                  &  $|$  & \verb+CastExpr+\\
                  &  $|$  & \verb+EqualityExpr+\\
                  &  $|$  & \verb+InvokeExpr+\\
                  &  $|$  & \verb+LambdaExpr+\\
                  &  $|$  & \verb+LogicalExpr+\\
                  &  $|$  & \verb+ArrayExpr+\\
                  &  $|$  & \verb+RecordExpr+\\
                  &  $|$  & \verb+ReferenceExpr+\\
                  &  $|$  & \verb+TermExpr+\\
                
\end{syntax}

% =======================================================================
% Arithmetic Expressions
% =======================================================================

\section{Arithmetic Expressions}
\label{c_expr_arithmetic}

Arithmetic expressions operate on values of numeric type (currently just \lstinline{int}).

\begin{syntax}
  \verb+ArithmeticExpr+ & $::=$ & \verb+ArithmeticNegationExpr+\\
                  &  $|$  & \verb+ArithmeticRelationalExpr+\\
                  &  $|$  & \verb+ArithmeticAdditiveExpr+\\
                  &  $|$  & \verb+ArithmeticMultiplicativeExpr+\\
\end{syntax}

% =======================================================================
% Negation Expression
% =======================================================================

\subsection{Negation Expressions}
\label{c_expr_negation}

A negation expression accepts one argument of numeric type and produces a result of matching type.  Specifically, the {\em negation operator} mathematically negates the given value, which is always equivalent to subtracting the operand from zero.

\begin{syntax}
\verb+ArithmeticNegationExpr+ & $::=$ & \token{-}\ \verb+Expr+\\
\end{syntax}

\paragraph{Example.} The following illustrates the negation operator:

\lstinputlisting{../examples/ch6/eg_13.whiley}

% =======================================================================
% Relational Expressions
% =======================================================================

\subsection{Relational Expressions}
\label{c_expr_relational}
Relational expressions are either {\em strict} (where only inequality is tested)  or {\em non-strict} (where both equality and inequality are tested).  The {\em less-than comparator}, \lstinline{<}, and {\em greater-than comparator}, \lstinline{>}, are strict.  Conversely, the {\em less-than-or-equal comparator}, \lstinline{<=}, and {\em greater-than-or-equal comparator}, \lstinline{>=}, are non-strict.

\begin{syntax}
  \verb+ArithmeticRelationalExpr+ & $::=$ & \verb+Expr+ \token{<} \verb+Expr+\\
  & $|$ & \verb+Expr+ \token{<=} \verb+Expr+\\
  & $|$ & \verb+Expr+ \token{=>} \verb+Expr+\\
  & $|$ & \verb+Expr+ \token{>} \verb+Expr+\\
\end{syntax}

\paragraph{Example.} The following example illustrates the strict inequality comparators:

\lstinputlisting{../examples/ch6/eg_5.whiley}

This function compares two integer arguments and returns the ``sign'' of their comparison.  The strict inequality comparators are used so the case where \lstinline{x == y} can be distinguished.

% =======================================================================
% Addititve and Multiplicative Expressions
% =======================================================================

\subsection{Additive Expressions}
\label{c_expr_additive}

An additive expression accepts two arguments of type \lstinline{int} and produces a result of the same type.  The {\em addition operator}, \lstinline{+}, adds both arguments together whilst the {\em subtraction operator}, \lstinline{-}, subtracts its right argument from its left argument.

\begin{syntax}
  \verb+ArithmeticAdditiveExpr+ & $::=$ & \verb+Expr+ \big(\ \token{+} $|$ \token{-}\ \big) \verb+Expr+\\
\end{syntax}

\paragraph{Example.}  The following illustrates the additive operators:

\lstinputlisting{../examples/ch6/eg_11.whiley}

This function simply computes the difference between its two arguments using the subtraction operator.


\subsection{Multiplicative Expressions}
\label{c_expr_multiplicative}

A multiplicative expression accepts two arguments of type \lstinline{int} and produces a result of the same type.  The {\em multiplication operator}, \lstinline{*}, multiplies both arguments together whilst the {\em division operator}, \lstinline{/}, divides its left argument by its right argument.  Finally, the {\em remainder operator} returns the remainder of its operands from an implied division.

\begin{syntax}
  \verb+ArithmeticMultiplicativeExpr+ & $::=$ & \verb+Expr+ \big(\ \token{*} $|$ \token{/} $|$ \token{\%}\ \big) \verb+Expr+\\
\end{syntax}

\paragraph{Example.} The following illustrates the remainder operator:

\lstinputlisting{../examples/ch6/eg_12.whiley}

This function accepts a non-negative integer and uses this to index into an array.  To ensure the array access is within bounds, the remainder operator is used.  Furthermore, the function requires the array is non-empty to prevent a fault with the remainder operator.

\paragraph{Notes.}  For division, the right operator must be non-zero otherwise a \gls{fault} is raised, and likewise for remainder.  For integer division, the result is rounded towards zero.  For a remainder operation, the result may be negative (e.g. \lstinline{-4 % 3 == -1}).

% =======================================================================
% Array Expressions
% =======================================================================

\section{Array Expressions}
\label{c_expr_array}

Array expressions operate on values of array type (e.g. \lstinline{int[]}, \lstinline{(bool|byte)[]}, etc).

\begin{syntax}
  \verb+ArrayExpr+ & $::=$ &\\
  & $|$ & \verb+ArrayLengthExpr+\\
  & $|$ & \verb+ArrayAccessExpr+\\
  & $|$ & \verb+ArrayGeneratorExpr+\\
  & $|$ & \verb+ArrayInitialiserExpr+\\
\end{syntax}

% =======================================================================
% LengthOf Expression
% =======================================================================

\subsection{Length Expressions}
\label{c_expr_length}

The {\em lengthof operator} accepts a value of array type, and produces a value of \lstinline{int} type which equals the number of elements in the array.

\begin{syntax}
\verb+ArrayLengthExpr+ & $::=$ & \token{|}\ \verb+Expr+\ \token{|}\\
\end{syntax}

\paragraph{Example.} The following example illustrates the lengthof operator:

\lstinputlisting{../examples/ch6/eg_19.whiley}

The above function iterates through all elements in an array looking for the first which is above a given item.  The length operator is used to ensure this iteration remains within bounds.

% =======================================================================
% Array Access Expressions
% =======================================================================

\subsection{Access Expressions}
\label{c_expr_array_access}

An array access expression accepts a array argument with one operand and produces a value of the array element type.  The {\em index-of operator} returns the element at the given operand position in the array.  

\begin{syntax}
  \verb+ArrayAccessExpr+ & $::=$ & \verb+Expr+\ \token{[}\ \verb+Expr+\ \token{]}\\
\end{syntax}

\paragraph{Examples.} The following example illustrates the array access operator:

\lstinputlisting{../examples/ch6/eg_20.whiley}

The above function determines whether a given array of integers is sorted from smallest to largest.  The array access operator is used to access successive elements in the array. 

% =======================================================================
% Generator Expressions
% =======================================================================

\subsection{Generator Expressions}
\label{c_expr_generator}

An {\em array generator} accepts two arguments and produces a value of array type.  The second argument must be of \lstinline{int} type and the array produced contains exactly this many occurrences of the first argument.

\begin{syntax}
  \verb+ArrayGeneratorExpr+ & $::=$ & \token{[}\ \verb+Expr+\ \token{;}\ \verb+Expr+\ \token{]}\\
\end{syntax}

\paragraph{Examples.} The following example illustrates an array generator:

\lstinputlisting{../examples/ch6/eg_8.whiley}

This function constructs a array by prepending a given element onto the front of a given array.  The array generator is used to construct the initial array of values whose size is one larger than the original array.

% =======================================================================
% Array Initialiser
% =======================================================================

\subsection{Array Initialiser}
\label{c_expr_array_initialiser}
An {\em array initialiser} accepts zero or more operands and produces a value of array type.  Array initialisers are used to construct arrays from their constituent elements.  

\begin{syntax}
  \verb+ArrayInitialiserExpr+ & $::=$ & \token{[}\ \big[\ \verb+Expr+\ \big(\ \token{,}\ \verb+Expr+\ \big)$^*$\ \big] \token{]}\\
\end{syntax}

\paragraph{Example.}  The following example illustrates an array initialiser:

\lstinputlisting{../examples/ch6/eg_21.whiley}

The above function converts an integer value into its string representation.  An array initialiser is used to map integer values to their corresponding digits.  An empty array initialiser is also used to initialise the string.

% =======================================================================
% Bitwise Expressions
% =======================================================================

\section{Bitwise Expressions}
\label{c_expr_bitwise}

Bitwise expressions operate on values of \lstinline{byte} type.

\begin{syntax}
  \verb+BitwiseExpr+ & $::=$ & \verb+BitwiseComplementExpr+\\
                  &  $|$  & \verb+BitwiseBinaryExpr+\\
                  &  $|$  & \verb+BitwiseShiftExpr+\\
\end{syntax}

% =======================================================================
% Bitwise Complement Expression
% =======================================================================

\subsection{Complement Expressions}
\label{c_expr_bitwise_complement}
The {\em bitwise complement operator} accepts an argument of \lstinline{byte} type (\S\ref{c_types_byte}) and produces a result of matching type.  The operator returns bitwise complement of the argument; that is, where the sign of each bit is reversed.

\begin{syntax}
\verb+BitwiseComplementExpr+ & $::=$ & \token{$\sim$}\ \verb+Expr+\\
\end{syntax}

\paragraph{Example.}  The following example illustrates the bitwise complement operator:

\lstinputlisting{../examples/ch6/eg_15.whiley}

\subsection{Binary Expressions}
\label{c_expr_bitwise_binary}
A bitwise binary expression operates on values of \lstinline{byte} type (\S\ref{c_types_byte}).  The {\em bitwise AND} operator, \lstinline{&}, performs a logical AND between the respective bits of each operand, and produces a \lstinline{byte}.   The {\em bitwise OR} operator, \lstinline{|}, performs a logical OR between the respective bits of each operand, and produces a \lstinline{byte}.  The {\em bitwise exclusive-OR} operator, \lstinline{^}, performs a logical exclusive-OR between the respective bits of each operand, and produces a \lstinline{byte}.

\begin{syntax}
  \verb+BitwiseBinaryExpr+ & $::=$ & \verb+Expr+ \big(\ \token{\&} $|$ \token{|}\ $|$ \token{\^}\ \big) \verb+Expr+\\
\end{syntax}

\paragraph{Example.}  The following example illustrates the bitwise OR operator:

\lstinputlisting[firstline=3]{../examples/ch6/eg_3b.whiley}

These functions provide mechanisms for manipulating a byte of ``flags'', as determined by the constant identifiers.  The bitwise OR operator is used to ensure a given bit is set, whilst the bitwise AND operator is used to check whether one is set or not.  This example also illustrates the left-shift operator (\S\ref{c_expr_shift}).

% =======================================================================
% Shift Expressions
% =======================================================================

\subsection{Shift Expressions}
\label{c_expr_shift}

A bitwise shift expression accepts an argument of \lstinline{byte} type (left) and one of \lstinline{int} type (right) and produces a value of \lstinline{byte} type.  The {\em left shift operator}, \lstinline{<<}, shifts the bits of a \lstinline{byte} in an upwards direction, such that the most significant bit is discarded and the least significant bit assigned \lstinline{0}.  The {\em right shift operator}, \lstinline{>>}, shifts bits in a downwards direction, such that the least significant bit is discarded and the most significant bit assigned \lstinline{0}.  

\begin{syntax}
  \verb+BitwiseShiftExpr+ & $::=$ & \verb+Expr+\ \big[\ \big(\
  \token{<<} $|$ \token{>>}\ \big)\ \verb+Expr+ \big]
\end{syntax}

\paragraph{Examples.} The following illustrates the left shift operator:

\lstinputlisting[firstline=3]{../examples/ch6/eg_10.whiley}

This function accepts an integer between \lstinline{0} and \lstinline{255} and converts this into an appropriate bit representation.  The left shift operator is used to maintain an internal mask for the bit currently being initialised.

% =======================================================================
% Cast Expressions
% =======================================================================

\section{Cast Expressions}
\label{c_expr_cast}
A {\em cast operator} accepts a value of one type and returns a value of a different, but equivalent, type and this may result in a change of the underlying representation.  

\begin{syntax}
\verb+CastExpr+ & $::=$ & \token{(}\ \verb+DefiniteType+ \token{)}\ \verb+Expr+\\
\end{syntax}

\paragraph{Example.}  The following illustrates a cast operator being used:

\lstinputlisting[firstline=3]{../examples/ch6/eg_16.whiley}

This function converts a record containing three {\em int} fields into one containing two {\em int} fields.  This requires that each field in the latter is a valid field in the former.

% =======================================================================
% Equality Expression
% =======================================================================

\section{Equality Expressions}
\label{c_expr_equality}
The {\em equality comparator}, \lstinline{==}, tests whether two values are equal.  Likewise, the {\em inequality comparator}, \lstinline{!=}, tests whether two values are {\em not} equal.

\begin{syntax}
  \verb+EqualityExpr+ & $::=$ &\\
  & $|$ & \verb+Expr+ \token{==} \verb+Expr+\\
  & $|$ & \verb+Expr+ \token{!=} \verb+Expr+\\
\end{syntax}

\paragraph{Example.}

The following example illustrates an equality expression:

\lstinputlisting{../examples/ch6/eg_4.whiley}

This function checks whether a given integer is contained in an array of integers.  This is done by iterating each element of the array and comparing it against the given item.

% =======================================================================
% Invocation Expression
% =======================================================================

\section{Invoke Expressions}
\label{c_expr_invoke}
A {\em function or method invocation} executes a named function or method declared in a given source file.  An {\em indirect function or method invocation} executes a function determined by a given expression.  An invocation passes arguments of appropriate number and type to the executed function or method.  An invocation may also return one or more values which can be subsequently used.

% Could also discuss function or method lookup; this is really part of typing.

\begin{syntax}
  \verb+InvokeExpr+ & $::=$ & \verb+Name+ \big[\ \verb+LifetimeArgsList+\ \big]\ \verb+ArgsList+\\
  \verb+IndirectInvokeExpr+ & $::=$ & \verb+Expr+\ \big[\ \verb+LifetimeArgsList+\ \big]\ \verb+ArgsList+\\
&&\\
\verb+ArgsList+ & $::=$ & \token{(}\ \big[\ \verb+Expr+\ \big(\ \token{,}\ \verb+Expr+\ \big)$^*$\ \big]\ \token{)}\\
\verb+LifetimeArgsList+ & $::=$ & \token{<}\ \verb+Lifetime+\ \big(\ \token{,}\ \verb+Lifetime+\ \big)$^*$\ \token{>}\\
\end{syntax}

\paragraph{Example.}

The following example illustrates a function invocation:

\lstinputlisting{../examples/ch6/eg_17.whiley}

This example illustrates one function being called from another.  Both functions have the same name and are said to {\em overload} one another.  Function resolution identifies the appropriate function based on the number and type of arguments supplied.  

% =======================================================================
% LambdaExpression
% =======================================================================

\section{Lambda Expressions}
\label{c_expr_lambda}

A {\em lambda expression} creates an anonymous function or method which can accept zero or more arguments and whose return type is inferred from the body of the lambda.

\begin{syntax}
  \verb+LambdaExpr+ & $::=$ & \token{\&} \big[\ \verb+ContextLifetimes+\ \big]\ \big[\ \verb+LifetimeParameters+\ \big]\\
                    &       & \token{(} \big[\ \verb+Type+\ \verb+Ident+\ \big( \token{,}\ \verb+Type+ \verb+Ident+ \big)$^*$\ \big] \token{->} \verb+Expr+ \token{)}\\
\end{syntax}

\paragraph{Example.}

The following example illustrates a lambda expression:

\lstinputlisting{../examples/ch6/eg_25.whiley}

This function illustrates the classical {\em map} function which applies a function to all elements of a collection.  In this case, a lambda is used to create a function which adds a constant value to its argument.  This lambda is used to implement \lstinline{addAll()} in terms of \lstinline{map()}.

% =======================================================================
% Logical Expressions
% =======================================================================

\section{Logical Expressions}
\label{c_expr_logical}

Logical expressions operate on values of \lstinline{bool} type.

\begin{syntax}
  \verb+LogicalExpr+ & $::=$ &\\
  & $|$ & \verb+LogicalNotExpr+\\
  & $|$ & \verb+LogicalBinaryExpr+\\
  & $|$ & \verb+LogicalQuantExpr+\\
\end{syntax}


% =======================================================================
% Logical Not Expression
% =======================================================================

\subsection{Not Expressions}
\label{c_expr_logical_not}

The {\em logical not} operator accepts an argument of \lstinline{bool} type and produces a value of \lstinline{bool}.  The value returned is the logical opposite of the argument.  

\begin{syntax}
\verb+LogicalNotExpr+ & $::=$ & \token{!}\ \verb+Expr+\\
\end{syntax}

\paragraph{Example.}

The following example illustrates the logical not operator:

\lstinputlisting{../examples/ch6/eg_26.whiley}

This function computes the maximum of two \lstinline{int} values.  The expression \lstinline{!(a < b)} is equivalent to \lstinline{a >= b} and is used purely to illustrate the logical not operator.

% =======================================================================
% Logical Binary Expressions
% =======================================================================

\subsection{Connective Expressions}
\label{c_expr_logical_connectives}

A logical connective operates on values of \lstinline{bool} type (\S\ref{c_types_bool}) to produce another \lstinline{bool} value.  The {\em if-and-only-if (iff)} operator, \lstinline{<==>}, returns \lstinline{true} if either both operands are \lstinline{true} or both are \lstinline{false}.  The {\em implication} operator, \lstinline{==>}, returns \lstinline{true} if either the left operand is \lstinline{false}, or both operands are \lstinline{true}.  The {\em logical OR} operator returns \lstinline{true} if either operand is \lstinline{true}, whilst the {\em logical AND} operator returns \lstinline{true} if both operands are \lstinline{true}.

\begin{syntax}

  \verb+LogicalBinaryExpr+ & $::=$ & \verb+Expr+ \big(\ \token{<==>} $|$ \token{==>} $|$ \token{\&\&} $|$ \token{||}\ \big) \verb+Expr+\\
\end{syntax}

\paragraph{Example.}  The following examples illustrate some of the logical operators:

\lstinputlisting{../examples/ch6/eg_2.whiley}

The function \lstinline{implies()} implements the well-known equivalence between implication and logical OR.  The function \lstinline{iff()} implements the well-known equivalence between implication and iff.

% =======================================================================
% Quantifier Expressions
% =======================================================================

\subsection{Quantifier Expressions}
\label{c_expr_quantifier}

A quantifier operates over an array of values and produces a value of \lstinline{bool} type.  The {\em universal quantifier}, \lstinline{all}, returns \lstinline{true} if the given expression evaluates to \lstinline{true} for every element in the array, and \lstinline{false} otherwise.  The {\em existential quantifier}, \lstinline{some}, returns \lstinline{false} if the given expression evaluates to \lstinline{false} for every element in the array, and \lstinline{true} otherwise.  The {\em inverted universal quantifier}, \lstinline{no}, returns \lstinline{true} if the given expression evaluates to \lstinline{false} for every element in the array, and \lstinline{false} otherwise

\begin{syntax}
\verb+LogicalQuantExpr+ & $::=$ & \big(\ \token{no} $|$ \token{some} $|$
\token{all}\ \big)\ \token{\{}\\
&&\ \verb+Ident+ \token{in} \verb+Expr+ \big( \token{,}\ \verb+Ident+\
\token{in}\ \verb+Expr+\ \big)$^*$\\
&&  \token{|}\ \verb+Expr+\ \token{\}}\\
\end{syntax}

\paragraph{Examples.}  The following example illustrates the universal quantifier:

\lstinputlisting{../examples/ch6/eg_18.whiley}

Here, the type \lstinline{natlist} represents those integer arrays for which every element is a natural number (i.e. greater-or-equal to zero).

% =======================================================================
% Record Expressions
% =======================================================================

\section{Record Expressions}
\label{c_expr_record}

Record expressions operate on values of record type (e.g. \lstinline|{int x, int y}|, etc).

% =======================================================================
% Field Access Expressions
% =======================================================================

\subsection{Access Expressions}
\label{c_expr_access}

The field access operator accepts a value of record type and returns the value held in a given field.

\begin{syntax}
  \verb+FieldAccessExpr+ & $::=$ & \verb+Expr+\ \token{.}\ \verb+Ident+\\
\end{syntax}

\paragraph{Examples.}  The following example illustrates a field access expression constructor:

\lstinputlisting{../examples/ch6/eg_23.whiley}

The above function computes the so-called {\em dot product} of two vectors.  The field access operator is used to access the three fields of each vector. 

% =======================================================================
% Record Expression
% =======================================================================

\subsection{Record Initialisers}
\label{c_expr_record_initialiser}
A {\em record initialiser} accepts one or more operands and produces a value of record type.  Record constructors are used to construct records from their constituent elements.  

\begin{syntax}
  \verb+RecordInitialiserExpr+ & $::=$ & \token{\{}\ \verb+FieldArgsList+\ \token{\}}\\
&&\\
  \verb+FieldArgsList+ & $::=$ & \verb+Ident+ \token{:} \verb+Expr+ \big(\ \token{,} \verb+Ident+ \token{:} \verb+Expr+\ \big)$^*$\\
\end{syntax}

\paragraph{Example.} The following example illustrates a record initialiser:

\lstinputlisting{../examples/ch6/eg_22.whiley}

The above function simply translates a \lstinline{Point} from one position to another based on a shift in \lstinline{x} and in \lstinline{y}.  The record initialiser is used to construct the new \lstinline{Point}.

% =======================================================================
% Reference Expressions
% =======================================================================
\section{Reference Expressions}

Reference expressions operate on values of reference type (e.g. \lstinline{&int}).

% =======================================================================
% New Expression
% =======================================================================

\subsection{New Expressions}
\label{c_expr_new}

A new expression accepts an argument of any type and produces a reference to that type.  The {\em new operator} allocates sufficient space on the heap and initialises it with the given value.  It then returns a reference to this heap object.

\begin{syntax}
\verb+NewExpr+ & $::=$ & \token{new}\ \verb+Expr+\\
               & $|$ & \verb+Lifetime+\ \token{:}\ \token{new}\ \verb+Expr+\\
\end{syntax}

\paragraph{Example.} The following example illustrates the new operator:

\lstinputlisting{../examples/ch6/eg_24.whiley}

This example illustrates an operation for adding an item onto the front of a classical linked list.  Here, a \lstinline{LinkedList} is either \lstinline{null} or a reference to a node containing a \lstinline{next} reference and \lstinline{data} item.  The add operation simply allocates a new node and places it on the front of the list.

% =======================================================================
% Dereference Expressions
% =======================================================================

\subsection{Dereference Expressions}
\label{c_expr_dereference}

A dereference expression accepts an argument of reference type and returns a value (or element) of the reference's target type.  The {\em dereference operator} returns the value referenced by the argument.  The {\em arrow operator} returns a field of the value referenced by the argument.

\begin{syntax}
  \verb+DereferenceExpr+ & $::=$ & \token{*}\ \verb+TermExpr+\\
  & $|$ & \verb+Expr+\ \token{->}\ \verb+Ident+\\
\end{syntax}

\paragraph{Example.} The following illustrates the dereference operator:

\lstinputlisting{../examples/ch6/eg_14.whiley}

This method traverses a linked list counting the number of links it contains.  The arrow operator is used to access the next link in the chain.

\paragraph{Notes.} The arrow operation ``\lstinline{e->f}'' is a short-hand notation for ``\lstinline{(*e).f}'' and can be used when \lstinline{*e} has effective record type (\S\ref{c_types_effective_records}).


% =======================================================================
% Term Expressions
% =======================================================================

\section{Terminal Expressions}
\label{c_expr_term}

A {\em terminal expression} is one which can terminate an expression tree (though does not necessarily do so).  For example, a numeric literal represents a terminal node in an expression tree.

\begin{syntax}
  \verb+TermExpr+ & $::=$ & \verb+Ident+\\
  & $|$ & \verb+Literal+\\
  & $|$ & \token{(}\ \verb+Expr+\ \token{)}\\
\end{syntax}

% =======================================================================
% Term Expressions
% =======================================================================

\section{Type Test Expressions}
\label{c_expr_type_tests}

\begin{syntax}
  \verb+TypeTestExpr+ & $::=$ & \verb+Expr+\;\token{is}\;\verb+Type+\\
\end{syntax}

