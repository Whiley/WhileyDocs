\newcommand*{\glossfirstformat}[1]{\textit{#1}}
\usepackage[toc,xindy]{glossaries}
\makeglossary
\usepackage[xindy]{imakeidx}
\makeindex
\renewcommand{\glsdisplayfirst}[4]{\glossfirstformat{#1#4}}

% ===============================================
% A
% ===============================================

% ===============================================
% B
% ===============================================

% ===============================================
% C
% ===============================================

% ===============================================
% D
% ===============================================

% ===============================================
% E
% ===============================================

\newglossaryentry{expression}{
  name={expression},
  description={A combination of constants, variables and operators that, when evaluated, produce a single value.  Expressions in certain circumstances may have side effects.}
}

% ===============================================
% F
% ===============================================

% ===============================================
% G
% ===============================================

% ===============================================
% H
% ===============================================

% ===============================================
% I
% ===============================================

% ===============================================
% J
% ===============================================

% ===============================================
% K
% ===============================================

% ===============================================
% L
% ===============================================

% ===============================================
% M
% ===============================================

% ===============================================
% N
% ===============================================

% ===============================================
% O
% ===============================================

% ===============================================
% P
% ===============================================

% ===============================================
% Q
% ===============================================

% ===============================================
% R
% ===============================================

% ===============================================
% S
% ===============================================

% ===============================================
% T
% ===============================================

\newglossaryentry{type}{
  name={type},
  description={An descriptor for a set of values, typically used to determine the set of values a given variable or \gls{expression} may hold.}
}

% ===============================================
% U
% ===============================================

% ===============================================
% V
% ===============================================

\newglossaryentry{variable_declaration}{
  name={variable declaration},
  description={A statement which declares one or more variable(s) for use in a given scope.  Each variable is given a \gls{type} which limits the possible values it may hold, and may not already be declared in an enclosing scope.}
}

\newglossaryentry{variable_initialiser}{
  name={variable initialiser},
  description={An optional \gls{expression} used to initialise variable(s) declared as part of a \gls{variable_declaration}.}
}

% ===============================================
% W
% ===============================================

% ===============================================
% X
% ===============================================

% ===============================================
% Y
% ===============================================

% ===============================================
% Z
% ===============================================

