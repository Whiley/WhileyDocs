\newcommand*{\glossfirstformat}[1]{\textit{#1}}
%\usepackage[toc,xindy]{glossaries}
\usepackage[toc]{glossaries}
\makeglossary
%\usepackage[xindy]{imakeidx}
%\makeindex
\renewcommand{\glsdisplayfirst}[4]{\glossfirstformat{#1#4}}

% ===============================================
% A
% ===============================================

\newglossaryentry{abrupt_termination}{
  name={abrupt termination},
  description={A statement terminates abruptly if a subexpression
    causes an exception to be thrown.  This includes exceptions thrown (and not caught) by an invoked function or method}
}

\newglossaryentry{assertion}{
  name={assertion},
  description={An assertion statement is specified with the \lstinline{assert} keyword and identifies a condition which must hold at that point for all possible executions}
}

\newglossaryentry{access_control}{
  name={access control},
  description={Mechanisms for restricting the visibility of named \glspl{declaration}}
}


% ===============================================
% B
% ===============================================

\newglossaryentry{boolean_expression}{
  name={boolean expression},
  description={An \gls{expression} which evaluates to a value of type \lstinline{bool}}
}

\newglossaryentry{block_comment}{
  name={block comment},
  description={A block comment begins with ``\lstinline{/*}'' and continues until the end-of-comment marker ``\lstinline{*/}''}
}

% ===============================================
% C
% ===============================================

\newglossaryentry{control_flow_graph}{
  name={control-flow graph},
  description={A directed graph representation of a block of code (e.g. a function or method body) with which one can reason about the set of possible \glspl{execution_path}}
}

\newglossaryentry{contractive}{
  name={contractive},
  description={A type is contractive if it does not describe an infinite series of self applications.}
}


\newglossaryentry{compile_time}{
  name={compile time},
  description={The point in time at which a given \gls{compilation_group} is compiled into binary form.}
}

\newglossaryentry{compilation_group}{
  name={compilation group},
  description={A group of one or more \gls{source_file}s being compiled together}
}

\newglossaryentry{compilation_unit}{
  name={compilation unit},
  description={A single unit of compilation.  In Whiley, this includes
    \gls{source_file}s and also binary \gls{wyil_file}s}
}

\newglossaryentry{constant_declaration}{
  name={constant declaration},
  description={A source-level declaration which associates a name with a constant expression.  The full name of the declared entity is determined from the package and name of the enclosing \gls{source_file}.}
}

% ===============================================
% D
% ===============================================

\newglossaryentry{default_package}{
  name={default package},
  description={The top-level package which has no name, and is considered to be a ``global'' package.}
}

\newglossaryentry{declaration}{
  name={declaration},
  description={A declaration defines a new named entity within its  enclosing \gls{source_file}.}
}

\newglossaryentry{declaration_modifier}{
  name={declaration modifier},
  description={A declaration modifier provides additional meaning to a \gls{declaration}.}
}

\newglossaryentry{definite_assignment}{
  name={definite assignment},
  description={Definite assignment is the process of checking that every variable is defined before being used.}
}

% ===============================================
% E
% ===============================================

\newglossaryentry{expression}{
  name={expression},
  description={A combination of constants, variables and operators that, when evaluated, produce a single value.  Expressions in certain circumstances may have side effects}
}

\newglossaryentry{execution_path}{
  name={execution path},
  description={A sequence of statements through a program, function or method which may be taken during execution.}
}

% ===============================================
% F
% ===============================================

\newglossaryentry{fault}{
  name={fault},
  description={A fault is raised when an unrecoverable error in the program occurs.  For a verified program, no faults are possible except to indicate an out-of-memory failure.}
}

\newglossaryentry{ffi}{
  name={foreign function interface},
  description={A mechanism provided to enable inter-operation between Whiley source files and source files written in other languages.}
}

\newglossaryentry{function_declaration}{
  name={function declaration},
  description={A source-level declaration which defines a named function.  The full name of the declared entity is determined from the package and name of the enclosing \gls{source_file}.}
}


% ===============================================
% G
% ===============================================

% ===============================================
% H
% ===============================================

% ===============================================
% I
% ===============================================

\newglossaryentry{indentation_syntax}{
  name={indentation syntax},
  description={A lexical organisation of \gls{source_file}s where indentation is significant and is used to group statements and blocks}
}

\newglossaryentry{intersection_type}{
  name={intersection type},
  description={A \gls{type} formed by combining two or more types
    together (e.g. \lstinline{[int]\&[any]}), such that it includes any value contained in both}
}

\newglossaryentry{infeasible_path}{
  name={infeasible path},
  description={A valid path through the \gls{control_flow_graph} of a function or method for which no valid parameter values exist which will let it be executed}
}

% ===============================================
% J
% ===============================================

% ===============================================
% K
% ===============================================

% ===============================================
% L
% ===============================================

\newglossaryentry{literal}{
  name={literal},
  description={A source-level entity which describes a value of primitive type}
}

\newglossaryentry{line_comment}{
  name={line comment},
  description={A line comment begins with ``\lstinline{//}'' and continues until the end of line}
}

\newglossaryentry{loop_invariant}{
  name={loop invariant},
  description={A \gls{boolean_expression} which must hold on every iteration of a loop}
}

% ===============================================
% M
% ===============================================

\newglossaryentry{method_declaration}{
  name={method declaration},
  description={A source-level declaration which defines a named method.  The full name of the declared entity is determined from the package and name of the enclosing \gls{source_file}.}
}

% ===============================================
% N
% ===============================================

\newglossaryentry{name_resolution}{
  name={name resolution},
  description={The process of determining the fully qualified name of an identifier within a \gls{source_file}.  Names are first resolved within the same source file, and then by searching the list of imported entities in reverse order}
}

\newglossaryentry{name_mangling}{
  name={name mangling},
  description={The process of encoding information (e.g. about type parameters) within the exported name of a declaration.}
}

\newglossaryentry{negation_type}{
  name={negation type},
  description={A \gls{type} formed from another (e.g. \lstinline{!int}), such that it includes
    any value not contained in the other}
}


% ===============================================
% O
% ===============================================

\newglossaryentry{overloading}{
  name={overloading},
  description={Overloading occurs when two entities in the same category exist with the same name, and is permitted only when their type allows for disambiguation.}
}

% ===============================================
% P
% ===============================================

\newglossaryentry{package}{
  name={package},
  description={A unit of hierarchical organisation within the Whiley namespace.}
}

\newglossaryentry{precondition}{
  name={precondition},
  description={A logical condition over the parameters of a function
    or method which must be true immediately prior to execution of
    that function or method.}
}

\newglossaryentry{postcondition}{
  name={postcondition},
  description={A logical condition over the parameters and returns of
    a function or method which must be true immediately after
    execution of that function or method.}  }

% ===============================================
% Q
% ===============================================

% ===============================================
% R
% ===============================================

% ===============================================
% S
% ===============================================

\newglossaryentry{safety_critical_system}{
  name={safety critical system},
  description={A system which operates in a high-risk setting where failure can lead to loss of life, injury, significant damage or environmental harm}
}

\newglossaryentry{side_effect}{
  name={side-effect},
  description={A side-effect refers to the mutation of state that existed before a function or method was called, or the production of external effects through I/O.  In Whiley, functions must be side-effect free, meaning they are not permitted to modify pre-existing state or interact through I/O}
}

\newglossaryentry{source_file}{
  name={source file},
  description={A file in which source code is located.  Source files
    for the Whiley programming language have the extension
    \lstinline{.whiley}.  In Whiley, source files must be compiled into a
    binary form before they can be executed.}
}

\newglossaryentry{statement}{
  name={statement},
  description={An program instruction which has an effect on the
    environment when executed, but does not produce a value}
}

\newglossaryentry{compound_statement}{
  name={compound statement},
  description={A statement (e.g. \lstinline{if}, \lstinline{while}, etc) which may contain blocks of other statements}
}

\newglossaryentry{statement_block}{
  name={statement block},
  description={A sequence of zero or more consecutive \gls{statement}s with the same indentation}
}

% ===============================================
% T
% ===============================================

\newglossaryentry{type}{
  name={type},
  description={An abstract entity which represents the set of values a given variable may hold, or a given \gls{expression} may evaluate to.}
}

\newglossaryentry{type_descriptor}{
  name={type descriptor},
  description={A source-level description of an underlying \gls{type}.  Unlike many languages, type descriptors and types are quite distinct in Whiley as, for example, two distinct descriptors may describe the same underlying type}
}

\newglossaryentry{type_declaration}{
  name={type declaration},
  description={A source-level declaration which associates a name with a \gls{type_descriptor}.  The full name of the declared entity is determined from the package and name of the enclosing \gls{source_file}.}
}

\newglossaryentry{type_pattern}{
  name={type pattern},
  description={A source-level description of an underlying \gls{type} (similar to a \gls{type_descriptor}) where one or more variables are associated with its subcomponent(s).}
}

% ===============================================
% U
% ===============================================

\newglossaryentry{union_type}{
  name={union type},
  description={A \gls{type} formed by combining two or more types
    together (e.g. \lstinline{int|null}), such that it includes any value contained in either}
}

% ===============================================
% V
% ===============================================

\newglossaryentry{value}{
  name={value},
  description={A value is an instance of a given type and permits a specific set of operations.  Examples include: the integer value \lstinline{1}; the list value \lstinline{[1,2]}; and the \lstinline{null} value.}
}


\newglossaryentry{variable_declaration}{
  name={variable declaration},
  description={A statement which declares one or more variable(s) for use in a given scope.  Each variable is given a \gls{type} which limits the possible values it may hold, and may not already be declared in an enclosing scope}
}

\newglossaryentry{variable_definition}{
  name={variable definition},
  description={A statement in which a variable is defined in its entirety, as opposed to a partial assignment of some part (e.g. field or array element).  This concept is important in the process of checking \gls{definite_assignment}}
}

\newglossaryentry{variable_initialiser}{
  name={variable initialiser},
  description={An optional \gls{expression} used to initialise variable(s) declared as part of a \gls{variable_declaration}}
}

\newglossaryentry{verifying_compiler}{
  name={verifying compiler},
  description={A compilers which employs automated mathematical and logical reasoning to check the correctness of the programs that it compiles}
}

\newglossaryentry{variable_use}{
  name={variable use},
  description={A statement in which a variable is directly referred to in an expression other than an \lstinline{LVal}.  This concept is important in the process of checking \gls{definite_assignment}}
}


% ===============================================
% W
% ===============================================


\newglossaryentry{wyil_file}{
  name={WyIL file},
  description={A compiled (i.e. binary) form of a Whiley \gls{source_file}}
}

% ===============================================
% X
% ===============================================

% ===============================================
% Y
% ===============================================

% ===============================================
% Z
% ===============================================

