\chapter{Lexical Structure}

This chapter specifies the lexical structure of the Whiley programming language.  Programs in Whiley are organised into one or more \gls{source_file}s written in Unicode.  The Whiley language uses \gls{indentation_syntax} to delimit blocks and statements, rather than curly-braces (or similar) as found in many other languages.  

\section{Line Terminators}
A Whiley compiler splits the sequence of (Unicode) input characters into lines by identifying {\em line terminators}:

\begin{syntax}
\verb+LineTerminator+ & $::=$ & \token{\textbackslash n} $|$ \token{\textbackslash r} $|$ \token{\textbackslash r}\ \token{\textbackslash n}\\
\end{syntax}

Here, \token{\textbackslash n} represents the ASCII character \verb+LF+ (\verb+0xA+), whilst \token{\textbackslash r} represents the ASCII character \verb+CR+ (\verb+0xD+).  The two characters \token{\textbackslash r} \token{\textbackslash n} taken together form one line terminator.

\section{Indentation}
\label{c_lex_indentation}
After splitting the input characters into lines, a Whiley compiler then identifies the {\em indentation} of each line.  This is necessary because Whiley employs \gls{indentation_syntax} meaning that indentation is significant in the meaning of Whiley programs.  

\begin{syntax}
\verb+Indentation+ & $::=$ & \^\ \big( \token{\textbackslash t} $|$ \token{ }\ \big)$^*$\\
\end{syntax}

Here, \^\ demarcates the start of a line and, hence, indentation may only occur at the beginning of a line.  Indentation may be compared using the $\le$ comparator, such that $i\le ir$ always holds (where $i$ is some indentation and $r$ is either empty or represents additional indentation).  In other words, some indentation $i$ is considered less-than-or-equal to another piece of indentation $ir$ which includes the first as a prefix.  This comparator is important for delimiting \gls{statement_block}s (\S\ref{c_stmts_blocks}).

\section{Comments}
There are two kinds of comments in Whiley: \gls{line_comment}s and \gls{block_comment}s:
\begin{lstlisting}
/* This is a block comment */
\end{lstlisting}
The above illustrates a block comment, which is all of the text between \token{/*} and \token{*/} inclusive.
\begin{lstlisting}
// This is a line comment
\end{lstlisting}
The above illustrates a line comment, which is all of the text from \token{//} up to the end-of-line.

\section{Identifiers}
An identifier is a sequence of one or more {\em letters} or {\em digits} which starts with a letter.
\begin{syntax}
\verb+Ident+ & $::=$ & \verb+Letter+\ \big(\ \verb+Letter+\ $|$\ \verb+Digit+\ \big)*\\
&&\\
\verb+Letter+ & $::=$ & \token{\_} $|$ \token{a} $|$ \ldots $|$ \token{z} $|$ \token{A} $|$ \ldots $|$ \token{Z}\\
&&\\
\verb+Digit+ & $::=$ & \token{0} $|$ \token{1} $|$ \token{2} $|$ \token{3} $|$ \token{4} $|$ \token{5} $|$ \token{6} $|$ \token{7} $|$ \token{8} $|$ \token{9}\\
\end{syntax}

Letters include lowercase and uppercase alphabetic characters (i.e. \lstinline+a-z+ and \lstinline+A-Z+) and the underscore (\lstinline+_+).

\section{Keywords}
The following strings are reserved for use as {\em keywords} and may not be used as identifiers:

\begin{syntax}
\verb+Keyword+ & $::=$ & \token{all} $|$ \token{any} $|$
\token{assert} $|$ \token{assume} $|$ \token{bool} $|$ \token{break}
$|$ \token{byte}\\

{\huge\strut} & $|$ & \token{case} $|$ \token{catch} $|$ \token{continue} $|$ \token{debug}\\
{\huge\strut} & $|$ & \token{default} $|$ \token{do} $|$ \token{else} $|$ \token{ensures} $|$ \token{export} $|$ \token{false}\\
{\huge\strut} & $|$ & \token{fail} $|$ \token{finite} $|$ \token{for} $|$ \token{function} $|$ \token{if} $|$ \token{import} $|$ \token{in} $|$ \token{int} \\
{\huge\strut} & $|$ & \token{is} $|$ \token{method} $|$ \token{native} $|$ \token{new} $|$ \token{no} $|$ \token{null} $|$ \token{package}\\
{\huge\strut} & $|$ &  \token{private} $|$ \token{protected} $|$ \token{public} $|$ \token{real} $|$ \token{requires}\\
{\huge\strut} & $|$ & \token{return} $|$ \token{skip} $|$ \token{some} $|$ \token{switch} $|$ \token{throw}\\
{\huge\strut} & $|$ & \token{throws} $|$ \token{total} $|$ \token{true} $|$ \token{try} $|$ \token{void} $|$ \token{where} $|$ \token{while}\\
\end{syntax}

The following strings are reserved for use as {\em keywords}, but may additionally be used as identifiers in certain contexts:
\begin{syntax}
\verb+KeywordIdentifier+ & $::=$ & \token{constant} $|$ \token{from} $|$ \token{type}\\
\end{syntax}


%\pagebreak

\section{Literals}

A \gls{literal} is a source-level entity which describes a value of primitive type (\S\ref{c_types_primitive_types}).

\begin{syntax}
\verb+Literal+ & $::=$ &  \verb+NullLiteral+ \\
  & $|$ & \verb+BoolLiteral+ \\
  & $|$ & \verb+ByteLiteral+ \\
  & $|$ & \verb+IntLiteral+ \\
  & $|$ & \verb+RealLiteral+ \\
  & $|$ & \verb+CharLiteral+ \\
  & $|$ & \verb+StringLiteral+ \\
\\
\end{syntax}

\subsection{Null Literal}

The \lstinline{null} type (\S\ref{c_types_null}) has a single value expressed as the \lstinline{null} literal.

\begin{syntax}
  \verb+NullLiteral+ & $::=$ & \token{null} \\
\end{syntax}


\subsection{Boolean Literals}

The \lstinline{bool} type (\S\ref{c_types_bool}) has two values expressed as the \lstinline{true} and \lstinline{false} literals.

\begin{syntax}
  \verb+BoolLiteral+ & $::=$ & \token{true} $|$ \token{false} \\
\end{syntax}


\subsection{Byte Literals}

The \lstinline{byte} type (\S\ref{c_types_byte}) has 256 values which are expressed as sequences of binary digits, followed by the suffix ``\lstinline{b}'' (e.g. \lstinline{0101b}).


\begin{syntax}
 \verb+ByteLiteral+ & $::=$ & \big(\ \token{0}\ $|$\ \token{1}\ \big)$^+$\ \token{b}\\
\end{syntax}

Byte literals do not need to contain exactly eight digits and, when fewer digits are given, are padded out to eight digits by appending zero's from the left (e.g. \lstinline{00101b} becomes \lstinline{00000101b}).


\subsection{Integer Literals}

An {\em integer literal} is a sequence of numeric or hexadecimal digits (e.g. \lstinline{123456}, \lstinline{0xffaf}, etc) corresponding to a value of \lstinline{int} type (\S\ref{c_types_int}).

\begin{syntax}
  \verb+IntLiteral+ & $::=$ & \big( \token{0}\ $|$\ \ldots\ $|$\ \token{9}\ \big)$^+$ \\
  & $|$ & \token{0} \token{x}\ \big( \token{0}\ $|$\ \ldots\ $|$\ \token{9}\ $|$\ \token{a}\ $|$\ \ldots\ $|$\ \token{f}\ $|$\ \token{A}\ $|$\ \ldots\ $|$\ \token{F}\ \big)$^+$\\
\end{syntax}

Since integer values in Whiley are of arbitrary size (\S\ref{c_types_int}), there is no limit on the size of an integer literal.

\subsection{Decimal Literals}

A {\em decimal literal} is a sequence of numeric digits separated by a period (e.g. \lstinline{1.0}, \lstinline{0.223}, \lstinline{12.55}, etc) corresponding to a value of \lstinline{real} type (\S\ref{c_types_real}).

\begin{syntax}
  \verb+RealLiteral+ & $::=$ & \big( \token{0}\ $|$\ \ldots\ $|$\ \token{9}\ \big)$^+$\ \token{.}\ \big( \token{0}\ $|$\ \ldots\ $|$\ \token{9}\ \big)$^+$ \\
\end{syntax}

\subsection{Character Literals}

A {\em character literal} is expressed as a single character or an escape sequence enclosed in single quotes (e.g. \lstinline{'c'}).  Character literals generate integer constants corresponding to Unicode code points, which is necessary because there is no native character type.

\begin{syntax}
  \verb+CharLiteral+ & $::=$ & \token{'}\ \big(\ \verb+Character+\ $|$ \verb+Escape+\ \big)\ \token{'} \\
\end{syntax}

\subsection{String Literals}

A {\em string literal} is expressed as a sequence of zero or more characters or escape sequences enclosed in double quotes (e.g. \lstinline{"Hello World"}).  String literals generate lists of integers corresponding to Unicode code points, which is necessary as there is no native string type.

\begin{syntax}
  \verb+StringLiteral+ & $::=$ & \token{"}\ \big(\ \verb+Character+\ $|$ \verb+Escape+\ \big)$^*$\ \token{"} \\
\end{syntax}
