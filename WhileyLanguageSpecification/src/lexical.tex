\chapter{Lexical Structure}

This chapter specifies the lexical structure of the Whiley programming language.  Programs in Whiley are organised into one or more \gls{source_file}s written in Unicode.  The Whiley language uses \gls{indentation_syntax} to delimit blocks and statements, rather than curly-braces (or similar) as found in many other languages.  

\section{Indentation}
\section{Blocks}
\section{Whitespace}
\section{Identifiers}
\section{Keywords}
The following strings are reserved for use as {\em keywords} and may not be used as identifiers:

\begin{syntax}
\verb+Keyword+ & $::=$ & \token{all} $|$ \token{any} $|$ \token{assert} $|$ \token{assume}\\
         & $|$ & \token{bool} $|$ \token{break} $|$ \token{byte}\\
         & $|$ & \token{case} $|$ \token{catch} $|$ \token{char} $|$ \token{constant} $|$ \token{continue}\\
         & $|$ & \token{debug} $|$ \token{default} $|$ \token{do}\\
         & $|$ & \token{else} $|$ \token{ensures} $|$ \token{export}\\
         & $|$ & \token{false} $|$ \token{for} $|$ \token{function}\\
         & $|$ & \token{if} $|$ \token{import} $|$ \token{in} $|$ \token{int} $|$ \token{is}\\
         & $|$ & \token{method} \\
         & $|$ & \token{native} $|$ \token{new} $|$ \token{no} $|$ \token{null}\\
         & $|$ & \token{package} $|$ \token{private} $|$ \token{protected} $|$ \token{public}\\
         & $|$ & \token{real} $|$ \token{requires} $|$ \token{return}\\
         & $|$ & \token{skip} $|$ \token{some} $|$ \token{string} $|$ \token{switch}\\
         & $|$ & \token{throw} $|$ \token{throws} $|$ \token{true} $|$ \token{try} $|$ \token{type} \\
         & $|$ & \token{void} \\
         & $|$ & \token{where} $|$ \token{while}\\
\end{syntax}

\section{Literals}

A \gls{literal} is a source-level entity which describes a value of primitive type (\S\ref{c_types_primitive_types}).

\begin{syntax}
\verb+Literal+ & $::=$ &  \verb+NullLiteral+ \\
  & $|$ & \verb+BoolLiteral+ \\
  & $|$ & \verb+ByteLiteral+ \\
  & $|$ & \verb+IntLiteral+ \\
  & $|$ & \verb+RealLiteral+ \\
  & $|$ & \verb+CharLiteral+ \\
  & $|$ & \verb+StringLiteral+ \\
\\
\end{syntax}

\subsection{Null Literal}

\begin{syntax}
  \verb+NullLiteral+ & $::=$ & \token{null} \\
\end{syntax}


\subsection{Boolean Literals}

\begin{syntax}
  \verb+BoolLiteral+ & $::=$ & \token{true} $|$ \token{false} \\
\end{syntax}


\subsection{Byte Literals}

\begin{syntax}
 \verb+ByteLiteral+ & $::=$ & \big(\ \token{0}\ $|$\ \token{1}\ \big)$^+$\ \token{b}\\
\end{syntax}


\subsection{Integer Literals}

\begin{syntax}
  \verb+IntLiteral+ & $::=$ & \big( \token{0}\ $|$\ \ldots\ $|$\ \token{9}\ \big)$^+$ \\
  & $|$ & \token{0} \token{x}\ \big( \token{0}\ $|$\ \ldots\ $|$\ \token{9}\ $|$\ \token{a}\ $|$\ \ldots\ $|$\ \token{f}\ $|$\ \token{A}\ $|$\ \ldots\ $|$\ \token{F}\ \big)$^+$\\
\end{syntax}


\subsection{Real Literals}

\begin{syntax}
  \verb+RealLiteral+ & $::=$ & \big( \token{0}\ $|$\ \ldots\ $|$\ \token{9}\ \big)$^+$\ \token{.}\ \big( \token{0}\ $|$\ \ldots\ $|$\ \token{9}\ \big)$^+$ \\
\end{syntax}

\subsection{Character Literals}

\begin{syntax}
  \verb+CharLiteral+ & $::=$ & \token{'}\ .\ \token{'} \\
\end{syntax}

\subsection{String Literals}
