\chapter{Source Files}
\label{c_source_files}
Whiley programs are split across one or more \gls{source_file}s which are compiled into \gls{wyil_file}s prior to execution. \Gls{source_file}s contain declarations which describe the functions, methods, data types and constants which form the program. \Gls{source_file}s are grouped together into coherent units called \gls{package}s.


\section{Compilation Units}
\label{c_source_files_compilation_units}
Two kinds of \gls{compilation_unit} are taken into consideration when compiling a Whiley \gls{source_file}: other \gls{source_file}s; and, binary \gls{wyil_file}s.  The Whiley Intermediate Language (WyIL) file format is described elsewhere, but defines a binary representation of a Whiley source file.  

\begin{syntax}
\verb+SourceFile+ & $::=$ & \big[ \verb+PackageDecl+ \big] \\
&  $\big($ & \verb+ImportDecl+\\
& $|$ & $($\verb+Modifier+$)^*$ \verb+TypeDecl+\\
& $|$ & $($\verb+Modifier+$)^*$ \verb+StaticVarDecl+\\
& $|$ & $($\verb+Modifier+$)^*$ \verb+FunctionDecl+\\
& $|$ & $($\verb+Modifier+$)^*$ \verb+MethodDecl+ \\
& $\big)^*$ &\\
\end{syntax}

When one or more Whiley source files are compiled together, a \gls{compilation_group} is formed.  External symbols encountered during compilation are first resolved from the compilation group, and then from previously compiled \gls{wyil_file}s.

\section{Packages}
\label{c_source_files_packages}

Programs in Whiley are organised into \gls{package}s to help reduce name conflicts and provide some grouping of related concepts.  A Whiley source file may provide an optional \lstinline{package} declaration to identify the package it belongs to.  This declaration must occur at the beginning of the source file.

\begin{syntax}
\verb+PackageDecl+ & $::=$ & \token{package}\ \verb+Ident+\ \big(\ \token{.}\ \verb+Ident+\ \big)$^*$\\
\end{syntax}

Any source file which does not provide a \lstinline{package} declaration is considered to be in the \gls{default_package}.

\section{Names}
\label{c_source_files_names}
There are four functional entities which can be defined within a Whiley source file: \gls{type_declaration}s (\S\ref{c_source_files_type_decl}), \gls{constant_declaration}s (\S\ref{c_source_files_constant_decl}), \gls{function_declaration}s (\S\ref{c_source_files_function_decl}) and \gls{method_declaration}s (\S\ref{c_source_files_method_decl}).  These define {\em named entities} which may be referenced from other \gls{compilation_unit}s.  Every named entity has a unique {\em fully-qualified} name constructed from the enclosing package name, the source file name and the declared name.  For example:\\

\noindent \verb+Graphics.whiley+

\lstinputlisting{../examples/ch3/eg_1.whiley}

This declares two entities: \lstinline{g2d.Graphics.Point} and \lstinline{g2d.Graphics.Origin}.  

\noindent Two named entities may {\em clash} if they have the same fully qualified name and are in the same category.  There are three entity categories: {\em types}, {\em constants} and {\em functions/methods}.  The following illustrates a common pattern:

\lstinputlisting{../examples/ch3/eg_2.whiley}

Here, two named entities share the same fully qualified named.  This is permitted because they are in different categories.

Two named entities in the same category with different types are permitted in some circumstances, and this is referred to as \gls{overloading}.  Currently, overloading is only supported for entities representing function and methods or function and method types.

\section{Imports}

When performing \gls{name_resolution}, the Whiley compiler first attempts to resolve names within the same source file.  For any remaining unresolved, the compiler examines imported entities in reverse declaration order.  Entities are imported using an \lstinline{import} declaration:

\begin{syntax}
\verb+ImportDecl+ & $::=$ & \token{import} \big[\ \verb+FromSpec+\ \big]\ \verb+Ident+\ \Big(\ \token{.} \ \big(\ \verb+Ident+\ $|$ \token{*}\big)\ \Big)$^*$\\
&&\\
\verb+FromSpec+ & $::=$ & \big(\ \verb+Ident+ $|$ \token{*} \big) \token{from}\\
\end{syntax}

A declaration of the form ``\lstinline{import a.pkg.File}'' imports the \gls{compilation_unit} ``\lstinline{File}'' in package ``\lstinline{a.pkg}''.  Named entities (e.g. ``\lstinline{Entity}'') within that compilation unit can then be referenced using a {\em partially qualified} name which omits the package component (e.g. ``\lstinline{File.Entity}'').  

A declaration of the form ``\lstinline{import Entity from a.pkg.File}'' imports the named entity ``\lstinline{Entity}'' from the \gls{compilation_unit} ``\lstinline{File}'' residing in package ``\lstinline{a.pkg}''.  Note, this does {\em not} import the compilation unit ``\lstinline{a.pkg.File}'' (and, hence, does not subsume the statement ``\lstinline{import a.pkg.File}'').

A {\em wildcard} may be used in place of the compilation unit name to import {\em all} compilation units within the given package (e.g. ``\lstinline{import some.pkg.*}'').  A {\em wildcard} may be used in place of the entity name (e.g. ``\lstinline{import * from some.pkg.File}'') to import {\em all} named entities within the given compilation unit.  

% A {\em package match} may be used in place of some of all of the package component (e.g. ``\lstinline{import some..File}'') to import all compilation units with matching name and package {\em prefix} and/or {\em suffix}.

\section{Declarations}

A \gls{declaration} defines a new entity within a Whiley source file and provides a {\em name} by which it can be referred to within this source file, or from other source files.

\subsection{Access Control}

Several mechanisms for \gls{access_control} are provided through \glspl{declaration_modifier}.

\begin{syntax}
\verb+Modifier+ & $::=$ & \token{public}\ $|$\ \token{private}\ $|$\ \token{native}\ $|$\ \token{export}\ $|$\ \token{final}\\
\end{syntax}

\begin{itemize}
\item The \lstinline{public} modifier declares that the \gls{declaration} is visible from other Whiley source files.
\item The \lstinline{private} modifier declares that the \gls{declaration} is visible only within the enclosing Whiley source file.
\item The \lstinline{native} modifier declares that the \gls{declaration} is provided by the underlying system.
\item The \lstinline{export} modifier declares that the \gls{declaration} is visible to source files written in other languages.  Declarations with this modifier cannot be overloaded.
\item The \lstinline{final} modifier declares that the \gls{declaration} cannot be reassigned.
\end{itemize}

\noindent When no modifier is given, the default of \lstinline{private} is assumed.  

\paragraph{Notes.} The \lstinline{native} and \lstinline{export} modifiers together form the \gls{ffi}.  The restriction on declarations declared with the \lstinline{export} modifier is to enable names to be exported without \gls{name_mangling}.

% =======================================================================
% Type Declarations
% =======================================================================

\subsection{Type Declarations}
\label{c_source_files_type_decl}

A {\em type declaration} declares a named type within a Whiley \gls{source_file}.  The declaration may refer to named types in this or other source files and may also {\em recursively} refer to itself (either directly or indirectly).

\begin{syntax}
  \verb+TypeDecl+ & $::=$ & \token{type}\ \verb+Ident+\ \token{is}\
  \big[\ \verb+Type+\ $|$\ \token{(}\ \verb+Variable+\ \token{)}\ \big]\ \big(\ \token{where}\ \verb+Expr+\ \big)$^*$\\
  &&\\
  \verb+Variable+ & $::=$ & \verb+Type+\ \verb+Ident+
\end{syntax}

The optional \lstinline{where} clause defines a \gls{boolean_expression} which holds for any instance of this type.  This is often referred to as the type {\em invariant} or {\em constraint} which ranges over the declared variable (if provided).

\paragraph{Examples.}  Some simple examples illustrating type
declarations are:

\lstinputlisting{../examples/ch3/eg_3.whiley}

The first declaration defines an unconstrained record type named \lstinline{Point}, whilst the second defines a constrained integer type \lstinline{nat}.

\paragraph{Notes.}  A convention is that type declarations for {\em records} or {\em unions of records} begin with an upper case character (e.g. \lstinline{Point} above).  All other type declarations begin with lower case.  This reflects the fact that records are most commonly used to describe objects in the domain.  All types are also required to be \gls{contractive}.  This means, for example, that the declaration \lstinline+type X is X+ is considered invalid.



% =======================================================================
% Static Declarations
% =======================================================================

\subsection{Static Variable Declarations}
\label{c_source_files_constant_decl}

A {\em static variable declaration} declares a top-level variable
within a Whiley \gls{source_file} with an optional initialiser
expression.

\begin{syntax}
  \verb+StaticVarDecl+ & $::=$ & \verb+Type+ \verb+Ident+\ \big[= \verb+Expr+\big]\\
\end{syntax}

The given {\em initialiser expression} may not directly or indirectly
refer to itself and may only be omitted in conjunction with the
\lstinline{native} modifier.  Initialiser expressions are also {\em
  pure} and may not have \glspl{side_effect} (i.e. invoke methods or
allocate on the heap).

\paragraph{Examples.}  Some simple examples to illustrate static
variable declarations are:

\lstinputlisting{../examples/ch3/eg_4.whiley}

The first declaration defines the constant \lstinline{TEN} to have the
\lstinline{int} value \lstinline{10}.  The second declaration defines
a global variable initialised with the constant.

\paragraph{Notes.}  Since initialiser expressions across a compilation
group form a directed acyclic graph, static variables can always be
safely initialised.

% =======================================================================
% Function Declarations
% =======================================================================

\subsection{Function Declarations}
\label{c_source_files_function_decl}

A {\em function declaration} defines a function within a Whiley
\gls{source_file}.  Functions are {\em pure} and may not have
\glspl{side_effect}.  This means they are guaranteed to return the
same result given the same arguments, and are permitted within
specifications (i.e. in type invariants, \gls{loop_invariant}s, and
function/method \gls{precondition}s or \gls{postcondition}s).
Functions may call other functions, but may not call other methods.
Functions may not explicitly allocate memory on the heap and/or
instigate concurrent computation.

\begin{syntax}
  \verb+FunctionDecl+ & $::=$ & \token{function}\ \verb+Ident+\ \token{(}\ \verb+Parameters+\ \token{)}\ \token{->}\ \token{(}\ \verb+Parameters+\ \token{)}\\
  && \big(\ \token{requires}\ \verb+Expr+\ $|$\ \token{ensures}\ \verb+Expr+\ \big)$^*$\ \token{:}\ \verb+Block+\\
  &&\\
  \verb+Parameters+ & $::=$ & \big[ \verb+Variable+ \big( \token{,}\ \verb+Variable+ \big)$^*$\ \big]
\end{syntax}

Those variables declared before ``\lstinline{->}'' are referred to as the {\em parameters}, whilst those declared afterwards are referred to as the {\em returns}.  There are two kinds of optional clause which follow:

\begin{itemize}
% \item {\bf Throws clause}. This defines the exceptions which may be thrown by this function. Multiple clauses may be given, and these are taken together as a union. Furthermore, the convention is to specify \lstinline{throws} clause(s) first.

\item {\bf Requires clause(s)}. These define constraints on the permissible values of the parameters on entry to the function, and are often collectively referred to as the \gls{precondition}. These expressions may refer to any declared parameters.  Multiple clauses may be given, and these are taken together as a conjunction.  The convention is to specify the \lstinline{requires} clause(s) before any \lstinline{ensures} clause(s).

\item {\bf Ensures clause(s)}. These define constraints on the permissible values of the function's return value, and are often collectively referred to as the \gls{postcondition}. These expressions may refer to any declared parameters or returns.  Multiple clauses may be given, and these are taken together as a conjunction.  The convention is to specify \lstinline{ensures} clause(s) after \lstinline{requires} clause(s).
\end{itemize}

\paragraph{Examples.}
The following function declaration provides a small example to illustrate:

\lstinputlisting{../examples/ch3/eg_5.whiley}

This defines the specification and implementation of the well-known \lstinline{max()} function which returns the largest of its parameters. This does not enforce any preconditions on its parameters.

% =======================================================================
% Method Declarations
% =======================================================================

\subsection{Method Declarations}
\label{c_source_files_method_decl}

A {\em method declaration} defines a method within a Whiley \gls{source_file}.  Methods are {\em impure} and may have side-effects.  Thus, they cannot be used within specifications (i.e. in type invariants, \gls{loop_invariant}s, and function/method \gls{precondition}s or \gls{postcondition}s).  However, unlike functions, methods may call other functions and/or methods (including \lstinline{native} methods).  They may also explicitly allocate memory on the heap, and/or instigate concurrent computation.

\begin{syntax}
  \verb+MethodDecl+ & $::=$ & \token{method}\ \big[\ \verb+LifetimeParameters+\ \big]\ \verb+Ident+\\
  && \token{(}\ \verb+Parameters+\ \token{)}\ \big[\ \token{->}\ \token{(}\ \verb+Parameters+\ \token{)}\ \big]\\
  && \big(\ \token{requires}\ \verb+Expr+\ $|$\ \token{ensures}\ \verb+Expr+\ \big)$^*$\ \token{:}\ \verb+Block+\\
  &&\\
  \verb+LifetimeParameters+ & $::=$ & \token{<}\ \verb+Ident+ \big(\ \token{,}\ \verb+Ident+\ \big)$^*$\ \token{>}\\
\end{syntax}

Those variables declared before ``\lstinline{->}'' are referred to as the {\em parameters}, whilst those declared afterwards are referred to as the {\em returns}.  The two optional clauses are defined identically as for function declarations~(\S\ref{c_source_files_function_decl}).

\paragraph{Examples.}  The following method declaration provides a
small example to illustrate:

\lstinputlisting{../examples/ch3/eg_6.whiley}




