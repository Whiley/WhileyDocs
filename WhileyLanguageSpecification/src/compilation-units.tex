\chapter{Source Files}
\label{c_source_files}
Whiley programs are split across one or more \gls{source_file}s which
are compiled into \gls{wyil_file}s prior to execution.
\Gls{source_file}s contain declarations which describe the functions,
methods, data types and constants which form the program.
\Gls{source_file}s are grouped together into coherent units called
\gls{package}s.


\section{Compilation Units}
\label{c_source_files_compilation_units}
Two kinds of \gls{compilation_unit} are taken into consideration when compiling a Whiley \gls{source_file}: other \gls{source_file}s; and, binary \gls{wyil_file}s.  The Whiley Intermediate Language (WyIL) file format is described elsewhere, but defines a binary representation of a Whiley source file.  When one or more Whiley source files are compiled together, a \gls{compilation_group} is formed.  External symbols encountered during compilation are first resolved from the compilation group, and then from previously \gls{wyil_file}s.

\section{Packages}
\label{c_source_files_packages}

Programs in Whiley are organised into \gls{package}s to help reduce name conflicts and provide some grouping of related concepts.  A Whiley source file may provide an optional \lstinline{package} declaration to identify the package it belongs to.  This declaration must occur at the beginning of the source file.

\begin{syntax}
\verb+PackageDecl+ & $::=$ & \token{package}\ \verb+Ident+\ \big(\ \token{.}\ \verb+Ident+\ \big)$^*$\\
\end{syntax}

Any source file which does not provide a \lstinline{package} declaration is considered to be in the \gls{default_package}.

\section{Names}
\label{c_source_files_names}
There are four functional entities which can be defined within a Whiley source file: \gls{type_declaration}s (\S\ref{c_source_files_type_decl}), \gls{constant_declaration}s (\S\ref{c_source_files_constant_decl}), \gls{function_declaration}s (\S\ref{c_source_files_function_decl}) and \gls{method_declaration}s (\S\ref{c_source_files_method_decl}).  These define {\em named entities} which may be referenced from other \gls{compilation_unit}s.  Every named entity has a unique {\em fully-qualified} name constructed from the enclosing package name, the source file name and the declared name.  For example:\\
\pagebreak

\noindent \verb+Graphics.whiley+
\begin{lstlisting}
package tracer

type Point is { int x, int y }

constant Origin is { x: 0, y: 0 } 
\end{lstlisting}
This declares two named entities: \lstinline{tracer.Graphics.Point} and \lstinline{tracer.Graphics.Origin}.  

Two named entities may {\em clash} if they have the same fully qualified name and are in the same category.  There are three entity categories: {\em types}, {\em constants} and {\em functions/methods}.  The following illustrates a common pattern:

\begin{lstlisting}
type Point is { int x, int y }

function Point(int x, int y) => Point:
    return {x: x, y: y}
\end{lstlisting}

Here, two named entities share the same fully qualified named.  This is permitted because they are in different categories.


\section{Imports}

When performing \gls{name_resolution}, a Whiley compiler first attempts to resolve names within the same source file.  For any remaining unresolved, the compiler examines imported entities in reverse declaration order.  Entities are imported using an \lstinline{import} declaration:

\begin{syntax}
\verb+ImportDecl+ & $::=$ & \token{import} \Big[\ \verb+FromSpec+\ \Big]\ \verb+Ident+\ \Big(\ \big(\ \token{.} $|$ \token{..}\ \big)\ \big(\ \verb+Ident+\ $|$ \token{*}\big)\ \Big)$^*$\\
&&\\
\verb+FromSpec+ & $::=$ & \big(\ \verb+Ident+ $|$ \token{*} \big) \token{from}\\
\end{syntax}

A declaration of the form ``\lstinline{import some.pkg.File}'' imports the \gls{compilation_unit} ``\lstinline{File}'' residing in package ``\lstinline{some.pkg}''.  Named entities (e.g. ``\lstinline{Entity}'') within that compilation unit can then be referenced using a {\em partially qualified} name which omits the package component (e.g. ``\lstinline{File.Entity}'').  A declaration of the form ``\lstinline{import Entity from some.pkg.File}'' imports the named entity ``\lstinline{Entity}'' from the \gls{compilation_unit} ``\lstinline{File}'' residing in package ``\lstinline{some.pkg}''.  Note, this does {\em not} import the compilation unit ``\lstinline{some.pkg.File}'' (and, hence, does not subsume the statement ``\lstinline{import some.pkg.File}'').

A {\em wildcard} may be used in place of the compilation unit name (e.g. ``\lstinline{import some.pkg.*}'') to import {\em all} compilation units within the given package.  A {\em package match} may be used in place of some of all of the package component (e.g. ``\lstinline{import some..File}'') to import all compilation units same name and matching package {\em prefix} and/or {\em suffix}.  A {\em wildcard} may be used in place of the entity name (e.g. ``\lstinline{import * from some.pkg.File}'') to import {\em all} named entities within the given compilation unit.  

\section{Declarations}

Camel case

\subsection{Access Control}

% =======================================================================
% Type Declarations
% =======================================================================

\subsection{Type Declarations}
\label{c_source_files_type_decl}

A {\em type declaration} declares a named type within a Whiley
\gls{source_file}.  The declaration may refer to named types in this
or other source filess and may also {\em recursively}
refer to itself (either directly or indirectly).

\begin{syntax}
  \verb+TypeDecl+ & $::=$ & \token{type}\ \verb+Ident+\ \token{is}\
  \verb+TypePattern+\ \big[\ \token{where}\ \verb+Expr+\ \big]\\
\end{syntax}

The optional \lstinline{where} clause defines a
\gls{boolean_expression} which holds for any instance of this type.
This is often referred to as the type {\em invariant} or {\em
  constraint}.  Variables declared within the {\em type pattern} may be
referred to within the optional \lstinline{where} clause.

\paragraph{Examples.}  Some simple examples illustrating type
declarations are:

\begin{lstlisting}
// Define a simple point type
type Point is { int x, int y }

// Define the type of natural numbers
type nat is (int x) where x >= 0
\end{lstlisting}

The first declaration defines an unconstrained record type named
\lstinline{Point}, whilst the second defines a constrained integer
type \lstinline{nat}.

\paragraph{Notes.}  A convention is that type declarations for {\em
  records} or {\em unions of records} begin with an upper case
character (e.g. \lstinline{Point} above).  All other type declarations
begin with lower case.  This reflects the fact that records are most
commonly used to describe objects in the domain.

% =======================================================================
% Constant Declarations
% =======================================================================

\subsection{Constant Declarations}
\label{c_source_files_constant_decl}

A {\em constant declaration} declares a named constant within a Whiley
\gls{source_file}.  The declaration may refer to named constants in this
or other source filess, although it may not refer to itself (either
directly or indirectly).

\begin{syntax}
  \verb+ConstantDecl+ & $::=$ & \token{constant}\ \verb+Ident+\
  \token{is}\ \verb+Expr+\\
\end{syntax}

The given {\em constant expression} is evaluated at {\em compile time}
and must produce a constant value.  This prohibits the use of function
or method calls within the constant expression.  However, general
operators (e.g. for arithmetic) are permitted.

\paragraph{Examples.}  Some example to illustrate constant
declarations are:

\begin{lstlisting}
// Define the well-known mathematical constant to 10 decimal places.
constant PI is 3.141592654

// Define a constant expression which is twice PI
constant TWO_PI is PI * 2.0
\end{lstlisting}

The first declaration defines the constant \lstinline{PI} to have the
\lstinline{real} value \lstinline{3.141592654}.  The second
declaration illustrates a more interesting constant expression which
is evaluated to \lstinline{6.283185308} at compile time.

\paragraph{Notes.}  A convention is that constants are named in upper
case with underscores separating words (i.e. as in \lstinline{TWO_PI}
above).

% =======================================================================
% Function Declarations
% =======================================================================

\subsection{Function Declarations}
\label{c_source_files_function_decl}

A {\em function declaration} defines a function within a Whiley
\gls{source_file}.  Functions are {\em pure} and may not have
side-effects.  This means they are guaranteed to always return the
same result given the same arguments, and are permitted within
specifications (i.e. in type invariants, \gls{loop_invariant}s, and
function/method \gls{precondition}s or \gls{postcondition}s).
Functions may call other functions, but may not call other methods.
They also may not allocate memory on the heap and/or instigate
concurrent computation.

\begin{syntax}
  \verb+FunctionDecl+ & $::=$ & \token{function}\ \verb+Ident+\
  \verb+TypePattern+\ \token{=>}\ \verb+TypePattern+\ \big(\\
  && \ \ \token{throws}\ \verb+Type+\ $|$\ \token{requires}\
  \verb+Expr+\ $|$\ \token{ensures}\ \verb+Expr+\\
  && \big)$^*$\ \token{:}\ \verb+Block+\\
\end{syntax}

The first type pattern (i.e. before "\lstinline{=>}") is referred to
as the {\em parameter}, whilst the second is referred to as the {\em
  return}.  There are three kinds of optional clause which follow:

\begin{itemize}
\item {\bf Throws clause}. This defines the exceptions which may be
  thrown by this function. Multiple clauses may be given, and these
  are taken together as a union. Furthermore, the convention is to
  specify the throws clause before the others.

\item {\bf Requires clause(s)}. These define constraints on the
  permissible values of the parameters on entry to the function or
  method, and are often collectively referred to as the
  \gls{precondition}. These expressions may refer to any variables
  declared within the parameter type pattern. Multiple clauses may be
  given, and these are taken together as a conjunction. Furthermore,
  the convention is to specify the requires clause(s) before any
  ensure(s) clauses.

\item {\bf Ensures clause(s)}. These define constraints on the
  permissible values of the the function or method's return value, and
  are often collectively referred to as the \gls{postcondition}. These
  expressions may refer to any variables declared within either the
  parameter or return type pattern.  Multiple clauses may be given,
  and these are taken together as a conjunction. Furthermore, the
  convention is to specify the requires clause(s) after the others.
\end{itemize}

\paragraph{Examples.}
The following function declaration provides a small example to
illustrate:

\begin{lstlisting}
function max(int x, int y) => (int z)
// return must be greater than either parameter
ensures x <= z && y <= z
// return must equal one of the parmaeters
ensures x == z || y == z:
    // implementation
    if x > y:
        return x
    else:
        return y
\end{lstlisting}

This defines the specification and implementation of the well-known
\lstinline{max()} function which returns the largest of its
parameters. This does not throw any exceptions, and does not enforce
any preconditions on its parameters.

% =======================================================================
% Method Declarations
% =======================================================================

\subsection{Method Declarations}
\label{c_source_files_method_decl}

A {\em method declaration} defines a method within a Whiley
\gls{source_file}.  Methods are {\em impure} and may have
side-effects.  Thus, they cannot be used within specifications
(i.e. in type invariants, \gls{loop_invariant}s, and function/method
\gls{precondition}s or \gls{postcondition}s).  However, unlike
functions, they methods call other functions and/or methods (including
\lstinline{native} methods).  They may also allocate memory on the
heap, and/or instigate concurrent computation.

\begin{syntax}
  \verb+MethodDecl+ & $::=$ & \token{method}\ \verb+Ident+\
  \verb+TypePattern+\ \token{=>}\ \verb+TypePattern+\ \big(\\
  && \ \ \token{throws}\ \verb+Type+\ $|$\ \token{requires}\
  \verb+Expr+\ $|$\ \token{ensures}\ \verb+Expr+\\
  && \big)$^*$\ \token{:}\ \verb+Block+\\
\end{syntax}

The first type pattern (i.e. before "\lstinline{=>}") is referred to
as the {\em parameter}, whilst the second is referred to as the {\em
  return}.  The three optional clauses are defined identically as for
functions above.

\paragraph{Examples.}  The following method declaration provides a
small example to illustrate:

\begin{lstlisting}
// Define the well-known concept of a linked list
type LinkedList is null | { &LinkedList next, int data }

// Define a method which inserts a new item onto the end of the list
method insertAfter(&LinkedList list, int item):
    if *list is null:
        // reached the end of the list, so allocate new node
        *list = new { next: null, data: item }
    else:
        // continue traversing the list
        insertAfter(list->next, item)
\end{lstlisting}




