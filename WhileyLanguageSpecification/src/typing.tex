\newcommand{\tenv}{\Gamma}
\newcommand{\ftype}[3]{#1\vdash #2\dashv #3}

\chapter{Type Checking}
The Whiley programming language is {\em statically typed}, meaning
that: firstly, every expression has a type determined at compile time;
second, evaluating an expression is guaranteed to yield a value of its
type.  Whiley's {\em type system} governs how the type of any variable
or expression is determined.  Whiley's type system is unusual in that
it operates in a {\em flow-sensitive} manner allowing variables to
have different types at different program points.

\section{Overview}

A {\em type environment}, ${\tt\tenv}$, binds variables declared in
the enclosing scope(s) to their {\em current type}.  The current type
of a variable may be its declared type, or a refinement thereof.  The
environment ${\tt\tenv[x\mapsto T]}$ contains all of the bindings in
${\tt\tenv}$, except where ${\tt x}$ now binds to ${\tt T}$.  The
initial type environment, ${\tt\tenv_0}$, for the
\lstinline{requires}, \lstinline{where} clause(s) and body of a
\lstinline{function}, \lstinline{method} or \lstinline{type}
declaration contains exactly one binding for each parameter to its
declared type.  The initial type environment, ${\tt\tenv_r}$, for the
\lstinline{ensures} clause(s) of a \lstinline{function} or
\lstinline{method} additionally contains exactly one binding for each
return to its declared type.  For example, consider the following
partial declaration:
\begin{lstlisting}
function f(int x, bool y) -> (null|int r):
    ...
\end{lstlisting}
Here, ${\tt\tenv_0=\{x\mapsto int, y\mapsto bool\}}$ and
${\tt\tenv_r=\{x\mapsto int, y\mapsto bool, r\mapsto(int\lor null)\}}$.

\subsection{Flow Typing}

Whiley's type system employs flow-sensitive typing --- {\em flow
  typing} --- for determining the type of each local variable within a
given statement block.  The {\em pre-environment} gives the type
environment immediately before a given statement.  Likewise, the {\em
  post-environment} gives the type environment immediately after a
given statement.  The flow typing system is responsible for
calculating, for each statement, the post-environment from the
pre-environment.  The judgment ${\tt\ftype{\tenv_0}{S}{\tenv_1}}$
indicates that typing statement ${\tt S}$ with environment
${\tt\tenv_0}$ produces the (potentially updated) environment
${\tt\tenv_1}$.  For a given statement block ${\tt S_1\ldots S_n}$, it
follows that ${\tt\ftype{\tenv_0}{S_1\ldots S_n}{\tenv_n}}$ expands by
chaining the post-environment for each statement ${\tt S_n}$ into its
successor ${\tt S_{n+1}}$.  That is,
${\tt\ftype{\tenv_0}{S_1}{\tenv_1}}$,
${\tt\ftype{\tenv_1}{S_2}{\tenv_2}}$, and so on.


\subsection{Scoping}

The {\em lifetime} of a local variable extends from its declaration
within a given statement block to the end of that block.  For example,
if statement ${\tt S}$ declared variable ${\tt x}$ to have type
${\tt T}$, it follows that
${\tt \ftype{\tenv}{S}{\tenv[x\mapsto T]}}$.  Furthermore, we require
that ${\tt x}$ was not already declared in ${\tt\tenv}$ (i.e. that
${\tt x\not\in\tenv}$).  Observe that variables of the same name may
be declared in different blocks, provided one is not nested within the
other.

\subsection{Environment Joining}

At meet points in the control-flow graph of a statement block the
typing environments from each branch must be {\em joined} together. If
${\tt\tenv_a}$ and ${\tt\tenv_b}$ are type environments then their
join, denoted ${\tt\tenv_a\sqcup\tenv_b}$, is a single environment
carefully constructed from them.  This join operator is defined as
follows:
\begin{displaymath}
\begin{array}{rcl}
{\tt \tenv_a\sqcup\tenv_b} & = & {\tt \{ x\mapsto(T_a\lor T_b)\;|\;x\!\mapsto\!T_a\in\tenv_a\;\land\;x\!\mapsto\!T_b\in\tenv_b\}}\\
\end{array}
\end{displaymath}

Every variable defined in both environments is bound in their join to
the union of its type in each environment.  The following illustrates
a situation where joining is necessary:

\begin{lstlisting}
function f(int|null x) -> (int r):
   //
   if x is null:
      return 0
   else:
      x = x + 1
   //
   return x
\end{lstlisting}

The pre-environment for the \lstinline{return} statement is formed
from the post-environments of the true- and false-branches of the
conditional.  The former is ${\tt\{x\mapsto void\}}$ and the latter is
${\tt\{x\mapsto int\}}$ and the resulting join is
${\tt\{x\mapsto int\}}$.

\section{Type Refinement}

In certain circumstances a runtime type test may result in {\em type
  refinement}.  That is, where the type of a variable is refined from
its current type (e.g. ${\tt T_1}$) to a more precise type (e.g.
${\tt T_2}$ where ${\tt T_2\le T_1}$).  More specifically when a type
test ``\lstinline{e is T$_2$}'' holds, the type of \lstinline{e} may
be refined to ${\tt T_1\land T_2}$.  Likewise, when ``\lstinline{e is T$_2$}'' 
does not hold, the type of ${\tt e}$ may be refined to
${\tt T_1\land\neg T_2}$.  Type refinement may only occur when a type
test is used as the conditional expression for an \lstinline{if},
\lstinline{while}, \lstinline{do-while}, \lstinline{assert} or
\lstinline{assume} statement.  Furthermore, an expression
\lstinline{e} will be refined only if it is a
\gls{refinable_expression}.  A refinable expression is a variable access or
a field access acting on a refinable expression.  The following
illustrates a common scenario:

\begin{lstlisting}
function f(int|null x) -> (int r):
   //
   if x is int:
      return x
   else:
      return 0
\end{lstlisting}

The initial environment for the body of \lstinline{f()} is given by
${\tt\tenv_0=\{x\mapsto(int\lor null)\}}$.  In this case, the type of
variable ${\tt x}$ is refined to ${\tt (int\lor null)\land int}$ in
the true branch (which is equivalent to ${\tt int}$) and
${\tt (int\lor null)\land\neg int}$ in the false branch (which is
equivalent to ${\tt null}$) .

\subsection{Expressions}

Type refinement may occur within expressions when a given type test is
known to hold or not.  The following illustrates:

\begin{lstlisting}
if x is int && x >= 0:
   //
else:
   //
\end{lstlisting}

Here, the type of variable \lstinline{x} is refined to \lstinline{int}
within \lstinline{x >= 0}.  Observe, however, that no refinement
occurs on the \lstinline{else} branch as the given expression does not
capture all possible integer values.

Since the logical connectives have short-circuiting behaviour, so does
type refinement within expressions.  That is, the refinement must
occur before the expression where the refined type is required.

\section{Function and Method Resolution}
Look at the rules for determining which function or method is being selected.

\begin{itemize}
\item Most precise type selected
\item If no unique precise type, then ambiguous

\end{itemize}


\section{Coercions}
Look at the rules for when a coercion is permitted or not.
