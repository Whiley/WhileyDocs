\chapter{Definite Assignment}
\label{c_definite_assignment}

The Whiley programming language requires that variables are known at \gls{compile_time} to be {\em definitely assigned} (i.e. that they are defined before used).  A conservative approach is taken to determining whether or not this is the case.  This ensures the language can be compiled efficiently, but also means that some provably safe programs are not valid Whiley programs.  In this chapter, we specify the process by which definite assignment is determined.

\section{Overview}

The following illustrates a simple function which will be rejected by the compiler because it cannot determine definite assignment for all variables.  The function is said to {\em fail definite assignment}:

\begin{lstlisting}
function f(int x) -> (int r):
   int y
   //
   if x < 0:
       y = 1
   //
   return x + y
\end{lstlisting}

In the above program, variable \lstinline{y} is {\em not} definitely assigned before its use in the \lstinline{return} statement.  This is because there is an \gls{execution_path} through the function which reaches the \lstinline{return} statement and on which variable \lstinline{y} is not defined.  In fact, there are {\em two} possible execution paths through this function, but variable \lstinline{y} is only defined on one of them.  Observe that, since it is a parameter, variable \lstinline{x} is automatically considered to have been defined on entry to the function.

\subsection{Loops}

The treatment of loops with respect to definite assignment warrants special attention.  Recall that the mechanism for determining definite assignment is conservative.  In the context of loops, the effect of this is that it simply assumes {\em every loop can be executed zero or more times} (i.e. even if this is not correct).  The following illustrates a simple example:

\begin{lstlisting}
function f(int x) -> int:
   int y
   //
   while x < 0:
       y = 1
       x = x + 1
   //
   return x + y
\end{lstlisting}

The above function fails definite assignment because variable \lstinline{y} is not defined when zero iterations of the loop are executed (e.g. when \lstinline{x==0} on entry).  To illustrate the conservative nature of the definite assignment mechanism, we can refined the above example as follows:

\begin{lstlisting}
function ten() -> int:
   int x = 0
   int y
   //
   while x < 10:
       y = 1
       x = x + 1
   //
   return y
\end{lstlisting}

In this case, it can be shown that \lstinline{y==1} must hold when the \lstinline{return} statement is reached.  Nevertheless, this function fails definite assignment because the mechanism naively assumes that the loop will execute {\em zero} or more iterations when, in fact, it will never execute zero iterations.

\subsection{Infeasible Paths}

Functions and methods may contain \glslink{infeasible_path}{infeasible paths} which are valid execution paths that, in practice, cannot be executed.  The mechanism for checking definite assignment assumes for simplicity that any valid path can be executed.  This means that some programs will fail definite assignment, even though they can be shown as safe.   The following illustrates such a program:

\begin{lstlisting}
function abs(int x) -> int:
    int y
    //
    if x >= 0:
        y = x
    //
    if x < 0:
        y = -x
    //
    return y
\end{lstlisting}

This function contains four valid execution paths which can be denoted by \lstinline{ff}, \lstinline{tf}, \lstinline{ft}, \lstinline{tt} where, for example, \lstinline{tf} represents the path where the first condition evaluates to \lstinline{true} and the second to \lstinline{false}.  However, it is easy to see that the execution paths \lstinline{ff} and \lstinline{tt} are infeasible.  Furthermore, we can see that on the other two paths, \lstinline{tf} and \lstinline{ft}, the variable \lstinline{y} is definitely assigned when the \lstinline{return} statement is reached.  Despite this, the above function fails definite assignment because the mechanism considers all valid paths whilst ignoring infeasible execution paths.

\subsection{Partial Assignments}

A variable will never be considered definitely assigned after the application of one or more {\em partial assignments}.  The following illustrates a program which fails definite assignment even though it can be shown as safe:

\begin{lstlisting}
type Point is {int x, int y}

function Point(int x, int y) -> Point:
    Point p
    p.x = x
    p.y = y
    return p
\end{lstlisting}

In the above program, the variable \lstinline{p} can be shown as definitely assigned at the \lstinline{return} statement.  Nevertheless, the conservative mechanism for checking definite assignment will reject this program because variable \lstinline{p} is initialised via partial assignments.  

\section{Description}

Definite assignment is defined over the \gls{control_flow_graph} over a block of code, such as a \lstinline{requires} or \lstinline{ensures} clause, or the body of a function or method.  Appendix~\S\ref{c_cfg} details the process for constructing a control-flow graph from a block of code.

Definite assignment is defined over the set of all {\em valid paths} through a \gls{control_flow_graph}.  A valid path is a path through the graph starting from its root.  Every vertex in the graph is associated with a set of variables which are {\em defined} at that vertex, as well as those which are {\em used} at that vertex.  A variable $x$ is said to be {\em definitely assigned} on entry to a vertex $v$ if, for every valid path which includes $v$, some ancestor vertex $u$ exists on which $x$ is defined.

\subsection{Variable Definitions and Uses}

Discussion of what are definite assignment statements.  In particular, update operations are not definitely assignments, neither are assignments through references.

Parameters are defined on entry to a function or method.  However, returns are not.

Discussion of parameters as being considered definitely assigned on entry to a function or method.

\section{Other}

See error reported for this check~\S\ref{c_err_var_uninitialised}.

