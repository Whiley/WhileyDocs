\chapter{Errors and Warnings}

When the Whiley compiler encounters an invalid program it will report an {\em error}.  In contrast, when it encounters something undesirable in an otherwise valid program, it may report a {\em warning}.  This chapters details the complete list of error messages and warnings which can be reported for a Whiley program.

\section{Overview}

\section{Parse Errors}

\section{Declarations}

Declarations are top-level entities in a \gls{source_file}, and their syntax is defined in \S\ref{c_source_files}.

\subsection{``Cyclic Constant Declaration'' (E301)}

A {\em cyclic constant declaration} occurs when a \gls{constant_declaration} refers to itself, either {\em directly} or {\em indirectly}.  This is an error because constants must be evaluated at \gls{compile_time}.

\paragraph{Example.}  The following illustrates several cyclic constant declarations:

\begin{lstlisting}
constant const1 is 1 + const1

constant const2 is 1 + const3
constant const3 is 1 + const2
\end{lstlisting}
Here, all three constant declarations are cyclic.  The declaration for \lstinline{const1} has a {\em direct} cycle, because its definition refers to itself.  The declaration for \lstinline{const2} has an indirect cycle, because its definition refers to \lstinline{const3} which, in turn, refers back to \lstinline{const2}.

\subsection{``Reference Not Permitted in Function'' (E302)}

A {\em reference not permitted in function} error occurs when an attempt is made to declare or use a variable of reference type in a function (as opposed to a method).  Functions in Whiley must be free from \glspl{side_effect} --- i.e. they must be {\em pure}.  Thus, the potential side-effects made possible through the use of references is not permitted.

\paragraph{Example.}  The following illustrates a very simple example:

\begin{lstlisting}
function f(&int x) -> (&int r):
   return x
\end{lstlisting}

Here, function \lstinline{f()} accepts a parameter \lstinline{x} of reference type \lstinline{&int}, which is not permitted.  In this case the function does not, in fact, exhibit any side-effects; nevertheless, the function will currently be rejected.

\subsection{``Reference Operation Not Permitted in Function'' (E303)}

A {\em reference operation not permitted in function} error occurs when an attempt is made to operate on a variable of reference type in a function (as opposed to a method).  Functions in Whiley must be free from \glspl{side_effect} --- i.e. they must be {\em pure}.  Thus, the potential side-effects made possible through the use of references is not permitted.

\paragraph{Example.}  The following illustrates a very simple example:

\begin{lstlisting}
function f(int x) -> (int r):
   int y = x
   return *(&y)
\end{lstlisting}

Here, function \lstinline{f()} obtains a reference to local variable \lstinline{y} and then immediately dereferences it, neither of which is permitted.  In this case the function does not, in fact, exhibit any side-effects; nevertheless, the function will currently be rejected.

\subsection{``Method Invocation Not Permitted In Function'' (E304)}

A {\em method invocation not permitted in function} error occurs when an attempt is made to call a method from a function (as opposed to another method).  Functions in Whiley must be free from \glspl{side_effect} --- i.e. they must be {\em pure}.  Thus, the potential side-effects made possible through the method call are not permitted.

\paragraph{Example.}  The following illustrates a very simple example:

\begin{lstlisting}
method g(int x) -> (int y):
   return x

function f(int x) -> (int r):
   return g(x)
\end{lstlisting}

Here, function \lstinline{f()} accepts a parameter \lstinline{x} and passes it through a call to method \lstinline{g()}.  In this case method \lstinline{g()} does not, in fact, exhibit any side-effects; nevertheless, the method call will be rejected.

\subsection{``Insufficient Return Values'' (E305)}
An {\em insufficient return values} error occurs when a \lstinline{return} statement is encountered which does not provide as many return values as declared by the enclosing function or method.  

\paragraph{Example.}  The following illustrates two simple examples:

\begin{lstlisting}
method g(int x) -> (int y, int z):
   return x+1

function f(int x) -> int:
   return
\end{lstlisting}

Here, method \lstinline{g()} is required to return {\em two} values but only one is actually being returned.  Likewise, function \lstinline{f()} is required to return one value but none are actually being returned.

\subsection{``Too Many Return Values'' (E306)}
A {\em too many return values} error occurs when a \lstinline{return} statement is encountered which provides more return values than declared by the enclosing function or method.  

\paragraph{Example.}  The following illustrates two simple examples:

\begin{lstlisting}
method g(int x):
   return x

function f(int x) -> int:
   return x,x+1
\end{lstlisting}

Here, method \lstinline{g()} is required to return {\em zero} values but one is actually being returned.  Likewise, function \lstinline{f()} is required to return one value but two are actually being returned.

\section{Types}
\label{c_err_types}

\subsection{``Subtype Error'' (401)}

A {\em subtype} error arises when an attempt is made to use an expression of type \lstinline{T} in a position where an expression of type \lstinline{S} is expected, and \lstinline{T} is not a subtype of \lstinline{S}.  This is a common error precisely because it can occur in a large number of different situations.

\paragraph{Example.}  The following illustrates an example:

\begin{lstlisting}
function f(int x) -> (int r):
    if x:
        return 0
    else:
        return 1
\end{lstlisting}

Here, variable \lstinline{x} has type \lstinline{int} but is being used in a position (i.e. as the condition of the \lstinline{if}-statement) which expects a type \lstinline{bool}.  Another example is as follows:

\begin{lstlisting}
function g(int[] items) -> (int r):
    return items
\end{lstlisting}

Here, variable \lstinline{items} has type \lstinline{int[]} but is being used in a position (i.e. as the return value) which expects a type \lstinline{int}.

\subsection{``Incomparable Operands'' (402)}

\subsection{``Record Type Required'' (403)}

\subsection{``Record Missing Field'' (404)}


\section{Statements}

Statements are used frequently in a Whiley program, and their syntax is defined elsewhere (see~\ref{c_statements}).  The error messages reported in this section are those related to specific statement forms.  Other, more general, errors can also be reported for a statement (e.g. {\em type errors}, \S\ref{c_err_types}) and are discussed elsewhere.

\subsection{``Invalid LVal'' (E501)}

An {\em invalid lval} error occurs when an invalid expression is used on the left-hand side of an assignment.  Only expressions which are also lval's maybe used in such a situation  (see~\S\ref{c_stmt_assign}).

\paragraph{Example.}  The following illustrates two invalid lval's:

\begin{lstlisting}
function f(int x):
   1 = x   // constant not valid lval
   x+1 = x // arithmetic expression not valid lval\end{lstlisting}

The first assignment statement is invalid because one cannot assign to a constant.  The second is invalid because one cannot assign to an arithmetic expression.

\subsection{``Invalid Destructuring LVal'' (E502)}

An {\em invalid destructuring lval} error occurs when a destructuring assignment is used on the left-hand side, but the right-hand side returns an incorrect number of values.

\paragraph{Example.}  The following illustrates an invalid destructuring LVal:

\begin{lstlisting}
function f(int x, int y) -> int:
    return x+y

function g(int x, int y) -> int:
    x,y = f(x,y)
    return x - y
\end{lstlisting}

Here, the invocation of \lstinline{f()} in \lstinline{g()} uses an invalid destructuring assignment because \lstinline{f()} returns one value, but the assignment expects two.

\subsection{``Variable Already Defined'' (E503)}

A {\em variable redefinition} error occurs when a variable is declared with a name matching another variable already in scope.  This is an error because it is not permitted for one variable to shadow another.

\paragraph{Example.}  The following illustrates an example of a variable redefinition:

\begin{lstlisting}
function sum(int[] items) => int[]:
    int i = 0
    int r = 0
    //
    while i < |items|:
       //
       int r = items[i]
       i = i + 1
    //
    return r
\end{lstlisting}

Here, the \lstinline{while} loop attempts to declare a variable \lstinline{r}, but another variable \lstinline{r} was already declared beforehand.

\subsection{``Unreachable Code'' (E504)}

An {\em unreachable statement} error arises in a function or method when no possible execution path could reach them.  

\paragraph{Example.} The following illustrates some unreachable code:

\begin{lstlisting}
function abs(int x) -> int:
    //
    if x < 0:
        return -x
    else:
        return x
    //
    return 0 // unreachable
\end{lstlisting}

Here, the final \lstinline{return} statement can never be reached by any execution path through the \lstinline{abs()} function.  This is considered an error because it indicates something undesirable which may not have been intended.

\subsection{``Branch Always Taken'' (E506)}

A {\em branch always taken} error occurs when a conditional branch is determine to always evaluate to either \lstinline{true} or \lstinline{false}.  

\paragraph{Example.} The following illustrates a branch always taken:

\begin{lstlisting}
function f(int x) -> int:
   if x is int:
       return x
   else:
       return -1
\end{lstlisting}

Here, the condition \lstinline{x is int} always evaluates to \lstinline{true} and, hence, the true branch of this conditional is always taken, whilst the false branch is never taken.

\subsection{``Break Outside of Loop'' (E507)}

A {\em break outside loop} error occurs when a \lstinline{break} statement is given which is not contained within one or more loops.  This is an error because the break statement is used specifically to exit a loop early, and must be contained within the loop to be exited.  

\paragraph{Example.}  The following illustrates a break outside of a loop:

\begin{lstlisting}
function f(int x) -> int:
   break
   return x
\end{lstlisting}

Here, the \lstinline{break} statement is meaningless as it is not associated with a loop.

\subsection{``Duplicate Default Label'' (E508)}
A {\em duplicate default label} error occurs when a \lstinline{switch} statement includes more than one \lstinline{default} label.  This is an error because at most one \lstinline{default} is permitted.

\paragraph{Example.}  The following illustrates an example of a duplicate \lstinline{default} label:

\begin{lstlisting}
function f(int x):
    switch x:
        case 0:
            return 0
        default:
            return 1
        default:
            return 2
\end{lstlisting}

Here, the \lstinline{switch} statement has two \lstinline{default} labels.  This must be an error as, otherwise, it would be ambiguous as to which executed.

\subsection{``Duplicate Case Label'' (E509)}
A {\em duplicate case label} error occurs when a \lstinline{switch} statement includes more than one \lstinline{case} label matching the same value.  This is an error because at most one \lstinline{case} matching a given value is permitted.

\paragraph{Example.}  The following illustrates an example of a duplicate \lstinline{case} label:

\begin{lstlisting}
function f(int x):
    switch x:
        case 0:
            return 0
        case 0,1:
            return 1
        default:
            return 2
\end{lstlisting}

Here, the \lstinline{switch} statement has two \lstinline{case} labels, both of which match the value \lstinline{0}.  This must be an error as, otherwise, it would be ambiguous as to which executed.


\section{Expressions}

%\subsection{``Invalid File Access'' (000)}

%\subsection{``Invalid Package Access'' (000)}

%\subsection{``Resolution Error'' (000)}

Expressions typically form the bulk of a Whiley program, and their syntax is defined elsewhere (see~\S\ref{c_expressions}).  The error messages reported in this section are those related to specific expression forms.  More general errors can also be reported for an expression, such as {\em type errors} (see~\S\ref{c_err_types}).

\subsection{``Variable Possibly Uninitialised'' (E601)}
\label{c_err_var_uninitialised}
A {\em variable possibly uninitialised} error occurs when a variable may be used without being defined.  That is, when a simple path exists through the control-flow graph of a function or method from that variable's declaration to a use which contains no definition for that variable.  This error is reported as part of the {\em definite assignment} checking performed during compilation (see~\S\ref{c_definite_assignment}).

\paragraph{Example.}  The following illustrates a variable which is possibly uninitialised:

\begin{lstlisting}
function f(int x) => int:
   int y
   return x + y
\end{lstlisting}

Here, variable \lstinline{y} is definitely uninitialised in the expression ``\lstinline{x + y}''.  For more examples of variables which are possibly uninitialised, see~\S\ref{c_definite_assignment}.

\subsection{``Unknown Variable'' (E602)}

An {\em unknown variable} error occurs when an attempt is made to access a variable which has not been declared in the current scope.  All variables must be declared before they can be used.

\paragraph{Example.}  The following illustrates an unknown variable:

\begin{lstlisting}
function f(int x) -> int:
   return x+y
\end{lstlisting}

Here, the \lstinline{return} statements refers to an unknown variable \lstinline{y}.  In contrast, the reference to variable \lstinline{x} is valid because \lstinline{x} has been declared within scope.

\subsection{``Unknown Function or Method'' (E603)}

An {\em unknown function or method} error occurs when an attempt is made to access a function or method which is not visible in the current scope.  Functions and methods which are not declared in the same file as the invocation can be brought into the current scope using \lstinline{import} statements.  
\paragraph{Example.}  The following illustrates an invocation of an unknown function or method:

\begin{lstlisting}
function f(int x) -> int:
   return g(x)
\end{lstlisting}

Here, function \lstinline{g()} is not defined in the current source file and has not been brought into scope through an \lstinline{import} statement.

\subsection{``Ambiguous Coercion'' (E604)}

An {\em ambiguous coercion} error occurs when the target of a cast expression is uncertain.  That is, when attempting to cast a value to a given type \lstinline{T}, but there is more than one way this can be achieved.  This error is reported as part of the {\em coercion check} performed during compilation.

\paragraph{Example.}  The following illustrates an ambiguous coercion:

\begin{lstlisting}
type Ambiguous is { int f1, real f2} | { real f1, int f2 }

function f(int x, int y) -> Ambiguous:
   return (Ammbiguous) {f1: x, f2: y}
\end{lstlisting}

The cast is ambiguous here because it's unclear whether, for example, \lstinline|{f1: 1, f2: 2}| should become \lstinline|{f1: 1.0, f2: 2}| or \lstinline|{f1: 1, f2: 2.0}|.

