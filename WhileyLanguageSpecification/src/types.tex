\chapter{Types}
\section{Overview}
Discuss syntactic versus semantic types.
\section{Primitives}

% =======================================================================
% Any
% =======================================================================

\subsection{Any Type}

\begin{syntax}
  AnyType & $::=$ & \token{any} & {\em // type any}\\
\end{syntax}

\paragraph{Description.}

\paragraph{Examples.}

\paragraph{Semantics.}

\paragraph{Notes.} 

% =======================================================================
% Void 
% =======================================================================

\subsection{Void Type}

\begin{syntax}
   VoidType & $::=$ & \token{void} & {\em // type void}\\
\end{syntax}

\paragraph{Description.} The \lstinline{void} type represents the type whose variables cannot exist! That is, they cannot hold any possible value. Void is used to represent the return type of a function which does not return anything. However, it is also
used to represent the element type of an empty list of set. 

\paragraph{Examples.}

\paragraph{Semantics.}

\paragraph{Notes.} The void type is a subtype of everything; that is, it is bottom in the type lattice.

% =======================================================================
% Null
% =======================================================================

\subsection{Null Type}

\begin{syntax}
  NullType & $::=$ & \token{null} & {\em // type null}\\
\end{syntax}

\paragraph{Description.}

\paragraph{Examples.}

\paragraph{Semantics.}

\paragraph{Notes.} 

% =======================================================================
% Bool 
% =======================================================================

\subsection{Bool Type}

\begin{syntax}
  BoolType & $::=$ & \token{bool} & {\em // type bool}\\
\end{syntax}

\paragraph{Description.}

\paragraph{Examples.}

\paragraph{Semantics.}

\paragraph{Notes.} 

% =======================================================================
% Char
% =======================================================================

\subsection{Char Type}

\begin{syntax}
  CharType & $::=$ & \token{char} & \\
\end{syntax}

\paragraph{Description.}

\paragraph{Examples.}

\paragraph{Semantics.}

\paragraph{Notes.} 

% =======================================================================
% Int
% =======================================================================

\subsection{Int Type}

\begin{syntax}
  IntType & $::=$ & \token{int} &\\
\end{syntax}

\paragraph{Description.}

\paragraph{Examples.}

\paragraph{Semantics.}

\paragraph{Notes.} 

% =======================================================================
% Real
% =======================================================================

\subsection{Real Type}

\begin{syntax}
  RealType & $::=$ & \token{real} &\\
\end{syntax}

\paragraph{Description.}

\paragraph{Examples.}

\paragraph{Semantics.}

\paragraph{Notes.} 

% =======================================================================
% Collections
% =======================================================================

\section{Collection Types}

% =======================================================================
% Set
% =======================================================================

\subsection{Set Type}

\begin{syntax}
  SetType & $::=$ & \token{\{} \ Type \ \token{\}} &\\
\end{syntax}

\paragraph{Description.}

\paragraph{Examples.}

\paragraph{Semantics.}

\paragraph{Notes.} 

% =======================================================================
% Map
% =======================================================================

\subsection{Map Type}

\begin{syntax}
  MapType & $::=$ & \token{\{} \ Type \ \token{=>} \ Type \ \token{\}} &\\
\end{syntax}

\paragraph{Description.}

\paragraph{Examples.}

\paragraph{Semantics.}

\paragraph{Notes.} 

% =======================================================================
% List
% =======================================================================

\subsection{List Type}

\begin{syntax}
  ListType & $::=$ & \token{[} \ Type \ \token{]}&\\
\end{syntax}

\paragraph{Description.}

\paragraph{Examples.}

\paragraph{Semantics.}

\paragraph{Notes.} 

% =======================================================================
% Unions
% =======================================================================

\section{Union Types}

\begin{syntax}
  UnionType & $::=$ & IntersectionType\ \big(\ \token{|}\ IntersectionType\
  \big)$^+$&\\
\end{syntax}

\paragraph{Description.}

\paragraph{Examples.}

\paragraph{Semantics.}

\paragraph{Notes.}

% =======================================================================
% Intersections
% =======================================================================

\section{Intersection Types}

\begin{syntax}
  IntersectionType & $::=$ & TermType\ \big(\ \token{\&}\ TermType\
  \big)$^+$&\\
\end{syntax}

\paragraph{Description.}

\paragraph{Examples.}

\paragraph{Semantics.}

\paragraph{Notes.}

% =======================================================================
% Negations
% =======================================================================

\section{Negation Types}

\begin{syntax}
  NegationType & $::=$ & \token{!}\ \ Type&\\
\end{syntax}

\paragraph{Description.}

\paragraph{Examples.}

\paragraph{Semantics.}

\paragraph{Notes.}

% =======================================================================
% Negations
% =======================================================================

\section{Reference Types}

\begin{syntax}
  \verb+ReferenceType+ & $::=$ & \token{\&}\ \ \verb+Type+&\\
\end{syntax}

\paragraph{Description.}

\paragraph{Examples.}

\paragraph{Semantics.}

\paragraph{Notes.}

\section{Subtyping}
Discussion or present subtyping algorithm?
