\chapter{Introduction}

The Whiley programming language was developed from scratch to simplify
the process of {\em verifying software correctness}.  The language
allows one to specify a function's permitted behaviour through the use
of {\em preconditions} (i.e. constraints in input values) and {\em
  postconditions} (i.e. constraints on output values).  The Whiley
Compiler (WyC) is used to check that the implementation of a function
meets its specification.  WyC is referred to as a
\gls{verifying_compiler} which ``{\em uses automated mathematical and
  logical reasoning to check the correctness of the programs that it
  compiles}''~\cite{Hoare03}.  As an example, Whiley would be ideally
suited for use in \gls{safety_critical_system}s.  A critical aspect of
Whiley is the use of complex algorithms for reasoning about programs
and their correctness.  The {\em Whiley Theorem Prover (WyTP)} is the
component responsible for this in the Whiley Compiler.

WyTP is an interactive and automated theorem prover designed for
discharging verification conditions generated by the Whiley Compiler.
In this context, it can be compared with an SMT solver such as
Simplify~\cite{DNS05} or Z3~\cite{MB08}.  WyTP is closely tied with
Whiley to mitigate any potential for an impedance mismatch.  WyTP can
be used in a standalone fashion and includes a domain-specific
language, called the {\em Whiley Assertion Language (WyAL)}, for this
purpose.  WyAL is a variant of first-order logic which has been
extended with theories necessary for reasoning about Whiley.
Furthermore, the Whiley Compiler can be configured to emit
verification conditions in WyAL and this can be extremely helpful for
debugging.

The Whiley Theorem Prover is released under an open source license
(Apache 2.0) and its source code can be obtained
online\footnote{\url{https://github.com/Whiley/WhileyTheoremProver}}.
The intention of this document is to provide both a reference to the
Whiley Assertion Language, as well as providing a technical description
of how WyTP operates.

