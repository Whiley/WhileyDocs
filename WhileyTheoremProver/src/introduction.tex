\chapter{Introduction}

The Whiley programming language is designed to produce programs with
as few errors as possible.  Whiley allows explicit specifications to
be given for functions, methods and data structures, and employs a
\gls{verifying_compiler} to check whether programs meet their
specifications.  For example, Whiley would be ideally suited for use
in \gls{safety_critical_system}s.  A critical aspect of Whiley is the
use of complex algorithms for reasoning about programs and their
correctness.  The {\em Whiley Theorem Prover (WyTP)} is a component of
the Whiley Compiler (WyC) which is responsible for this.

WyTP is an interactive and automated theorem prover designed for
discharging verification conditions generated by the Whiley Compiler.
In this context, it can be compared with an SMT solver such as
Simplify~\cite{DNS05} or Z3~\cite{MB08}.  However, WyTP is very
closely tied with the Whiley language in order to mitigate any
potential impedance mismatch between Whiley and the language of
verification conditions.

WyTP can be used in a standalone fashion and includes a
domain-specific language, called the {\em Whiley Assertion Language
  (WyAL)}, for this purpose.  WyAL is a variant of first-order logic
which has been extended with theories necessary for reasoning about
Whiley.  Furthermore, the Whiley Compiler can be configured to emit
verification conditions in WyAL and this can be extremely helpful for
debugging.

The Whiley Theorem Prover is released under an open source license
(Apache 2.0) and its source code can be obtained
online\footnote{\url{https://github.com/Whiley/WhileyTheoremProver}}.
The intention of this document is to provide both a reference to the
Whiley Assertion Language, as well as providing a technical description
of how WyTP operates.

